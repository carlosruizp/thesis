% Chapter 2

\chapter{Multi-Task Learning} % Write in your own chapter title
\label{Chapter2}
\lhead{Chapter \ref{Chapter2}. 
\emph{Gaussian Processes And Approximate Inference}} % Write in your own chapter title to set the page header

{\bf \small{
This chapter presents\dots
}}

\section{Introduction}
% What is MTL
% Notation

% Examples and Motivation

% Some important references
% Caruana, Baxter... (ver review de Caruana para ver más)



%%%%%%%%%%%%%%%%%%%%%%%%%%%%%%%%%%%%%%%%%%%%%%%%%%%%%%%%%%%%%%%%%%%%%%%%%%%%%%%%%%%%%
  %%%%%%%%%%%%%%%%%%%%%%%%%%             SECTION         %%%%%%%%%%%%%%%%%%%%%%%%%%
%%%%%%%%%%%%%%%%%%%%%%%%%%%%%%%%%%%%%%%%%%%%%%%%%%%%%%%%%%%%%%%%%%%%%%%%%%%%%%%%%%%%%

\section{Multi-Task Learning Methods: An Overview}

\subsection{Feature-Based MTL}
% Feature learning or
%   Feature transformation (relacion con Deep Learning)
The feature-based methods try to find a set of features that are useful for all tasks. Two main approaches are taken: Feature Learning, which tries to learn new features from the original ones, and Feature Selection which selects a subset of the original features.
Deep Learning is a particular case of Feature Learning that has had a tremendous relevance and that is treated separately. 

\subsubsection*{Feature Learning approach}


% Caruana R. Multitask learning. 1997 (multi-task feedforward NN)
% Multi-Task feature Learning 2006
% Convex multi-task feature learning 2008
% A spectral regularization framework for multi-task structure learning 2007
Apart from the Multi-Task feedforward Neural Network first shown in~\cite{Caruana97}, which will be described later, the first work of Multi-Task Feature Learning is presented in~\cite{ArgyriouEP06}. Argyriou et. al assume a multi-task linear model in some RKHS $\rkhs$, where that the task parameters $w_t$ lie in a linear subspace, i.e. $\mymat{w} = \mymat{u}\mymat{a}$ where
$w_t$ are the columns of $\mymat{w}$, $\mymat{u}$ is an orthogonal matrix and $\mymat{a}$ is a row-sparse matrix. The minimization problem is
\begin{equation}
    \label{eq:mtl_feat_learning}
    \argmin_{\mymat{u} \in \reals^{d \times d}, \mymat{a} \in \reals^{d \times \ntasks}} \sum_{r=1}^\ntasks \sum_{i=1}^{\npertask_r} \lossf(y_i^r, \dotp{\mymat{u} a_r}{x_i^r}) + \lambda \norm{\mymat{a}^\intercal}_{2, 1}^2 \text{ s.t. } \mymat{u}^\intercal \mymat{u} = \mymat{i} .
\end{equation}
Here, the $L_{2, 1}$ regularizer is used to impose row-sparsity across tasks, i.e. forcing some rows of $\mymat{a}$ to be zero, and the matrix $\mymat{u}$ is restricted to be orthonormal.
Although problem~\eqref{eq:mtl_feat_learning} is not jointly convex in $\mymat{u}$ and $\mymat{a}$, in~\cite{ArgyriouEP06} and~\cite{ArgyriouEP08} is shown to be equivalent to the convex problem
\begin{equation}
    \label{eq:convmtl_feat_learning}   
    \begin{aligned}
        &\argmin_{\mymat{w} \in \reals^{d \times \ntasks}, \mymat{d}  \in \reals^{d \times d}}  \sum_{r=1}^\ntasks \sum_{i=1}^{\npertask_r} \lossf(y_i^r, \dotp{w_r}{x_i^r}) + \lambda \sum_{r=1}^\ntasks \dotp{w_r}{D^{-1}w_r} \\ &\text{s.t.} \; \mymat{d} \succeq 0,\; \Tr{\mymat{d}} \leq 1 ,\; \Span{\mymat{w}} \subseteq \Span{\mymat{d}}.
    \end{aligned}
\end{equation}
Here $\mymat{d}^+$ denotes the pseudo-inverse of $\mymat{d}$ and $\mymat{d} \succeq 0$ restricts the $\mymat{d}$ to the semidefinite positive matrices. Problem~\eqref{eq:convmtl_feat_learning} is solved using a two-step optimization algorithm. First they fix $\mymat{d}$ and solve for $\mymat{w}$; while in the second step, with fixed $\mymat{w}$, there exists a closed solution for $\mymat{d}^* = \left(\mymat{w}^\intercal \mymat{w}\right)^\frac{1}{2} / \Tr\left( \left(\mymat{w}^\intercal \mymat{w}\right)^\frac{1}{2} \right)$.
% It is shown that solving~\eqref{eq:mtl_feat_learning} with a perturbed version of $ \mymat{W} \mymat{W}^\intercal $, i.e. $ \mymat{W} \mymat{W}^\intercal + \epsilon I$, is equivalent to solving
%
The regularizer of~\eqref{eq:convmtl_feat_learning} can be expressed as $\Tr \left( \mymat{w}^\intercal \mymat{d}^+ \mymat{w} \right)$ and by plugging $\mymat{d}^*$ in this formula we obtain the squared-trace norm regularizer for $\mymat{w}$.
%
In~\cite{Maurer09} some bounds on the excess risks are given for this Multi-Task Feature Learning method.
Then, in~\cite{ArgyriouMPY07} this idea is extended to any spectral funcion $F: \mathbb{S}^d_{++} \to \mathbb{S}^d_{++}$ where $\mathbb{S}^d_{++}$ is the set of matrices $A \in \mathbb{R}^{d \times d}$ symmetric and positive definite. The definition for the spectral function $F(\mymat{d})$, where we can write $D = \mymat{u}^\intercal \Diag (\lambda_1, \ldots, \lambda_d)  \mymat{u}$ is:
$ F(\mymat{d}) = \mymat{u}^\intercal \Diag (f(\lambda_1), \ldots, f(\lambda_d)) \mymat{u} \; .$
Then, a generalized regularizer for problem~\eqref{eq:convmtl_feat_learning} can be expressed as
$$ \sum_t \langle w_t, F(\mymat{d}) w_t\rangle = \Tr (\mymat{w}^\intercal F(\mymat{d}) \mymat{w}) \; .$$
It is easy to see that problem~\eqref{eq:convmtl_feat_learning} is a particular case where $f(\lambda) = \lambda^{-1}$.
%  Learning multiple tasks using manifold regularization. Agarwal
Another relevant extension is shown in~\cite{AgarwalDG10}, where instead of assuming that the task parameters $w_r$ lie in a linear subspace, the authors generalize this idea by assuming that $w_r$ lies in a manifold $\mathcal{M} \in \rkhs$.
\begin{equation}
    \label{eq:mtl_feat_learning_manifold}   
    \begin{aligned}
        &\argmin_{\mymat{w}, \mathcal{M}, \myvec{b}}  \sum_{r=1}^\ntasks \sum_{i=1}^{\npertask_r} \lossf(y_i^r, \dotp{w_r}{x_i^r} + b_r) + \lambda \sum_{r=1}^\ntasks \mathcal{P}_\mathcal{M}(w_r) ,
    \end{aligned}
\end{equation}
where $\mathcal{P}_\mathcal{M}(w_t)$ represents the distance between $w_t$ and its projection on the manifold $\mathcal{M}$. Again, an approximation of~\eqref{eq:mtl_feat_learning_manifold} is used to obtain a convex problem and it is solved using a two-step optimization algorithm.

% Sparse coding and MTL. 
% K-Dimensional Coding Schemes in Hilbert Spaces
% Sparse coding for multitask and transfer learning Maurer 2013
% Learning task grouping and overlap in multi-task learning.
Other distinct, relevant approach for Feature Learning is the one described in~\cite{MaurerPR13}, where a sparse-coding~\cite{MaurerP10} method is used for MTL. Maurer et al. present the problem
\begin{equation}
    \label{eq:mtl_sparse_coding}
    \argmin_{\mymat{d} \in \mathcal{D}_k, \mymat{a} \in \reals^{k \times \ntasks}} \sum_{r=1}^\ntasks \sum_{i=1}^{\npertask_r} \lossf(y_i^r, \dotp{D a_r}{x_i^r}) + \lambda \norm{\mymat{d}}_{2, \infty} +\mu \norm{\mymat{a}}_{1, \infty} .
\end{equation}
Here, $\mathcal{D}_k$ is the set of $k$-dimensional dictionary and every $\mymat{d} \in \mathcal{D}_k$ is a linear map $\mymat{D}: \reals^k \to \rkhs$; in the linear case, where $\rkhs = \reals^d$, the set $\mathcal{D}_k$ is the set of matrices $\reals^{d \times k}$, such that 
$$\underset{d \times T}{\mymat{w}} = \underset{d \times k}{\mymat{D}} \underset{k \times T}{\mymat{a}}$$.
Although~\eqref{eq:mtl_feat_learning} and~\eqref{eq:mtl_sparse_coding} share a similar form, there are crucial differences. The matrix $\mymat{u}$ in~\eqref{eq:mtl_feat_learning} is an orthogonal square matrix, while the matrix $\mymat{d}$ of~\eqref{eq:mtl_sparse_coding} is overcomplete with $k > d$ columns of bounded norm.
A problem very similar to~\eqref{eq:mtl_sparse_coding} is presented in~\cite{KumarD12} where the idea is the same but the regularizers are the $L_{2, 2}$ (Frobenius) norm for $\mymat{d}$ and the $L_{1, 1}$ norm for $\mymat{a}$:
\begin{equation}
    \label{eq:mtl_go}
    \argmin_{\mymat{d} \in \mathcal{D}_k, \mymat{a} \in \reals^{k \times \ntasks}} \sum_{r=1}^\ntasks \sum_{i=1}^{\npertask_r} \lossf(y_i^r, \dotp{D a_r}{x_i^r}) + \lambda \norm{\mymat{d}}_{2, 2} +\mu \norm{\mymat{a}}_{1, 1} .
\end{equation}


\subsubsection*{Feature Selection approach}
%   Feature selection or block sparse regularization
The feature selection is also driven by learning a good set of features for all tasks, however it focuses on subsets of the original features. This is a more rigid approach than that of Feature Learning but is also more interpretable.

% G. Obozinski, B. Taskar, and M. Jordan, “Multi-task feature selection,” (2006)
% H. Liu, M. Palatucci, and J. Zhang, “Blockwise coordinate descent  procedures for the multi-task lasso, with applications to neural semantic basis discovery,” in ICML, 2009.
Most works on Multi-Task Feature Selection uses an $L_{p, q}$ regularization of the weights matrix $\mymat{w}$. The first work~\cite{obozinski2006multi} solves the problem
\begin{equation}
    \label{eq:mtl_feat_selection}   
    \begin{aligned}
        &\argmin_{\mymat{w}}  \sum_{r=1}^\ntasks \sum_{i=1}^{\npertask_r} \lossf(y_i^r, \dotp{w_r}{x_i^r}) + \lambda \norm{\mymat{w}^\intercal}_{2, 1}^2 ,
    \end{aligned}
\end{equation}
where the $L_{2, 1}$ regularization enforces row sparsity and forces different tasks to share some features $\mymat{w}^i$, which are the rows of $\mymat{w}$. In~\cite{LiuPZ09} the $L_{\infty, 1}$ regularization is used for the same goal. 
Then, in~\cite{GongYZ12} this idea is generalized with a capped-$L_{p, 1}$ penalty of $\mymat{W}^\intercal$, which is defined as
$ \sum_{i=1}^d min(\theta, \norm{W^i}_p).$
That is, the parameter $\theta$ enables a more flexible regularization, with small values of $\theta$ many rows should be zero, since only the small rows will be fully penalized and, as $\theta$ grows this penalty will degenerate to the standard $L_{p, 1}$ norm.

To allow more flexibility, a multi-level selection is presented in~\cite{LozanoS12}. The main idea is to decompose each $w_r^i$, that is, the $i$-th feature of the $r$-th task as 
$w_r^i = \theta^i \gamma_r^i$ and using an $L_1$ regularizer on the vector $\myvec{\theta}$ and the vectors $\myvec{\gamma}^i$, which are the concatenations of $\theta^i$ and $\gamma_r^i$ for all tasks respectively. By doing this, the features $i$ such that $\theta^i = 0$, are discarded for all tasks, but the rest may be shared among tasks or not depending on $\gamma_r^i$.

% Bayesian approaches
% 15. Zhang Y, Yeung DY and Xu Q. Probabilistic multi-task feature selection. In:
% Advances in Neural Information Processing Systems 23. 2010, 2559–67.
% 16. Hern ́andez-Lobato D and Hern ́andez-Lobato JM. Learning feature selection
% dependencies in multi-task learning. In: Advances in Neural Information Pro-
% cessing Systems 26. 2013, 746–54.
% 17. Hern ́andez-Lobato D, Hern ́andez-Lobato JM and Ghahramani Z. A probabilis-
% tic model for dirty multi-task feature selection. In: Proceedings of the 32nd
% International Conference on Machine Learning. 2015, 1073–82.
The feature selection methods based on $L_{p, 1}$ regularization are shown to be equivalent to a Bayesian approximation with a generalized Gaussian prior in~\cite{ZhangYX10}. Moreover, this approach also allows to find the relationship among tasks and to identify outliers. In~\cite{Hernandez-LobatoH13} a horseshoe prior is used instead to learn feature covariance; and~\cite{Hernandez-Lobato15} this prior is also used to identify outlier tasks.


\subsubsection*{Deep Learning approach}
% Hard parameter sharing
% Soft parameter sharing

% Deep multi-task representation learning: A tensor factorisation approach

% Trace norm regularised deep multi-task learning

%  Linear algorithms for online multitask classification. 

% cross-stitch networkf for multi-task learning


\subsection{Parameter-Based MTL}
The parameter-based MTL does not focus on shared sets of features across tasks, instead other types of dependencies among the task (the task-parameters $w_r$ specifically) are taken into account. Some approaches rely on the assumption that the Multi-Task weight matrix $\mymat{w}$ has a low rank, others try to learn the pairwise task relations or to cluster the tasks. A different approach is the decomposition one, where the assumption is that the matrix $\mymat{w}$ can be expressed as the summation of multiple matrices.

\subsubsection*{Low-Rank approach}
% Low-rank

% .  K.  Ando  and  T.  Zhang,  “A  framework  for  learning  predictive structures  from  multiple  tasks  and  unlabeled  data,” 2005

% A convex formulation for learning shared structures from multiple tasks 2009

% Trace Norm Regularization: Reformulations, Algorithms, and Multi-Task Learning

% Multi-Stage Multi-Task Learning with Reduced Rank
In the low-rank approach, the assumption is that the task parameters $w_r$ share a low-dimensional space, or are close to this subspace. This is similar to the Feature Learning approach, but it is not that rigid, since it allows for some flexibility.
The idea in~\cite{AndoZ05} is that the task parameters can be decomposed as
$$ w_r = u_r + \mymat{\Theta}^\intercal v_r,$$
where $\mymat{\Theta} \in \reals^{\dimx \times k}$ spans a shared low dimensional space, that is $\mymat{\Theta}^\intercal \mymat{\Theta} = \mymat{i}_k$ with $k < \dimx$, and $d$ is the dimension of the data. Under this consideration, the proposed model is
\begin{equation}
    \label{eq:mtl_struct_learn}
    \argmin_{\mymat{\Theta} \in \reals^{\dimx \times k}, \myvec{u}, \myvec{v}} \sum_{r=1}^\ntasks \sum_{i=1}^{\npertask_r} \lossf(y_i^r, \dotp{u_r + \mymat{\Theta}^\intercal v_r}{x_i^r}) + \lambda \sum_{r=1}^\ntasks \norm{u_r}^2 \text{ s.t. } \mymat{\Theta}^\intercal \mymat{\Theta} = \mymat{i}_k .
\end{equation}
Observe that this problem shares some similarities with~\eqref{eq:mtl_feat_learning}. However, this is a more flexible approach, since the vectors $u_r$ allow for deviations of the task parameters from the shared subspace.
Problem~\eqref{eq:mtl_struct_learn} is solved using a two-step optimization fixing $\{\mymat{\Theta}, \myvec{v}\}$ and minimizing in $\myvec{u}$ and iterating until convergence. However problem~\eqref{eq:mtl_struct_learn} is not convex. 
Note that~\eqref{eq:mtl_struct_learn} can be reformulated as
\begin{equation}
    \label{eq:mtl_struct_learn_ref}
    \argmin_{\mymat{\Theta} \in \reals^{\dimx \times k}, \myvec{w}, \myvec{v}} \sum_{r=1}^\ntasks \sum_{i=1}^{\npertask_r} \lossf(y_i^r, \dotp{w_r}{x_i^r}) + \lambda \sum_{r=1}^\ntasks \norm{w_r - \mymat{\Theta}^\intercal v_r}^2 \text{ s.t. } \mymat{\Theta}^\intercal \mymat{\Theta} = \mymat{i}_k ,
\end{equation}
where the terms $\norm{w_r - \mymat{\Theta}^\intercal v_r}^2$ enforces the similarity across tasks by binging them closer to the shared subspace.
In~\cite{ChenTLY09} the following extension is proposed:
\begin{equation}
    \label{eq:mtl_struct_learn_frob}
    \argmin_{\mymat{\Theta} \in \reals^{\dimx \times k}, \myvec{w}, \myvec{v}} \sum_{r=1}^\ntasks \sum_{i=1}^{\npertask_r} \lossf(y_i^r, \dotp{w_r}{x_i^r}) + \lambda \sum_{r=1}^\ntasks \norm{w_r - \mymat{\Theta}^\intercal v_r}^2 + \mu \sum_{r=1}^\ntasks \norm{w_r}^2 \text{ s.t. } \mymat{\Theta}^\intercal \mymat{\Theta}  = \mymat{i}_k .
\end{equation}
And it is shown that~\eqref{eq:mtl_struct_learn_frob}, when relaxing the orthogonality constraint, can be expressed as a convex minimization problem.
Note that the difference between~\eqref{eq:mtl_struct_learn_ref} and~\eqref{eq:mtl_struct_learn_frob} is the addition of the Frobenius norm of matrix $\mymat{w}$, also known as trace norm. This norm penalizes $\sum_{i=1}^d {\lambda_i(\mymat{w})}^2$, where $\lambda_i(\mymat{w})$ is an eigenvector of $\mymat{w}$, and thus, forcing $\mymat{w}$ to be low-rank.
In the work of~\cite{PongTJY10} new formulations for problems with this trace norm penalty and a primal-dual method for solving the problem is developed.
A modification of the trace norm can be found in~\cite{HanZ16}, where a capped-Frobenius norm is defined as $\sum_{i=1}^d \min(\theta, {\lambda_i(\mymat{w})}^2)$. This capped norm, as in the capped-$L_{p, q}$ norm, can enforce a lower rank matrix for small $\theta$ and also degenerates to the trace norm for large enough $\theta$. 


\subsubsection*{Task-Relation Learning approach}
% Task relation learning
In other approaches, like the Feature Learning approach or the Low-Rank approach, the assumption is that all task parameters share the same subspace, which may be detrimental when there exists a negative or neutral transfer. The Task-Relation Learning approach aims to find the pairwise dependencies among tasks and to possibly model positive, neutral and negative transfers between tasks.

% GP approach
% Multi-task Gaussian Process Prediction. Bonilla. 2007
% A Convex Formulation for Learning Task Relationships in Multi-Task Learning 2010
% A regularization approach to learning task relationships in multitask learning 2013
One of the first works with the goal of explicitly modelling the pairwise task-relations is~\cite{BonillaCW07}, where a Multi-Task Gaussian Process (GP) formulation is presented. Assuming a Gaussian noise model 
$y_i^r \sim \normal{f_r(x_i^r), \sigma_r^2}$, 
Bonilla et al. place a GP prior over the latent functions $\myvec{f}_r$ to induce correlation among tasks:
$$ \myvec{f} \sim \normal{\myvec{0}_{d \ntasks}, \mymat{K}^f \otimes K_\theta^\Xspace}$$ 
where $\myvec{f} = (\myvec{f}_1, \ldots, \myvec{f}_\ntasks)$ and $\mymat{a} \otimes \mymat{b}$ is the Kronecker product of two matrices, so $\Cov(f_r(x), f_s(x')) = \mymat{K}^f_{rs} k^\Xspace_\theta(x, x')$. That is, instead of assuming a block-diagonal covariance matrix for $\myvec{f}$ as in previous works of Joint Learning~\cite{LawrenceP04}, the authors in~\cite{BonillaCW07} model the covariance as a product of inter-task covariance and inter-feature covariance. The inference of this model can be done using the standard GP inference for the mean and the variance of the prediction distribution. The mean prediction for a new data point $x_*$ of task $s$ is:
\begin{equation}
    \nonumber
    f_s(x_*) = (\mymat{K}^f_{:,l} \otimes \mymat{K}^\Xspace(:, x_*))^\intercal \Sigma^{-1} \myvec{y}, \;
     \Sigma = \mymat{K}^f \otimes \mymat{K}^\Xspace_\theta + \Diag(\sigma_1, \ldots, \sigma_\ntasks) \otimes \mymat{i}_{\nsamples}.
\end{equation}
However the interest resides in learning the task-covariance matrix $\mymat{K}^f$, but this leads to a non-convex problem. The authors propose a low-rank approximation of $\mymat{K}^f$, which weaken its expressive power.
To overcome this disadvantage, in~\cite{ZhangY10,ZhangY13a}, using the idea of the Multi-Task GP they consider linear models $f(x_i^r) = \dotp{w_r}{x_i^r} + b_r$ and the prior on matrix $\mymat{w} = (w_1 \ldots, w_\ntasks)$ is defined as
\begin{equation}
    \nonumber
    \mymat{W} \vert \sigma_r \sim \left(\prod_{r=1}^\ntasks \normal{\myvec{0}_d, \theta_i^2 I_d}  \right) \multinormal{\mymat{0}_{d \times m}, \mymat{i}_d \otimes \mymat{\Omega}}
\end{equation}
where $\multinormal{0, \mymat{M} \otimes \mymat{B}}$ denotes the matrix-variate normal distribution with mean $\mymat{M}$, row covariance matrix $\mymat{a}$ and column covariance matrix $\mymat{b}$. It is shown that the problem of selecting the maximum a posteriori estimation of $\mymat{w}$ and the maximum Likelihood estimations of $\mymat{\Omega}$ and $\myvec{b}$ has a regularized minimization problem that, when relaxing the restrictions on $\Omega$, can be expressed as
% \begin{equation}
%     \label{eq:mtl_relation_learn_trace}
%     \argmin_{\mymat{\Omega} \in \reals^{d \times \ntasks}, \mymat{w}, \myvec{b}} \sum_{r=1}^\ntasks \sum_{i=1}^{\npertask_r} \lossf(y_i^r, \dotp{w_r}{x_i^r}) + \lambda \Tr\left({\mymat{w} \mymat{w}^\intercal}\right) + \mu \Tr \left(\mymat{w} \mymat{\Omega}^{-1} \mymat{w}^\intercal \right)\text{ s.t. } \mymat{\mymat{\Omega}} \succeq 0, \Tr\mymat{\Omega} = 1 .
% \end{equation}
\begin{equation}
    \label{eq:mtl_relation_learn}
    \begin{aligned}
        &\argmin_{\mymat{\Omega} \in \reals^{d \times \ntasks}, \mymat{w}, \myvec{b}} \sum_{r=1}^\ntasks \sum_{i=1}^{\npertask_r} \lossf(y_i^r, \dotp{w_r}{x_i^r} + b_r) + \lambda \sum_{r=1}^\ntasks \norm{w_r}^2 + \mu \sum_{r,s=1}^\ntasks \left(\Omega^{-1}\right)_{rs} \norm{w_r - w_s}^2 \\
        &\text{ s.t. } \mymat{\mymat{\Omega}} \succeq 0, \Tr\mymat{\Omega} = 1 .
    \end{aligned}    
\end{equation}
This is a convex formulation and a two-step procedure is developed to find the solution. 

% Laplacian approach
% Learning the Graph of Relations Among Multiple Tasks. 2013
% Learning Output Kernels for Multi-TaskProblems. 2013
% Convex learning of multiple tasks and their structure 2015
Other approaches like~\cite{argyriou2013learning} reach a similar problem from other perspective. Argyriou et al. assume a representation of the structure of the tasks as a graph, then the graph Laplacian in the optimization problem can incorporate the knowledge about the task structure as shown in~\cite{EvgeniouMP05}:
\begin{equation}
    \label{eq:mtl_laplacian}
    \begin{aligned}
        &\argmin_{\mymat{w}, \myvec{b}} \sum_{r=1}^\ntasks \sum_{i=1}^{\npertask_r} \lossf(y_i^r, \dotp{w_r}{x_i^r} + b_r) + \mu \sum_{r,s=1}^\ntasks \left(L^{-1}\right)_{rs} \norm{w_r - w_s}^2 \\
        &\text{ s.t. } \mymat{\mymat{\Omega}} \succeq 0, \Tr\mymat{\Omega} = 1 .
    \end{aligned}    
\end{equation}
That is, we restrict the task covariance matrix $\Omega$ to be a valid graph Laplacian matrix.
The goal of~\cite{argyriou2013learning} is to jointly learn the task parameters and the graph Laplacian $\mymat{L}$. For that, first they add the penalty $ \lambda \sum_{r=1}^\ntasks \norm{w_r}^2$ to the objective function of~\eqref{eq:mtl_laplacian}. Then, according to~\cite{EvgeniouMP05} the problem~\eqref{eq:mtl_laplacian} can be solved in the dual space using a positive definite kernel
$$ k_\mymat{L}(x_i^r, x_j^s) = \left(\mymat{L}+ \lambda \mymat{I} \right)^{-1} _{rs} k(x_i^r, x_j^s). $$
Argyriou et al. propose the joint learning in this dual space by solving
\begin{equation}
    \label{eq:mtl_laplacian_dual}
    \begin{aligned}
        &\argmin_{\myvec{\alpha}, \mymat{L}} \myvec{\alpha}^\intercal \mymat{K}_L \myvec{\alpha} + \nu \myvec{\alpha}^\intercal \bm{y}  + \Tr \left(\mymat{L}+ \lambda \mymat{I} \right)^{-1} \\
        &\text{ s.t. } \mymat{0} \preceq \left(\mymat{L}+ \lambda \mymat{I} \right)^{-1} \preceq \frac{1}{\lambda} I,\; \left(\mymat{L}+ \lambda \mymat{I} \right)^{-1}_\text{off} \leq 0, \; \left(\mymat{L}+ \lambda \mymat{I} \right)^{-1} \myvec{1}_\nsamples = \frac{1}{\lambda} \myvec{1}_\nsamples ,
    \end{aligned}
\end{equation}
where $\mymat{A}_\text{off}$ denotes the off diagonal entries of $\mymat{A}$. The restrictions on of problem~\eqref{eq:mtl_laplacian_dual} are to ensure that $\mymat{L}$ is a valid graph Laplacian, and the trace term in the objective function is to force $\mymat{L}$ to be low-rank so it is easier to find clusters of tasks. Although the objective function of~\eqref{eq:mtl_laplacian_dual} is convex, the feasible space is not convex.
Another work focused on learning the task-relations is~\cite{Dinuzzo13}, where instead of using a graph Laplacian matrix, a decomposition of $\Theta$ in the product of two low-rank matrices.
In~\cite{CilibertoMPR15} some results are developed for the convergence of alternating minimization algorithms as that used in~\cite{ZhangY13a}. However, since the problems presented in~\cite{argyriou2013learning,Dinuzzo13} are not convex, they do not hold this result.

% learning to multitask ?

\subsubsection*{Task Clustering approach}
The Task Clustering approach tries to find $K$ clusters or groups among the original set of $T$ tasks. Usually, the goal is to learn jointly only the tasks in the same cluster, so no negative transfer takes place.
% Discovering structure in multiple learning tasks: The TC algorithm. 1996
% Task clustering and gating for Bayesian multitask learning. 1999
% Clustered multi-task learning: A convex formulation. 2008
% Learning  with  whom  to  share  in multi-task feature learning. 2011
% Learning task grouping and overlap in multi-task learning. 2012
% Learning Multiple Tasks using Shared Hypotheses 2012
% Convex Multi-Task Learning by Clustering. 2015
The first clustering approach~\cite{ThrunO96} divides the optimization process in two separate steps: independently learning the task-parameters and jointly learning the clusters of tasks.
It assumes that the models involves $f(x)$ needs a definition of distance, e.g. kernel methods, such that
$$ \text{dist}_\phi(x, x') = \sqrt{\sum_{i=1}^\dimx \phi^i \left( x^i - x'^i \right)^2}. $$
That is, $\phi$ parametrizes the distance with a different weight for each feature.
Then, for each task $r=1, \ldots, \ntasks$ an optimal $\phi_r^*$ is computed minimizing the distance between examples of the same class and maximizing the distance among different classes:
\begin{equation}
    \nonumber
    A_r(\phi) = \sum_{i, j}^{\npertask_r} (y_i^r  y_j^r) \text{dist}_\phi(x_i^r, x_j^r).
\end{equation}
Then, the procedure is the following. First the empirical loss on task $r$ of a model fitted (on task $r$) using a distance parametrized by $\phi_s^*$ is defined as $e_{rs}$. Then, for $K = 1, 2,  \ldots, \ntasks$, the following functional is minimized
\begin{equation}
    \nonumber
    J(K) = \sum_{\kappa = 1}^K \sum_{r \in B_\kappa} \frac{1}{\abs{B_r}} \sum_{s \in B_\kappa} e_{r, s} ,
\end{equation}
and the number of clusters $K$ with minimum $J(K)$ is selected. That is, the clusters are selected using the results of independently trained tasks and the transfer is done through the parameters $\phi$.
%

A different approach is the Bayesian proposal of~\cite{BakkerH03} where a Multi-Task Neural Network~\cite{Caruana97} with one shared hidden layer is used. The task-dependent parameters can be modeled together using a prior that is a mixture of $K$ Gaussians $\myvec{a}_r \sim \sum_{\kappa = 1}^K \alpha_\kappa \normal{\myvec{\mu}_\kappa, \mymat{\Sigma}_\kappa}$; and by learning the variables $\alpha_i$ using Bayesian inference, the tasks can be clustered. In this model, unlike in~\cite{ThrunO96} the clusters and task parameters are jointly learned.
%

From the regularization approach some proposals have also been made. In~\cite{JacobBV08} a problem based on~\cite{EvgeniouP04} is proposed. Considering $\mymat{U} = \frac{1}{\ntasks} \myvec{1} \myvec{1}^\intercal$, $\mymat{E}$ the $\ntasks \times K$ cluster assignment binary matrix, and defining the adjacency matrix $M = E (E^\intercal E)^{-1} E^\intercal$ the problem is
\begin{equation}
    \label{eq:mtl_clustered}
    \begin{aligned}
        &\argmin_{\mymat{w}} \sum_{r=1}^\ntasks \sum_{i=1}^{\npertask_r} \lossf(y_i^r, \dotp{w_r}{x_i^r}) + \lambda (\mu_\text{m} \Omega_\text{m}(\mymat{w}) + \mu_\text{b} \Omega_\text{b}(\mymat{w}) + \mu_\text{w} \Omega_\text{w}(\mymat{w})) ,
    \end{aligned}    
\end{equation}
where $\Omega_\text{m} = \Tr\left( \mymat{W} \mymat{U} \mymat{W}^\intercal \right)$ is the mean regularization, $\Omega_\text{b} = \Tr\left( \mymat{W} (\mymat{M} - \mymat{U}) \mymat{W}^\intercal \right)$ is the inter-cluster variance regularization and $\Omega_\text{w} = \Tr\left( \mymat{W} (\mymat{I} - \mymat{M}) \mymat{W}^\intercal \right)$ is the intra-cluster variance regularization.
This problem cannot be solved using the results of~\cite{EvgeniouMP05} because the regularization used is not convex, so a convex relaxation is needed.
%

A similar approach is presented in~\cite{KangGS11} where, using the results from~\cite{ArgyriouEP08}, they propose a trace norm regularizer of the matrices $\mymat{W}_\kappa = \mymat{W} Q_\kappa$, where $Q_\kappa$ are diagonal matrices and the $r$-th element of the diagonal indicates if task $r$ corresponds to cluster $\kappa$. They consider the problem
\begin{equation}
    \label{eq:mtl_clustered_featlearn}   
    \begin{aligned}
        &\argmin_{\mymat{w}, \mymat{Q_g}, \myvec{b}}  \sum_{r=1}^\ntasks \sum_{i=1}^{\npertask_r} \lossf(y_i^r, \dotp{w_r}{x_i^r} + b_r) + \lambda \sum_{\kappa=1}^K \Tr\left(\mymat{W}_\kappa \mymat{W}_\kappa^\intercal \right) .
    \end{aligned}
\end{equation}
This can be seen as a clusterized version of Multi-Task Feature Learning~\cite{ArgyriouEP06}, that is, instead of assuming that all tasks share the same subspace, only the tasks in the same cluster do, however the explicit learned features cannot be recovered.
%

Another approximation to clusterized MTL is provided in~\cite{CrammerM12}, where a two-step procedure if described as follows. Considering that $K$ initial clusters are fixed containing the $T$ tasks, then two steps are repeated:
First $K$ single task models $f_\kappa$ are fitted using the pooled data from tasks in cluster $\kappa$.
Secondly each task $r$ is assigned to the cluster $\kappa$ whose function $f_\kappa$ obtains the lowest error in task $r$.
%
The proposal of~\cite{BarzilaiC15} takes the idea of the cluster assignation step from~\cite{CrammerM12} and is also inspired by the work of~\cite{KumarD12}. In this work a model is presented where $\mymat{w} = \mymat{D} \mymat{A}$, where $\mymat{D} \in \reals^{\dimx \times K}$ contains as columns the hypothesis for each cluster and $\mymat{G} \in \reals^{K \times \ntasks}$ is the task assignment matrix, that is ${g}_r = G_r \in \set{0, 1}$ and $\norm{{g}_r}^2 = 1$. The corresponding optimization problem is
\begin{equation}
    \label{eq:mtl_clustering_convex}
    \begin{aligned}
        &\argmin_{\mymat{d} \in \reals^{\dimx \times K}, \mymat{G} \in \reals^{K \times \ntasks}} \sum_{r=1}^\ntasks \sum_{i=1}^{\npertask_r} \lossf(y_i^r, \dotp{D g_r}{x_i^r}) + \lambda \norm{\mymat{d}}_{2, 1} \\
        & \text{s.t. } {g}_r \in [0, 1], \; \norm{{g}_r}^2 = 1 ,
    \end{aligned}
\end{equation}
where the constraints on $g_r$ have been relaxed to be in the $[0,1]$ interval in order to make problem~\eqref{eq:mtl_clustering_convex} convex. Observe that~\eqref{eq:mtl_clustering_convex} is similar to~\eqref{eq:mtl_go} but different restrictions are used to ensure that $\mymat{G}$ can be seen as a clustering assignment matrix.

\subsubsection*{Decomposition approach}
The Decomposition approach considers that the assumption that the task parameters resides in the same subspace or that the parameter matrix $\mymat{w}$ is too restrictive for real world scenarios. The proposition is then to decompose the parameter matrix in the sum of two matrices, i.e. $\mymat{W} = \mymat{U} + \mymat{V}$ where usually $\mymat{U}$  captures the shared properties of the tasks and $V$ accounts for the information that cannot be shared among tasks.
This models also receive the name of \emph{dirty models} because they assume that the data is \emph{dirty} and does not place perfectly in constrained subspaces.
% A Dirty Model for Multi-task Learning. 2010
% Learning incoherent sparse and low-rank patterns from multiple tasks. 2010
% Robust multi-task feature learning 2012
% Integrating low-rank and group-sparse structures for robust multi-task learning. 2012
% A convex feature learning formulation for latent task structure discovery. 2012
% Hierarchical regularization cascade for joint learning 2013
% Learning tree structure in multi-task learning. 2015
The optimization problem that is solved can be expressed as
\begin{equation}
    \label{eq:mtl_dirty}
    \argmin_{\mymat{u}, \mymat{v} \in \reals^{\dimx \times \ntasks}} \sum_{r=1}^\ntasks \sum_{i=1}^{\npertask_r} \lossf(y_i^r, \dotp{u_r + v_r}{x_i^r}) + \lambda g(\mymat{U}) +\mu h(\mymat{V}),
\end{equation}
where $g(\mymat{U})$ and $h(\mymat{V})$ are different regularizers for $\mymat{U}$ and $\mymat{V}$, respectively.
In~\cite{JalaliRSR10} $g(\mymat{U}^\intercal) = \norm{\mymat{U}}_{1, \infty}$ to enforce block-sparsity and $h(\mymat{V}) = \norm{\mymat{V}}_{1, 1}$ to enforce element-sparsity. 
In~\cite{ChenLY10} $g(\mymat{U}) = \norm{\mymat{U}}_{2, 2}$ to enforce low-rank while maintaining $h(\mymat{V}) = \norm{\mymat{V}}_{1, 1}$. 
In~\cite{ChenZY11} both regularizers seek properties shared among all tasks, $g(\mymat{U}) = \norm{\mymat{U}}_{2, 2}$ to enforce a low-rank and $h(\mymat{V}) = \norm{\mymat{V}^\intercal}_{1, 2}$ for row-sparsity.
In~\cite{GongYZ12rmfl} they propose $g(\mymat{U}) = \norm{\mymat{U}^\intercal}_{1, 2}$ to enforce row-sparsity, i.e. the tasks share a common subspace; and $h(\mymat{V}) = \norm{\mymat{V}}_{1, 2}$ which penalizes the orthogonal parts to the common subspace of task-parameter, the authors state that it penalizes outlier tasks.

Other approaches generalize the decomposition method by assuming that the parameter matrix can be expressed as $\mymat{W} = \sum_{l=1}^L \mymat{W}_l$, then the problem to solve has the form
\begin{equation}
    \label{eq:mtl_decomposition_gen}
    \argmin_{\mymat{W}_1, \ldots, \mymat{W}_L \in \reals^{\dimx \times \ntasks}} 
    \sum_{r=1}^\ntasks \sum_{i=1}^{\npertask_r} 
    \lossf(y_i^r, \dotp{\sum_{l=1}^L {(W_l)}_r}{x_i^r}) 
    + \sum_{l=1}^L \lambda_l r(\mymat{W}_l) ,
\end{equation}
In~\cite{ZweigW13} the regularizer used is $r(\mymat{W_l}) = \norm{\mymat{W_l}^\intercal}_{2, 1} + \norm{\mymat{W_l}}_{1, 1}$ to enforce the row and element-sparsity. 
In~\cite{HanZ15} the regularizer is $r(\mymat{W_l}) = \sum_{r,s=1}^T \norm{(W_l)_r - (W_l)_s}^2$, which alongside some constraints it allows to build a tree of task groups, where the root contains all the tasks and the leafs only correspond to one task.


\subsection{Joint MTL}
% Joint Learning
% Sometimes in task clustering approaches, sometimes in task relations learning
The Joint Learning methods for MTL can be divided into the frequentist and the Bayesian approaches.
The frequentist approaches uses a combination of task-specific models and models that are common to all tasks. These two models are learned simultaneously with the goal of leveraging the common and specific information.
In the Bayesian approaches common prior is shared for all the tasks models, and the diverse sources of data definetele different posterior distributions for each task.

\subsubsection*{Frequentist Approach}
% Frequentist approach
% Evgeniou, T. and Pontil, M. (2004). Regularized multi-task learning.
The first proposal of the frequentist approach, which uses the SVM as the foundation, is found in~\cite{EvgeniouP04} where the \emph{regularized MTL} SVM is presented. The goal is to find a decision function for each task, each being defined by a vector
$$w_r = w + v_r,$$
where $w$ is common to all tasks and $v_r$ is task-specific.
The primal problem of \emph{regularized MTL} SVM, using the unified formulation, is 
\begin{equation}
    \label{eq:regmtlsvm_primal}
    \begin{aligned}
        & \argmin_{w, v_r, \xi_i^r}
        & & C \sum_{r=1}^\ntasks \sum_{i=1}^{\npertask_r} \xi_i^r + \frac{1}{2} \dotp{w}{w} + \sum_{r=1}^\ntasks \frac{\mu}{2} \dotp{v_r}{v_r} \\
        & \text{s.t.}
        & & y_{i}^r ( \dotp{w}{x_{i}^r} + \dotp{v_r}{x_{i}^r}) \geq p_i^r - \xi_i^r ,\\
        & & &\xi_i^r \geq 0, \\
        & \text{for } & & r=1, \ldots, \ntasks; \; i=1, \ldots, \npertask_r.
    \end{aligned}
\end{equation}
% Leveraging common and specific information
Note that $\mu$ is a parameter that controls the tradeoff between the relevance of common and specific models. That is, when $\mu$ tends to infinite, the resulting model approaches a common-task standard SVM; when $\mu$ tends to zero, a independent task approach is taken, with one standard SVM problem for each task.
This is also reflected in the corresponding dual problem
\begin{equation}\label{eq:regmtlsvm_dual}
    \begin{aligned}
        & \argmin_{\alpha_i} 
        & & \frac{1}{2} \sum_{r, s=1}^\ntasks \sum_{i, j=1}^{\npertask_r} y_i^r y_j^s \alpha_i^r \alpha_j^s \dotp{x_i^r}{x_j^s} + \frac{1}{2 \mu} \sum_{r, s=1}^\ntasks  \sum_{i, j=1}^{\npertask_r} y_i^r y_j^s \alpha_i^r \alpha_j^s \delta_{rs} \dotp{x_i^r}{x_j^s} \\
        & & & \qquad - \sum_{r=1}^\ntasks \sum_{i=1}^{\npertask_r} p_i^r \alpha_i^r \\
        & \text{s.t.}
        & & 0 \leq \alpha_i^r \leq C \\
        & \text{for } & & r=1, \ldots, \ntasks; \; i=1, \ldots, \npertask_r.
        \end{aligned}
\end{equation}
In this dual form, as $\mu$ grows, the task-specific part goes to zero, and the most important term is the first one, corresponding to the common part. The opposite effect is obtained when $\mu$ shrinks.
% Common + specific model which is equivalent to penalizing individual norm and variance
Moreover, in~\cite{EvgeniouP04} it is shown that solving~\eqref{eq:regmtlsvm_primal} is equivalent to solving the problem
\begin{equation}
    \nonumber
    \begin{aligned}
        & \argmin_{ww_r, \xi_i^r}
        & & C \sum_{r=1}^\ntasks \sum_{i=1}^{\npertask_r} \xi_i^r +  \frac{1}{2} \sum_{r=1}^\ntasks \norm{w_r}^2 + \frac{\mu}{2} \sum_{r=1}^\ntasks  \norm{w_r - \sum_{s=1}^\ntasks w_s}^2 \\
        & \text{s.t.}
        & & y_{i}^r ( \dotp{w_r}{x_{i}^r}) \geq p_i^r - \xi_i^r ,\\
        & & &\xi_i^r \geq 0, \\
        & \text{for } & & r=1, \ldots, \ntasks; \; i=1, \ldots, \npertask_r.
    \end{aligned}
\end{equation}
Now, only the $w_r$ variables are included, and it is clearer that $\mu$ penalizes the variance of the $w_r$ vectors, so all models $w_r$ will tend to a common model as $\mu$ grows.
%

Multiple extensions of the work of~\cite{EvgeniouP04} have been presented: in~\cite{XuAQZ14, LiTST15} the method is extended to the Proximal SVM~\cite{FungM01} and Least Squares SVM~\cite{SuykensV99}, respectively. Also, in~\cite{ParameswaranW10} the idea is adapted for the Large Margin Nearest Neighbor model~\cite{WeinbergerS09}.
%
% Evgeniou, T., Micchelli, C. A., and Pontil, M. (2005).  Learning multiple tasks with kernel methods.
% MTLSVM and Generalized SMO
However, in this work we are interested mainly in two extensions: one is the work of~\cite{EvgeniouMP05}, pivotal for this thesis, which will be described in Section\comm{TODO}; the other relevant extension is developed in~\cite{LiangC08}, which has already been discussed in the previous section.
The multi-task problem described in~\cite{LiangC08} for classification, is also adapted for regression problems in~\cite{CaiC09}. Using the unified notation we can express the primal problem as
\begin{equation}
    \label{eq:mtlsvm_primal_unif}
    \begin{aligned}
        & \argmin_{w, b, v_r, b_r, \xi_i^r}
        & & C \sum_{r=1}^\ntasks \sum_{i=1}^{\npertask_r} \xi_i^r + \frac{1}{2} \dotp{w}{w} + \sum_{r=1}^\ntasks \frac{\mu}{2} \dotp{v_r}{v_r} \\
        & \text{s.t.}
        & & y_{i} ( \dotp{w}{\phi(x_{i}^r)} + b + \dotp{v_r}{\phi_r(x_{i}^r)} + b_r) \geq p_i^r - \xi_i^r ,\\
        & & &\xi_i^r \geq 0, \\
        & \text{for } & & r=1, \ldots, \ntasks; \; i=1, \ldots, \npertask_r.
    \end{aligned}
\end{equation}
Comparing~\eqref{eq:regmtlsvm_primal} and~\eqref{eq:mtlsvm_primal_unif} we observe that the subyacent idea is the same, but there exists some differences. In first place, note that~\eqref{eq:regmtlsvm_primal} is described as a linear model, while in~\eqref{eq:mtlsvm_primal_unif} not only non-linear transformations of the data are used, but different transformations can be selected $\phi, \phi_r$ for the common part and for each task-specific term, respectively. Moreover, it is relevant to note the incorporation of the bias terms in~\eqref{eq:mtlsvm_primal_unif}.
The dual form of~\eqref{eq:mtlsvm_primal_unif} is
\begin{equation}\label{eq:mtlsvm_dual_unif}
    \begin{aligned}
        & \argmin_{\alpha_i} 
        & & \frac{1}{2} \sum_{r, s=1}^\ntasks \sum_{i, j=1}^{\npertask_r} y_i^r y_j^s \alpha_i^r \alpha_j^s \left[k(x_i^r, x_j^s) + \delta_{rs} k_r(x^r_i, x^s_j) \right] - \sum_{r=1}^\ntasks \sum_{i=1}^{\npertask_r} p_i^r \alpha_i^r \\
        & \text{s.t.}
        & & 0 \leq \alpha_i^r \leq C \\
        & & & \sum_{i=1}^{\npertask_r}{\alpha_i^r y_i^r} = 0, \\
        & \text{for } & & r=1, \ldots, \ntasks; \; i=1, \ldots, \npertask_r.
        \end{aligned}
\end{equation}
In~\eqref{eq:mtlsvm_dual_unif} $\ntasks$ equality constraints, not present in~\eqref{eq:regmtlsvm_dual} are added. This a direct consequence of the incorporation of bias terms in the primal form. Since the original SMO algorithm does not account for multiple equality constraints, in~\cite{CaiC12} a generalized SMO algorithm is developed.
Also, it is important to observe the use of different kernel spaces through the functions $k$ and $k_r$.
% Connection with LUPI!
This has connections with the LUPI paradigm~\cite{VapnikI15a}, as described in Section\comm{TODO}. The kernel space for the common part is named the decision space, and the spaces corresponding to the kernel functions $k_r$ are the correcting spaces. That is, each task can correct the similarity defined by the decision space independently.

% Eigenfunction-Based Multitask Learning in a Reproducing Kernel Hilbert Space ?

\subsubsection*{Bayesian Approach}
% Bayesian approach
% Learning to learn with the informative vector machine (2004)
In the Bayesian approaches, the first work of~\cite{LawrenceP04} presents a GP model where all the tasks share a common prior.
That is, given $T$ tasks, a noise model with latent variables $\myvec{f}_r$ is considered, i.e. $y_i^r = f_i^r + \epsilon$, and $\myvec{f}_r$ follows a GP prior
$$ p(\myvec{f}_r \vert \mymat{x}, \myvec{\theta}) = \normal{0, \mymat{k}_{\myvec{\theta}}} $$
where $\mymat{k}$ is a kernel matrix parametrized by $\myvec{\theta}$ and evaluated at the points $\mymat{x}$. Note that a single $\myvec{\theta}$ parameter is used to model a prior shared for all tasks.
The posterior probability can be expressed as
\begin{equation}
    \nonumber
    p (\myvec{y}^1, \ldots, \myvec{y}^\ntasks \vert \myvec{f}_1, \ldots, \myvec{f}_\ntasks, \myvec{\theta},  \mymat{x}^1, \ldots, \mymat{x}^\ntasks) \propto  p (\myvec{f}_1, \ldots, \myvec{f}_\ntasks, \myvec{\theta} \vert \mymat{x}^1, \ldots, \mymat{x}^\ntasks) \prod_{r=1}^\ntasks p(\myvec{y}^r \vert \myvec{f}_r, \myvec{\theta}) %p(\myvec{\theta} | \myvec{f}_r)
\end{equation}
where the distribution for the latent parameters is
\begin{equation}
    \nonumber
    p (\myvec{f}_1, \ldots, \myvec{f}_\ntasks, \myvec{\theta} \vert \mymat{x}^1, \ldots, \mymat{x}^\ntasks) \propto \prod_{r=1}^\ntasks p(\myvec{f}_r \vert \mymat{x}^r, \myvec{\theta}) , %p(\myvec{\theta} | \myvec{f}_r).
\end{equation}
where $p(\myvec{f}_r \vert \mymat{x}^r, \myvec{\theta}) = \normal{0, K_\theta}$, that is 
$\Cov(f^r(x), f^r(x')) = k_\theta(x, x') .$
Although this idea is interesting for MTL, it presents a rigid framework since we use a fixed model for the prior $ p(\myvec{f}_r \vert \mymat{x}, \myvec{\theta})$. To use Bayesian induction over the prior too, the hierarchical Bayesian model is considered. That is, we consider a different prior $ p(\myvec{f}_r \vert \mymat{x}, \myvec{\theta}^r)$ for each task and a hyperprior for $\myvec{\theta}^r$, $p(\myvec{\theta}^r \vert \myvec{\phi})$. Then, the distribution for latent parameters is expressed as
\begin{equation}
    \nonumber
    p (\myvec{f}_1, \ldots, \myvec{f}_\ntasks, \myvec{\theta}^1, \ldots, \myvec{\theta}^\ntasks, \myvec{\phi} \vert \mymat{x}^1, \ldots, \mymat{x}^\ntasks) \propto \prod_{r=1}^\ntasks p(\myvec{f}_r \vert \mymat{x}^r, \myvec{\theta}^r, \myvec{\phi}) p(\myvec{\theta}^r | \myvec{\phi}).
\end{equation}
In~\cite{YuTS05}, a Gaussian hyperprior $p(\myvec{\theta}^r | \myvec{\phi})$ is considered. Note that in this formulation, each task parameter $\myvec{\theta}^r$ is an independent from the rest of parameters $\myvec{\theta}^s, s \neq r$ given $\myvec{\phi}$ .
That is,
\begin{equation}
    \nonumber
    p(\myvec{\theta}^1, \ldots, \myvec{\theta}^\ntasks | \myvec{\phi}) = \prod_{r=1}^\ntasks p(\myvec{\theta}^r | \myvec{\phi}) .
\end{equation} 

%
% Learning Gaussian processes from multiple tasks (2005)
% Multi-Task Learning for Classification with Dirichlet Process Priors (2007)
% Bayesian multitask learning with latent hierarchies (2009)
Then, in~\cite{XueLCK07} a Dirichlet Process Prior is considered for modelling the task parameters. An explicit dependence is then defined over the task parameters
\begin{equation}
    \nonumber
    p(\myvec{\theta}^1, \ldots, \myvec{\theta}^\ntasks | \myvec{\phi}) = \prod_{r=1}^\ntasks p(\myvec{\theta}^r |\myvec{\theta}^{-r} , \myvec{\phi}) .
\end{equation} 
where $\myvec{\theta}^{-r}  = \set{\myvec{\theta}^s, s \neq r }$.
This formulation converts this model in a task-clustering approach, where the clusters are learned jointly with the task parameters $\myvec{\theta}$.
Following this approach of hierarchical Bayes,~\cite{Daume09} uses a prior for the task parameters $\myvec{\theta}^r$ that learns backwards a genealogy tree. That is, beginning at the leafs, which are the task parameters $\myvec{\theta}^r$, the branches merge until a common root to all the tasks. Thus, by selecting different thresholds or levels of this tree, we can obtain different clusters.

%%%%%%%%%%%%%%%%%%%%%%%%%%%%%%%%%%%%%%%%%%%%%%%%%%%%%%%%%%%%%%%%%%%%%%%%%%%%%%%%%%%%%
  %%%%%%%%%%%%%%%%%%%%%%%%%%             SECTION         %%%%%%%%%%%%%%%%%%%%%%%%%%
%%%%%%%%%%%%%%%%%%%%%%%%%%%%%%%%%%%%%%%%%%%%%%%%%%%%%%%%%%%%%%%%%%%%%%%%%%%%%%%%%%%%%

\section{Why does Multi-Task Learning work?}
% First works in Learning to Learn
% Overview of section
 % Ruder and Caruana
% 
\subsection{Learning to Learn: A notion of environment of tasks} % Baxter
% The first theoretical work on MTL... MTL as a Bias Learning Problem
Tipically in Machine Learning the goal is to find the best hypothesis $\hyp{x}{\param_0}$ from a space of hypothesis $\hypspace = \set{\hyp{x}{\param}, \param \in \paramspace}$, where $\paramspace$ is any set of parameters. This best candidate can be selected according to different inductive principles, which define a method of approximating a global function $f(x)$ from a training set:
$ \sample \defeq \set{(x_i, y_i),\; \idotsn} $
where $(x_i, y_i)$ are sampled from a distribution $\distf$.
%
In the classical statistics we find the Maximum Likelihood approach, where the goal is to estimate the density $\fun{x} = \cond{y}{x}$ and the hypothesis space is parametric, i.e. $\hypspace = \set{\hyp{x}{\param}, \param \in \paramspace \subset \reals^m}$. The learner select the parameter $\param$ that maximizes the probability of the data given the hypothesis.
%
Another more direct inductive principle is Empirical Risk Minimization (ERM), which is the most common one. In ERM the densities are ignored and an empirical error $\emprisk$ is minimized with the hope of minimizing the true expected error $\exprisk$, which would result in a good generalization. 
%
Several models use the ERM principle to generalize from data such as Neural Networks or Support Vector Machines. These methods are designed to find a good hypothesis $\hyp{x}{\param}$ from a given space $\hypspace$. The definition of such space $\hypspace$ define the bias for these problems. If $\mathcal{H}$ does not contain any good hypothesis, the learner will not be able to learn.
%

The best hypothesis space we can provide is the one containing only the optimal hypothesis, but this is the original problem that we want to solve. Therefore, in the single task scenario, there is no difference between bias learning and ordinary learning.
Instead, we focus on the situation where we want to solve multiple related tasks. In that case, we can obtain a good space $\hypspace$ that contains good solutions for the different tasks.
%
In~\cite{baxter2000model} an effort is made to define the concepts needed to construct the theory about inductive bias learning or Learning to Learn, which can be seen as a generalization of strict Multi-Task Learning. This is done by defining an environment of tasks and extending the work of~\cite{vapnik2013nature}, which defines the capacity of space of hypothesis, Baxter defines the capacity of a family of spaces of hypothesis.

% Review of concepts for STL in Supervised Learning
Before presenting the concepts defined for Bias Learning, and to establish an analogy to those of ordinary learning, we briefly review some statistical learning concepts.
\subsubsection*{Ordinary Learning}
In the ordinary statistical learning, some theoretical concepts are used:
\begin{itemize}
    \item an \emph{input space} $\Xspace$ and an \emph{output space} $\Yspace$,
    \item a \emph{probability distribution} $\distf$, which is unknown, defined over $\Xspace \times \Yspace$,
    \item a \emph{loss function} $\loss{\cdot}{\cdot}:\Yspace \times \Yspace \to \reals$, and
    \item a \emph{hypothesis space} $\hypspace = \set{\hyp{x}{\param}, \param \in \paramspace \subset \reals^m}$ with hypothesis $\hyp{\cdot}{\param}: \Xspace \to \Yspace$.
\end{itemize}
The goal for the learner is to select a hypothesis $\hyp{x}{\param} \in \mathcal{H}$, or equivalently $\param \in \paramspace$, that minimizes the expected risk
$$ \exprisk(\param) =  \int_{\Xspace \times \Yspace} \loss{\hyp{x}{\param}}{y} d\distf(x, y) .$$
The distribution $\distf$ is unknown, but we have a training set $\sample = \{(x_1, y_1), \ldots, (x_\nsamples, y_\nsamples)\}$ of samples drawn from $\distf$. 
The approach is then is to apply the ERM inductive principle, that is to minimize the empirical risk
$$ \emprisk(\param) = \frac{1}{\nsamples} \sum_{i=1}^\nsamples l(h(x_i), y_i).$$
Thus, a learner $\mathcal{A}$ maps the set of training samples to a set of hypothesis:
\begin{equation}
    \nonumber
    \mathcal{A} : \bigcup {(\Xspace \times \Yspace)^\nsamples} \to \hypspace.
\end{equation}
Although $\emprisk(\param)$ is an unbiased estimator of $\exprisk(\param)$, it has been shown~\cite{vapnik2013nature} that this approach, despite being the most evident one, is not the best principle that can be followed.
This has relation with two facts: the first one is that the unbiased property is an asymptotical one, the second one has to do with overfitting.
Vapnik answers to the question of what can be said about $\exprisk$ when $\param$ minimizes $\emprisk(\param)$, and moreover, his results are valid also for small number of training samples $n$.
More specifically, Vapnik sets the sufficient and necessary conditions for the consistency of an inductive learning process, i.e. for $\emprisk(\param) \toprob \exprisk(\param) $ uniformly. Vapnik also defines the capacity of a hypothesis space and use it to derive bounds on the rate of this convergence for any $\param \in \paramspace$ and, more importantly, bounds on the difference $\inf_{\param \in \paramspace} \emprisk(\param) - \inf_{\param \in \paramspace} \exprisk(\param)$.
Under some general conditions, he proves that
\begin{equation}\label{eq:ordinary_generalization_bound}
    \inf_{\param \in \paramspace} \emprisk(\param) - \inf_{\param \in \paramspace} \exprisk(\param) \leq B(\nsamples/\vcdim{\hypspace})
\end{equation}
where $B$ is some non-decreasing function and $\vcdim{\hypspace}$ is the capacity of the space $\hypspace$, also named the VC-dimension $\hypspace$. This means that the generalization ability of a learning process can be controlled in terms of two factors:
\begin{itemize}
    \item The number of training samples $\nsamples$. A greater number of training samples assures a better generalization of the learning process.This looks intuitive and could be already inferred from the asymptotical properties. 
    \item The VC-dimension $\vcdim{\hypspace}$ of the hypothesis space $\hypspace$, which is desirable to be small. This term is not intuitive and is the most important term in Vapnik theory.
\end{itemize}
The VC-dimension measures the capacity of a set of hypothesis $\hypspace$. 
%In the case of a set of indicator functions, it is the maximum number of vectors $x_1, \ldots, x_\vcdim{\hypspace}$ that can be shattered (in two classes) by functions of this set. In the case of real functions, it is defined as the VC-dimension of the following set of indicator functions $ I(x, \param, \beta) = \myvec{1}_{\{\hyp{x}{\param} - \beta\}} $.
If the capacity of the set $\hypspace$ is too large, we may find a
hypothesis $\hyp{x}{\opt{\param}}$ that minimizes $\emprisk$ but does not 
generalize well and therefore, does not minimize $\exprisk$. This is the 
overfitting problem. 
On the other side, if we use a simple $\hypspace$, 
with low capacity, we could be in a situation where there is not a good hypothesis $\hyp{x}{\param} \in \hypspace$, so the empirical risk $\inf_{\param \in \paramspace} \exprisk$ is too large. This is the underfitting problem.

% The Structural Risk Minimization (SRM) as an inductive principle, proposed by Vapnik (as opposed to the ERM), tries to find a tradeoff between minimizing $\emprisk$ and minimizing $\vcdim{\hypspace}$. The idea is to define an admissible structure, that is a sequence of hypothesis spaces:
% $$ \mathcal{H}_1 \subset \mathcal{H}_2 \subset \ldots \subset \mathcal{H}_k \subset \ldots $$ 
% where their VC-dimensions are ordered:
% $$ d_1 < d_2 < \ldots < d_k < \ldots$$
% where $d_i$ is the VC-dimension of $\mathcal{H}_i$.
% SRM selects the hypothesis $\hyp{x}{\param^*} = \hyp{x}{\param_i^*} \in \hypspace_i$ that obtains the best bound for the actual risk $\exprisk$.
% This admissible structure can be built in various ways. In Neural Networks in can be the built by increasing the number of hidden layers. In other methods, such as SVM or Ridge Regression, this is done by decreasing the regularization.
% However, this SRM principle is usually replaced by a cross-validation (CV) procedure.

% Support Vector Machines, which are the most representative models of this theory, use the VC-dimension also in other way (apart from the SRM principle or the CV procedure). The goal of finding the optimal hyperplane, that is, that with the maximum margin between the classes, has its motivation in the fact the set of such type of hypothesis have a lower VC-dimension that the set of all hyperplanes do.


% Extension to MTL
\subsubsection*{Bias Learning: Concept and Components}
In ~\cite{baxter2000model} two main concepts are presented: the \emph{family of hypothesis spaces} and an \emph{environment} of related tasks. 
For simplicity we write $\hypf(x)$  instead of $\hypf(x, \param)$, and since $\param$ completely defines $\hypf$, we also substitute $\param$ by $\hypf$ for an easier notation.
Using these concepts, the bias learning problem has the following components:
\begin{itemize}
    \item an \emph{input space} $\Xspace$ and an \emph{output space} $\Yspace$,
    \item an \emph{environment} $(\bprobspace, \bdistf)$ where $\bprobspace$ is a set of distributions $P$ defined over $\Xspace \times \Yspace$, and we can sample from $\bprobspace$ according to a distribution $\bdistf$,
    \item a \emph{loss function} $\loss{\cdot}{\cdot}:\Yspace \times \Yspace \to \reals$, and
    \item a \emph{family of hypothesis spaces} $\hypspacef = \set{\hypspace_\param, \param \in \paramspace}$, where each element $\hypspace_\param$ is a set of hypothesis.
\end{itemize}
Analogous to ordinary learning, the goal is to minimize the expected risk, defined as
\begin{equation}\label{eq:biaslearn_exprisk}
    \bexprisk(\param) = \int_{\bprobspace} \inf_{\hypf \in \hypspace_\param} \risk_P(\hypf) d\bdistf(P) = \int_{\bprobspace} \inf_{\hypf \in \hypspace_\param} \int_{\Xspace \times Y} l(h(x), y) dP(x, y) d\bdistf(P).
\end{equation}
Again, we do not know $\bprobspace$ nor $\bdistf$, but we have a training set samples from the environment $(\bprobspace, \bdistf)$ obtained in the following way:
\begin{enumerate}
    \item Sample $T$ times from $\bdistf$ obtaining $P_1, \ldots, P_T \in \bprobspace$
    \item For $r=1, \ldots, T$ sample $m$ pairs $\sample_r = \{(x_1^r, y_1^r), \ldots, (x_m^r, y_m^r)\}$ according to $P_r$ where $(x_i^r, y_i^r) \in X \times Y$ .
\end{enumerate}
We obtain a sample $z=\{(x_i^r, y_i^r), r=1,\; i=1, \ldots, \;m=1, \ldots, T\}$, with $\npertask$ examples from $\ntasks$ different learning tasks, and
\begin{equation}
    \nonumber
    \bsample \defeq 
    \begin{matrix}
        (x_1^1, y_1^1) & \ldots & (x_m^1, y_m^1) \\
        \vdots & \ddots & \vdots \\
        (x_1^\ntasks, y_1^\ntasks) & \ldots & (x_m^\ntasks, y_m^\ntasks) \\
    \end{matrix}
\end{equation}
is named as a $(\ntasks, \npertask)$-sample.
Using $\bsample$ we can define the empirical loss as
\begin{equation}\label{eq:biaslearn_emprisk}
    \bemprisk(\param) = \sum_{r=1}^\ntasks \inf_{\hypf \in \hypspace_\param} \hat{\risk}_{\sample_r}(\hypf) = \sum_{r=1}^\ntasks \inf_{\hypf \in \hypspace_\param} \sum_{i=1}^m l(\hypf(x_i^r), y_i^r),
\end{equation}
which is an average of the empirical losses of each task. 
Note, however, that in the case of the bias learner, this estimate is biased, since $\risk_{P_r}(\hypf)$ does not coincide with $\hat{\risk}_{\sample_r}(\hypf)$. 
Putting all together, a bias learner $\mathcal{A}$ maps the set of all $(\ntasks, \npertask)$-samples to a family of hypothesis spaces:
\begin{equation}
    \nonumber
    \mathcal{A} : \bigcup {(\Xspace \times \Yspace)^{(\ntasks, \npertask)}} \to \hypspacef.
\end{equation}
%

To follow an analogous path to that of ordinary learning, the milestones in bias learning theory should include:
\begin{itemize}
    \item Checking the consistency of the Bias Learning methods, i.e. proving that $\bemprisk(\param)$ converges uniformly in probability to $\bexprisk(\param)$.
    \item Defining a notion of capacity of hypothesis space families $\hypspacef$.
    \item Finding a bound of $\bemprisk(\param) - \bexprisk(\param)$ for any $\param$ using the capacity of the hypothesis space family. If possible, finding also a bound for $\inf_{\param \in \paramspace}\bemprisk(\param) - \inf_{\param \in \paramspace} \bexprisk(\param)$.
\end{itemize}
To achieve these goals some previous definitions are needed. From this point, since any $\hypspace$ is defined by a $\param \in \paramspace$, we omit $\param$ and write just $\hypspace$ for simplicity.
%
\subsubsection*{Bias Learning: Capacities and Uniform Convergence}
% Pseudo-metrics, Covering numbers and Capacities
In first place, a \emph{sample-driven} pseudo-metric of $(\ntasks, 1)$-empirical risks is defined.
Consider a sequence of $\ntasks$ probabilities $\bprobseq = (P_1, \ldots, P_\ntasks)$ sampled from $\bprobspace$ according the the distribution $\bdistf$. 
\begin{definition}[\emph{sample-driven} pseudometric]
    \label{def:sample_pseudometric}
    Consider also the set of sequences of $\ntasks$ hypothesis 
$$\hypspace^\ntasks \defeq \set{ \myvec{\hypf} = (\hypf_1, \ldots, \hypf_\ntasks) , \hypf_1, \ldots, \hypf_\ntasks \in \hypspace} .$$
We can define then the set of $(\ntasks, 1)$-empirical risks as 
$$\hypspace^\ntasks_\lossf \defeq \set{ \myvec{\hypf}_\lossf(x_1, y_1, \ldots, x_\ntasks, y_\ntasks) = \sum_{r=1}^\ntasks \lossf(\hypf(x_i), y_i) , \hypf_1, \ldots, \hypf_\ntasks \in \hypspace} $$
The family of the set of $\ntasks$-risks of hypothesis is then $\bsetsample = \bigcup_{\hypspace \in \hypspacef} \hypspace^\ntasks$. Now we can define
\begin{equation}
    \nonumber
    \begin{aligned}
        \dist{\bprobseq}(\myvec{\hypf}_\lossf, \myvec{\hypf'}_\lossf) = \int_{(\Xspace \times \Yspace)^{\ntasks}} &\abs{\myvec{\hypf}_\lossf(x_1, y_1, \ldots, x_\ntasks, y_\ntasks) - \myvec{\hypf'}_\lossf(x_1, y_1, \ldots, x_\ntasks, y_\ntasks)} \\ 
        & d{P_1}(x_1, y_1) \ldots d{P_\ntasks}(x_\ntasks, y_\ntasks)
    \end{aligned}
\end{equation}
for $\myvec{\hypf}_\lossf, \myvec{\hypf'}_\lossf \in \hypspace_\lossf, \hypspace'_\lossf$ as a pseudo-metric in $\hypspacef^\ntasks$.
\end{definition}
%

Then, a \emph{distribution-driven} pseudo-metric is defined. 
\begin{definition}[\emph{distribution-driven} pseudo-metric]
    \label{def:dist_pseudometric}
    Given a distribution $P$ on $\Xspace \times \Yspace$. Consider the set of infimum expected risk for each $\hypspace$:
\begin{equation}
    \nonumber
    \hypspace^* \defeq \inf_{\hypf \in \hypspace} \risk_P(\hypf).
\end{equation}
The family of such sets is defined as 
$\bsetdist = \set{\hypspace^*, \hypspace \in \hypspacef}$.
The pseudo-metric in this space is given by $Q$:
\begin{equation}
    \nonumber
    \dist{\bdistf} = \int_\bprobspace \abs{\hypspace_1^* - \hypspace_2^*} d\bdistf
\end{equation}
\end{definition}
With these two pseudo-metrics, two capacities for families of hypothesis spaces are defined. For that the definition of $\epsilon$-cover is needed. 
\begin{definition}[$\epsilon$-cover]
    \label{def:epsilon_cover}
    Given a pseudo-metric $\dist{S}$ in a space $\mathcal{S}$, 
a set of $l$ elements $s_1, \ldots, s_l \in \mathcal{S}$ is an $\epsilon$-cover of $\mathcal{S}$ if 
$ \dist{S}(s, s_i) \leq \epsilon $
for some $i=1, \ldots, l$.  Let $\mathcal{N}(\epsilon, \mathcal{S}, \dist{S})$ denote the size of the smallest $\epsilon$-cover.
\end{definition}
%
Then, we can define the following capacities of a family space $\hypspacef$:
\begin{itemize}
    \item The \emph{sample-driven capacity} $\capacity{\epsilon}{\bsetsample} \defeq \sup_{\bprobseq} \mathcal{N}(\epsilon, \hypspacef^\ntasks, \dist{\bprobseq})$.
    \item The \emph{distribution-driven capacity} $\capacity{\epsilon}{\bsetdist} \defeq \sup_Q \mathcal{N}(\epsilon, \bsetdist, \dist{\bdistf})$.
\end{itemize}

% Uniform Convergence for bias learners (Comparison with Vapnik)
Using these capacities, the convergence (uniformly over all $\hypspace \in \hypspacef$) of bias learners can be proved~\cite[Theorem~2]{baxter2000model}. Moreover, the bias expected risk is bounded
\begin{equation}
    \nonumber
    \bemprisk(\hypspace) \leq \bexprisk(\hypspace) + \epsilon
\end{equation}
with probability $1 - \eta$, given sufficiently large $\ntasks$ and $\npertask$, 
\begin{equation}
    \nonumber
    \ntasks \geq \max \left( \frac{256}{\ntasks \epsilon^2} \log\frac{8\capacity{\frac{\epsilon}{32}}{\bsetdist}}{\eta} , \frac{64}{\epsilon^2}\right)  , \; \npertask \geq \max \left( \frac{256}{\ntasks \epsilon^2} \log\frac{8\capacity{\frac{\epsilon}{32}}{\bsetsample}}{\eta} , \frac{64}{\epsilon^2}\right) .
\end{equation}
It should be noted that the bound for $\npertask$ is inversely proportional to $\ntasks$, that is, the more tasks we have, the less samples we need for each task. 


% Multi-Task Learning (strictly speaking, with fixed tasks)
\subsubsection*{Multi-Task Learning}
The previous result is a result for pure Bias Learning, where we have an $(\bsetdist, \bdistf)$-environment of tasks. In Multi-Task Learning, we have a fixed number of tasks $\ntasks$ and a fixed sequence of distributions $\bprobseq = (P_1, \ldots, P_\ntasks)$, where $P_i$ is a distribution over $(\Xspace \times \Yspace)^\npertask$. The goal is not learning a hypothesis space $\hypspace$ but a sequence of hypothesis $\myvec{\hypf} = (\hypf_1, \ldots, \hypf_\ntasks), \hypf_1, \ldots, \hypf_\ntasks \in \hypspace $. Thus, the Multi-Task expected risk is
\begin{equation}\label{eq:mtlearn_exprisk}
    \risk_{\bprobseq}(\myvec{\hypf}) = \sum_{r=1}^\ntasks \risk_{P_r}(\hypf_r)  = \sum_{r=1}^\ntasks \int_{\Xspace \times Y} l(\hypf_r(x), y) d{P_r}(x, y),
\end{equation}
and the empirical risk is defined as
\begin{equation}\label{eq:mtlearn_emprisk}
    \bemprisk(\myvec{\hypf}) = \sum_{r=1}^\ntasks \hat{\risk}_{\sample_r}(\hypf_r) = \sum_{r=1}^\ntasks \sum_{i=1}^m l(\hypf_r(x_i^r), y_i^r).
\end{equation}
A similar result to that of Bias Learning is given for Multi-Task Learning~\cite[Theorem~4]{baxter2000model}:
\begin{equation}
    \nonumber
    \bemprisk(\myvec{\hypf}) \leq \risk_{\bprobseq}(\myvec{\hypf}) + \epsilon,
\end{equation}
with probability $1 - \eta$ given that the number of samples per task
\begin{equation}
    \nonumber
    \npertask \geq \max \left( \frac{64}{\ntasks \epsilon^2} \log\frac{4\capacity{\frac{\epsilon}{16}}{\bsetsample}}{\eta} , \frac{16}{\epsilon^2}\right).
\end{equation}
Observe that we do not need the \emph{distribution-driven} capacity in this case, just the \emph{sample-driven} capacity.
% Feature Learning
\subsubsection*{Feature Learning}
Feature Learning is a common way to encode bias. The most popular example are Neural Networks, where all the hidden layers can be seen as a Feature Learning engine that learns a mapping from the original space to a space with ''strong'' features.
In general, a set of ''strong'' feature maps is defined as $\mathcal{F} = \set{f, f: \Xspace \to \Fspace}$. Using these features, functions $g \in \mathcal{G}$ (which are tipically simple) are built:
$\Xspace \to_f \Fspace \to_g \Yspace$.
Thus, for each map $f$, the hypothesis space can be expressed as 
$\hypspace_f = \set{h = \mathcal{G} \circ f, g \in \mathcal{G}}$
, and the family of hypothesis spaces is 
$\hypspacef = \set{\hypspace_f, f \in \mathcal{F}}$.
Now, the Bias Learning problem is the problem of finding a good mapping $f$.
It is proved~\cite[Theorem~6]{baxter2000model} that in the Feature Learning case the capacities of $\hypspacef$ can be bounded by the capacities of $\mathcal{F}$ and $\mathcal{G}$ as
\begin{align*}
    \capacity{\epsilon}{\bsetsample} &\leq \capacity{\epsilon_1}{\mathcal{G}^\ntasks}^\ntasks \capf_{\mathcal{G}_\lossf}(\epsilon_2, \mathcal{F}) , \\
    \capacity{\epsilon}{\bsetdist} &\leq \capf_{\mathcal{G}_\lossf}(\epsilon, \mathcal{F})
\end{align*}
with $\epsilon = \epsilon_1 + \epsilon_2 $. Here, $\capf_{\mathcal{G}_\lossf}(\epsilon, \mathcal{F})$ is defined as 
$\capf_{\mathcal{G}_\lossf}(\epsilon, \mathcal{F}) \defeq \sup_P \mathcal{N}(\epsilon, \mathcal{F}, \dist{[P, \mathcal{G}_\lossf]})$, where
$$ \dist{[P, \mathcal{G}_\lossf]}(f, f') = \int_{\Xspace \times \Yspace} sup_{g \in \mathcal{G}} \abs{\loss{g \circ f(x)}{y} - \loss{g \circ f'(x)}{y}} dP(x, y)$$
is a pseudo-metric. Using these results alongside those presented for Bias Learning is useful to establish bounds for Feature Learning models like Neural Networks.

% Generalized VC-dim , Theorems 12, 13 and Corollary 13
\subsubsection*{Generalized VC-Dimension for Multi-Task Learning}
The concepts presented until now rely on the concepts of two capacities of a family of hypothesis spaces $\hypspacef$ to establish bounds in the difference $\bemprisk(\myvec{\hypf}) - \bexprisk(\myvec{\hypf})$, that is, the probability of large deviations between the empirical and expected risks for a given hypothesis sequence. However, it would be more interesting to establish some bounds between the empirical error and the \emph{best expected error}.
To overcome this limitations, a generalized VC-dimension is developed in~\cite{baxter2000model} for Multi-Task Learning with Boolean hypothesis.
%
\begin{definition}\label{def:gen_vcdim}
    Let $\hypspace$ be a space of boolean functions and $\hypspacef$ a boolean hypothesis space family. Denote the set of $\ntasks \times \npertask$ matrices in $\Xspace$ as $\Xspace^{\ntasks \times \npertask}$
For each $\mymat{x} \in \Xspace^{\ntasks \times \npertask}$ and each $\hypspace \in \hypspacef$ define the set of binary $\ntasks \times \npertask$ matrices
\begin{equation}
    \nonumber
    \hypspace_{\vert \mymat{x}} \defeq \set{\begin{pmatrix}
        \hypfun{x_1^1} & \ldots & \hypfun{x_\npertask^1} \\
        \vdots & \ddots & \vdots \\
        \hypfun{x_1^\ntasks} & \ldots & \hypfun{x_\npertask^\ntasks} \\
    \end{pmatrix}, \hypf \in \hypspace} ,
\end{equation}
and the corresponding family of such sets as
\begin{equation}
    \nonumber
    \hypspacef_{\vert \mymat{x}} = \bigcup_{\hypspace \in \hypspacef}  \hypspace_{\vert \mymat{x}}.
\end{equation}
For each $\ntasks, \npertask \geq 0$ define the number of binary matrices obtainable with $\hypspacef$ as
\begin{equation}
    \nonumber
    \Pi_\hypspacef(\ntasks, \npertask) \defeq \max_{\mymat{x} \in \Xspace^{\ntasks \times \npertask}} \cardinal{ \hypspacef_{\vert \mymat{x}} }.
\end{equation}
Note that $\Pi_\hypspacef(\ntasks, \npertask) \leq 2^{\ntasks \npertask}$ and if $\Pi_\hypspacef(\ntasks, \npertask) \leq 2^{\ntasks \npertask}$ we say that $\hypspacef$ shatters $\Xspace^{\ntasks \times \npertask}$.
For each $n > 0$ define
\begin{align}
    d_\hypspacef(\ntasks) &\defeq \max_{m: \Pi_\hypspacef(\ntasks, \npertask) = 2^{\ntasks \npertask}} m , \nonumber \\
    \overline{d}(\hypspacef) &\defeq \vcdim{\hypspacef^1} = \vcdim{\bigcup_{\hypspace \in \hypspacef} \hypspace}  ,  \nonumber \\
    \underline{d}(\hypspacef) &\defeq \max_{\hypspace \in \hypspacef} \vcdim{\hypspace}  . \nonumber
\end{align}
Here, $d_\hypspacef(\ntasks)$ is the generalized VC-dimension, and
\begin{equation}
    \nonumber
    d_\hypspacef(\ntasks) \geq \max\left(\floor*{\frac{\overline{d}(\hypspacef)}{\ntasks}}, \underline{d}(\hypspacef) \right)
\end{equation}
where it can be observed that 
\begin{equation}
    \label{eq:genvc_inequalities}
    \overline{d}(\hypspacef) \geq  d_\hypspacef(\ntasks)\geq \underline{d}(\hypspacef).
\end{equation}
\comm{Proof?}
\end{definition}
% Look at Ben-David version
Now we can present the relevant result expressed in~\cite[Corollary~13]{baxter2000model}.
\begin{theorem}\label{th:baxter_vcdim}
    Given a sequence $\bprobseq = (P_1, \ldots, P_\ntasks)$ on $(\Xspace \times \set{0, 1})^\ntasks$, and a sample $\bsample$ from this distribution. Consider also a sequence $\myvec{\hypf} = (\hypf_1, \ldots, \hypf_\ntasks)$ of boolean hypothesis $\hypf_i \in \hypspace$, then for every $\epsilon > 0$
% \begin{equation}
%     \nonumber
%     \bexprisk(\myvec{\hypf}) \leq \bemprisk(\myvec{\hypf}) + \epsilon,
% \end{equation}
\begin{equation}
    \nonumber
    \abs{\risk_{\bprobseq}(\myvec{\hypf}) - \bemprisk(\myvec{\hypf})} \leq \epsilon,
\end{equation}
with probability $1 - \eta$ given that the number of samples per task
\begin{equation}
    \label{eq:bound_npertask_genvcdim}
    \npertask \geq \frac{88}{\epsilon^2} \left[2 d_\hypspacef(\ntasks) \log \frac{22}{\epsilon} + \frac{1}{\ntasks}\log\frac{4}{\eta} \right] .
\end{equation}
\end{theorem}
Here, since $d_\hypspacef(\ntasks) \geq d_\hypspacef(\ntasks+1)$, it is easy to see that as the number of task $\ntasks$ increases, the number of examples needed per task can decrease. 
Moreover, as shown in~\cite[Theorem~14]{baxter2000model}, if this bound on $\npertask$ is not fulfilled, then we can always find a sequence of distributions $\bprobseq$ such that
\begin{equation}
    \nonumber
    \inf_{\hypf \in \hypspace} \bemprisk(\myvec{\hypf}) > \inf_{\hypf \in \hypspace} \risk_{\bprobseq}(\myvec{\hypf}) + \epsilon .
\end{equation}
With this results we can see that the condition~\eqref{eq:bound_npertask_genvcdim} has some important properties:
\begin{itemize}
    \item It is a computable bound, given that we know how to compute $d_\hypspacef(\ntasks)$.
    \item It provides a sufficient condition for the uniform convergence (in probability) of the empirical risk to the expected risk.
    \item It provides a necessary condition for the consistency of Multi-Task Learners, i.e. uniform convergence of the best empirical risk to the best expected risk.
\end{itemize}

\subsubsection*{Conclusion}
In~\cite{baxter2000model} several new concepts are developed. The $(\bprobspace, \bdistf)$-environment of tasks is useful to characterize the concept of related tasks. Moreover, using this definition, Baxter is able to give some important results of uniform convergence in the Bias Learning paradigm. From this general view, Multi-Task Learning
is a particular case and the uniform convergence results are also valid. The Feature Learning approach, which can be seen as a more particular method of Multi-Task Learning has some interesting results splitting the analysis into the feature learning process and the construction of models over these features. Finally, the most important result is the definition of a generalized VC-dimension and the uniform convergence of Multi-Task Learning models using this concept. Although this is a result only valid for boolean hypothesis, it helps to shed some light on Multi-Task Learning and the reasons of its effectiveness.

\subsection{Learning with Related Tasks} % Ben-David
% Baxter gives the foundation for a theoretical work of a framework where the tasks share a common learning bias...
Using the work of~\cite{baxter2000model} as the foundation, several important notions and results are presented in~\cite{Ben-DavidB08} for boolean hypothesis functions defined over $\Xspace \times \set{0, 1}$.
% task relatedness
One of the main contributions of this work is a notion of task relatedness. In~\cite{baxter2000model} the tasks are related by sharing a common inductive bias that can be learned. In~\cite{Ben-DavidB08} a precise mathematical definition for task relatedness is given.
% task individual risks
The other important contribution is the focus on the individual risk of each task. In~\cite{baxter2000model} all the results are given for the Multi-Task empirical and expected risks, which are an average of the risks of each task. However, bounding this average does not bound the risk of each particular task. This is specially relevant if we are in a Transfer Learning scenario, where there is a target task that we want to solve and the remaining tasks can be seen as an aid to improve the performance in the target.

% F-related tasks
\subsubsection*{A Notion of Task Relatedness: $\frelset$-Related Tasks}
The main concept for the theory developed in~\cite{Ben-DavidB08} is a set of $\frelset$ of transformations $\frelf: \Xspace \to \Xspace$. Given a probability distribution $\distf$ over $\Xspace \times \set{0, 1}$, a set of tasks with distributions $P_1, \ldots, P_\ntasks$ are $\frelset$-related if, for each task there exists some $\frelf_i \in \frelset$ such that $P_i = \frelf_i(\distf)$.

\begin{definition}[$\frelset$-related task]\label{def:frel_tasks}
    Consider a measurable space $(\Xspace, \mathcal{A})$ and the corresponding measurable product space $(\Xspace \times \set{0, 1}, \mathcal{A} \times \powerset{\set{0, 1}})$. Consider $P$ a probability distribution over this product space and a function $\frelf: \Xspace \to \Xspace$, then we define the distribution $\frel{P}$ such that for any $S \in \mathcal{A}$,
    $$ \frel{P}(S) \defeq P(\set{(f(x), b), (x, b) \in S }).$$
    %
    Let $\frelset$ be a set of transformations $\frelf: \Xspace \to \Xspace$, and let $P_1, P_2$ be distributions over $(\Xspace \times \set{0, 1}, \mathcal{A} \times \powerset{\set{0, 1}})$, then the distributions $P_1, P_2$ are $\frelset$-related if $\frel{P_1}= P_2$ or $\frel{P_2} = P_1$ for some $\frelf \in \frelset$.
    % \begin{itemize}
    %     \item The distributions $P_1, P_2$ are $\frelset$-related if $\frel{P_1}= P_2$ or $\frel{P_2} = P_1$ for some $\frelf \in \frelset$.
    %     \item Two samples $\sample_1, \sample_2$ sampled from $P_1, P_2$ respectively are $\frelset$-related if $P_1, P_2$ are $\frelset$-related.
    % \end{itemize}
\end{definition}
This notion establishes a clear definition of related tasks but we are interested in how a learner can use this relatedness to improve the learning process.
For that, considering that $\frelset$ is a group under function composition, we regard at the action of the group $\frelset$ over the set of hypothesis $\hypspace$. This action defines the following equivalence relation in $\hypspace$:
$$ \hypf_1 \sim_\frelset \hypf_2 \iff \exists \frelf \in \frelset,  \hypf_1 \circ \frelf = \hypf_2 .$$
%
This equivalence relation defines equivalence classes $\equivclass{\hypf}$, that is let $h' \in \hypspace$ be an hypothesis, then $h' \in \equivclass{\hypf}$ iff $h' \sim_\frelset h$. 
We consider the quotient space 
$$\hypspace_\frelset \defeq \hypspace / \sim_\frelset = \set{\equivclass{\hypf}, \hypf \in \hypspace}.$$
It is important to observe that $\hypspace_\frelset = \hypspacef'$ is a hypothesis space family, since it is a set of equivalence classes $\equivclass{\hypf} = \hypspace'$, which are set of hypothesis.
%

\subsubsection*{The Multi-Task Empirical Risk Minimization}
This equivalence classes are useful to divide the learning process in two stages, this is called the \emph{Multi-Task ERM}. Consider the samples $\sample_1, \ldots, \sample_\ntasks$ from $T$ different tasks, then
\begin{enumerate}
    \item Select the best hypothesis class $\equivclass{\hypf^\frelset} \in \hypspace_\frelset$:
    \begin{equation}
        \nonumber
        \equivclass{\hypf^\frelset} \defeq \min_{\equivclass{\hypf} \in \hypspace_\frelset} \inf_{\hypf_1, \ldots, \hypf_\ntasks \in \equivclass{\hypf}} \frac{1}{\ntasks} \sum_{r=1}^\ntasks \hat{\risk}_{\sample_i}(\hypf_i) ,
    \end{equation}
    \item Select the best hypothesis $h^\diamond$ for the target task (without loss of generality, consider the first one):
    \begin{equation}
        \nonumber
        h^\diamond = \inf_{\hypf \in \equivclass{\hypf^\frelset}} \hat{\risk}_{\sample_1}(\hypf) .
    \end{equation}
\end{enumerate}

For example, consider the handwritten digits recognition problem, we might integrate $T$ different datasets designed in different conditions. Each dataset have been created using certain conditions of light and some specific scanner for getting the images. Even different pens or pencils might be influential in the stroke of the numbers. All these conditions are the $\frelset$ transformations, and each $\frelf \in \frelset$ generate a different bias for the dataset. However, there exists a probability for ''pure'' digits, e.g. the pixels of digit one have higher probability around a line in the middle of the picture than in the sides. This ''pure'' probability distribution $P_0$ and all the distributions $P_1, \ldots, P_T$ from which our datasets have been sampled might be $\frelset$-related among them and with $P_0$. If we first determine the $\frelset$-equivalent class of hypothesis $\equivclass{\hypf}$ suited for digit recognition in the first stage, then it will be easier to select $\hypf_1, \ldots, \hypf_\ntasks \in \equivclass{\hypf}$ for each dataset in the second one.

% Results for F-related tasks (Theorems 2, 3)
\subsubsection*{Bounds for $\frelset$-Related Tasks}
% Relation with Baxter work
The results of Theorem~\ref{th:baxter_vcdim} can be applied to the hypothesis quotient space of equivalent classes $\hypspace_\frelset$. However the following results is needed first.
%
Let $P_1, P_2$ be $\frelset$-related distributions, then this statement can be proved~\cite[Lemma~2]{Ben-DavidB08}:
\begin{equation}
    \label{eq:ben-david_lemma2}
    \inf_{\hypf \in \hypspace} \risk_{P_1}(\hypf) = \inf_{\hypf \in \hypspace} \risk_{P_2}(\hypf).
\end{equation}
This indicates that the the expected risk is invariant under transformations of $\frelset$.
Now, one of the main results~\cite[Theorem~2]{baxter2000model} can be given.
\begin{theorem}\label{th:ben-david_th2}
    Let $\frelset$ be a set of transformations $\frelf: \Xspace \to \Xspace$ that is a group under function composition. Let $\hypspace$ be a hypothesis space so that $\frelset$ acts as a group over $\hypspace$, and consider the quotient space $\hypspace_\frelset = \set{\equivclass{\hypf}, \hypf \in \hypspace}$.
    Consider $\bprobseq = (P_1, \ldots, P_\ntasks)$ a sequence of $\frelset$-related distributions over $\Xspace \times \set{0, 1}$, and $\myvec{z} = (\sample_1, \ldots, \sample_\ntasks)$ the corresponding sequence of samples  where $\sample_i$ is sampled using $P_i$, then for every $\equivclass{\hypf} \in \hypspace_\frelset$ and $\epsilon > 0$ 
    \begin{equation}
        \nonumber
        \abs{\inf_{\hypf_1, \ldots, \hypf_\ntasks \in \equivclass{\hypf}} \frac{1}{\ntasks} \sum_{r=1}^\ntasks \hat{\risk}_{\sample_r}(\hypf_r) - \inf_{\hypf' \in \equivclass{\hypf}}\risk_{P_1}(\hypf')}  \leq \epsilon
    \end{equation}
    with probability greater than $\eta$ if the number of samples from each distribution satisfies
    \begin{equation}
        \label{eq:ben-david_sample_inequality}
        \cardinal{\sample_i} \geq  \frac{88}{\epsilon^2} \left[2 d_{\hypspace_\frelset}(\ntasks) \log \frac{22}{\epsilon} + \frac{1}{\ntasks}\log\frac{4}{\eta} \right].
    \end{equation}
\end{theorem}
Note that, in contrast to Theorem~\ref{th:baxter_vcdim}, this result bounds the expected risk of a single task, not the average risk. This is the consequence of applying Theorem~\ref{th:baxter_vcdim} and substituting the average empirical error using the result from~\eqref{eq:ben-david_lemma2}.
Also observe that here the hypothesis space family used is the quotient space $\hypspace_\frelset$, and the VC-dimension of such family is used.
%
Using this result, a bound for learners using the Multi-Task ERM principle is given~\cite[Theorem~3]{Ben-DavidB08}
\begin{theorem}\label{th:ben-david_th3}
    Consider $\frelset$ and $\hypspace$ as in the previous theorem. Consider also the previous sequences of distributions $(P_1, \ldots, P_\ntasks)$ and corresponding samples $(\sample_1, \ldots, \sample_\ntasks)$. Consider $\underline{d}(\hypspace_\frelset) = \max_{\hypf \in \hypspace} \vcdim{\equivclass{\hypf}}$.
    Let $h^\diamond$ be the hypothesis selected using the Multi-Task ERM principle, then for every $\epsilon_1, \epsilon_2 > 0$
    \begin{equation}
        \nonumber
        {\hat{\risk}_{\sample_1}(\hypf^\diamond) - \inf_{\hypf' \in \hypspace}\risk_{P_1}(\hypf')}  \leq 2(\epsilon_1 + \epsilon_2)
    \end{equation}
    with probability greater than $\eta$ if
    \begin{equation}
        \label{eq:ben-david_th3_ineq1}
        \cardinal{\sample_1} \geq  \frac{64}{\epsilon^2} \left[2 \underline{d}(\hypspace_\frelset) \log \frac{12}{\epsilon} + \frac{1}{\ntasks}\log\frac{8}{\eta} \right], \; 
    \end{equation}
    and for $i \neq 1$
    \begin{equation}
        \label{eq:ben-david_th3_ineq2}
        \cardinal{\sample_i} \geq  \frac{88}{\epsilon^2} \left[2 d_{\hypspace_\frelset}(\ntasks) \log \frac{22}{\epsilon} + \frac{1}{\ntasks}\log\frac{8}{\eta} \right] .
    \end{equation}
\end{theorem}
The idea of the proof of this theorem helps to understand how using different tasks can help to improve the performance in the target task. 
Consider $\hypf^* = \inf_{\hypf \in \hypspace} \risk_{P_1}(\hypf)$ the best hypothesis for the $P_1$ distribution.
According to Theorem~\ref{th:ben-david_th2}, for $\equivclass{\hypf^*}$ we have that
\begin{equation}
    \nonumber
    {\inf_{\hypf_1, \ldots, \hypf_\ntasks \in \equivclass{\hypf^*}} \frac{1}{\ntasks} \sum_{r=1}^\ntasks \hat{\risk}_{\sample_r}(\hypf_r) \leq \inf_{\hypf' \in \equivclass{\hypf^*}}\risk_{P_1}(\hypf')}  + \epsilon_1.
\end{equation}
%
Also, in the first stage of Multi-Task ERM principle, we select the hypothesis class $\equivclass{\hypf^\frelset}$ that minimizes $\inf_{\myvec{\hypf} \in \equivclass{\hypf}} \risk_{\bprobseq}(\myvec{\hypf})$ where $\myvec{\hypf}$ is a sequence of hypothesis of $\hypspace_\frelset$.
According to Theorem~\ref{th:ben-david_th2}, for $\equivclass{\hypf^\frelset}$ we have that
\begin{equation}
    \nonumber
    {\inf_{\hypf' \in \equivclass{\hypf^\frelset}}\risk_{P_1}(\hypf')} \leq \inf_{\hypf_1, \ldots, \hypf_\ntasks \in \equivclass{\hypf^\frelset}} \frac{1}{\ntasks} \sum_{r=1}^\ntasks \hat{\risk}_{\sample_r}(\hypf_r)  + \epsilon_1.
\end{equation}
Using these two inequalities we get
\begin{equation}
    \nonumber
    \inf_{\hypf' \in \equivclass{\hypf^\frelset}}\risk_{P_1}(\hypf') \leq \inf_{\hypf' \in \equivclass{\hypf^*}}\risk_{P_1}(\hypf') + 2\epsilon_1 
\end{equation}
under the condition~\eqref{eq:ben-david_sample_inequality}. This bounds the risk of the hypothesis space given by the equivalence class of $\hypf^\frelset$ and establishes the inequality~\eqref{eq:ben-david_th3_ineq2}.
%

Once we select $\equivclass{\hypf^\frelset}$, the second stage is just ERM using this hypothesis space.
%
According to the \comm{reference?}
\begin{equation}\nonumber
    \inf_{\hypf \in \hypspace} \risk_{z_1}(\hypf) - \inf_{\hypf \in \hypspace} \risk_{P_1}(\hypf) \leq \epsilon_2
\end{equation}
if
\begin{equation}
    \nonumber
    \cardinal{\sample_1} \geq  \frac{64}{\epsilon^2} \left[2 \vcdim{\hypspace} \log \frac{12}{\epsilon} + \frac{1}{\ntasks}\log\frac{8}{\eta} \right] .
\end{equation}
Since the ERM will not use the whole space $\hypspace$ but the subset $\equivclass{\hypf^\frelset} \subset \hypspace$, and
$$\vcdim{\equivclass{\hypf^\frelset}} \leq \max_{\equivclass{\hypf} \in \hypspace_\frelset} \vcdim{\equivclass{\hypf}} = \underline{d}(\hypspace_\frelset).$$
 then we can write the inequality~\eqref{eq:ben-david_th3_ineq1} of the theorem.
%
The advantage of using multiple tasks is then illustrated in this bound and it will be defined by the gap between $\vcdim{\hypspace}$ and $\underline{d}(\hypspace_\frelset)$. If $\underline{d}(\hypspace_\frelset)$ is smaller than $\vcdim{\hypspace}$, the number of samples needed to solve the target task will also be smaller.
Also, the sample complexity of the rest of tasks is given by $d_{\hypspace_\frelset}(\ntasks)$.

That is, Multi-Task Learning allows to select a subset of hypothesis from which a learner can use the ERM principle. In this stage, the sample complexity is controlled by the generalized VC-dimension of the set of equivalent classes of hypothesis. Once the best equivalent class has been selected, 
the VC-dimension of this subset, compared to the VC-dimension of the whole set of hypothesis, is what marks the difference between Single Task and Multi-Task Learning.

% Analysis of generalized VCdim
\subsubsection*{Analysis of generalized VC-dimension with $\frelset$-related tasks}
As we have seen in Theorem~\ref{th:ben-david_th3}, the VC-dimensions $\vcdim{\hypspace}, \underline{d}(\hypspace_\frelset)$ and $d_{\hypspace_\frelset}(\ntasks)$ are crucial for stating the advantage of Multi-Task over Single Task Learning. 
To understand better how these concepts interact, Ben-David et al. give some theoretical results.
Recall that, given a hypothesis space $\hypspace$, $\hypspace_\frelset$ is a family of hypothesis spaces composed by the hypothesis spaces $\equivclass{\hypf}, \hypf \in \hypspace$, then
\begin{align*}
    \nonumber
    d_{\hypspace_\frelset}(\ntasks) &= \max_{\set{\npertask ,\Pi_{\hypspace_\frelset} = 2^{\ntasks \npertask}} } \npertask, \\
    \underline{d}(\hypspace_\frelset) &= max_{\hypf \in \hypspace} \vcdim{\equivclass{\hypf}}, \\    
    \overline{d}(\hypspace_\frelset) &= \vcdim{\bigcup_{\equivclass{\hypf} \in \hypspace_\frelset} \equivclass{\hypf}}= \vcdim{\hypspace}.
\end{align*}
Using the result from~\eqref{eq:genvc_inequalities} we observe that
\begin{equation}
    \nonumber
    \underline{d}(\hypspace_\frelset) \leq d_{\hypspace_\frelset}(\ntasks) \leq \vcdim{\hypspace} .
\end{equation}
That is, the best we can hope when bounding the sample complexity in Theorem~\ref{th:ben-david_th3} is $\underline{d}(\hypspace_\frelset) = d_{\hypspace_\frelset}(\ntasks)$. 
Ben-David et al. give evidence that, with some restrictions on $\hypspace$, this lower bound can be achieved~\cite[Theorem~4]{Ben-DavidB08}.
\begin{theorem}
    If the support of $\hypf$ is bounded, i.e. $ \cardinal{\set{x \in \Xspace,  \hypf(x) = 1}} < M$, for all $\hypf \in \hypspace$, then there exists $\ntasks_0$ such that for all $\ntasks>\ntasks_0$
    \begin{equation}
        \nonumber
        d_{\hypspace_\frelset}(\ntasks) = \underline{d}(\hypspace_\frelset).
    \end{equation} 
\end{theorem}
Thus, a sufficient condition on the hypothesis space $\hypspace$ to achieve the lowest $d_{\hypspace_\frelset}(\ntasks)$ is a bounded support of any hypothesis. Although this condition may be too restricting, it can also be proved that the upper limit of $d_{\hypspace_\frelset}(\ntasks)$, that is, $\vcdim{\hypspace}$, under some conditions on $\frelset$ is not achieved. 

The following result~\cite[Theorem~6]{Ben-DavidB08} shows this.
\begin{theorem}\label{th:ben-david_th6}
    If $\frelset$ is finite and $\frac{\ntasks}{\log (\ntasks)} \geq \vcdim{\hypspace}$, then
    \begin{equation}
        \nonumber
        d_{\hypspace_\frelset}(\ntasks) \leq 2 \log(\cardinal{\frelset})
    \end{equation}
\end{theorem}
This inequality indicates that, given a finite set of transformation $\frelset$, there are scenarios when $\vcdim{\hypspace}$ is arbitrarily large but $d_{\hypspace_\frelset}(\ntasks)$ is bounded, and therefore, the right-hand side of inequality~\eqref{eq:ben-david_th3_ineq2} is also bounded. That is, the Multi-Task bound, which substitutes $\vcdim{\hypspace}$ by $d_{\hypspace_\frelset}(\ntasks)$ is a better one in this cases. 

\subsubsection*{Conclusion}
\comm{TODO?}


\subsection{Other bounds for Multi-Task Learning}

The work of Baxter~\cite{baxter2000model} set the foundations for the theoretical analysis of MTL. 
In this work, an MTL extension to the VC-dimension is given, and it is used to develop some results bounding the difference between the Multi-Task empirical and expected risks 
\begin{equation}
    \nonumber
    \abs{\risk_{\bprobseq}(\myvec{\hypf}) - \bemprisk(\myvec{\hypf})}
\end{equation}
for any sequence of hypothesis $\myvec{\hypf} \in \hypspace^\ntasks$, see Theorem~\ref{th:baxter_vcdim}. This is a necessary condition for the consistency of Multi-Task Learners, but not a sufficient one.
Then, Ben-David et al.~\cite{Ben-DavidS03,Ben-DavidB08} defines a notion of task relatedness, see Definition~\ref{def:frel_tasks}. Using this notion and building an appropiate hypothesis space $\hypspace = \equivclass{h}$, they bound the \emph{excess risk} of an Empirical Multi-Task Learner, that is, the difference between the best empirical risk, achieved by such Learner, and the best possible expected risk (Theorem~\ref{th:ben-david_th3})
\begin{equation}
    \nonumber
    \abs{\inf_{\myvec{\hypf} \in \hypspace^\ntasks} \risk_{\bprobseq}(\myvec{\hypf}) - \inf_{\myvec{\hypf} \in \hypspace^\ntasks} \bemprisk(\myvec{\hypf})} .
\end{equation}
Moreover, using this task relatedness definition not only the Multi-Task average excess risk is bounded, but the individual excess risk of each task (Theorem~\ref{th:ben-david_th6}).

%
The works discussed until this point on the VC-dimension, and the corresponding extensions to the MTL framework expressed in Definition~\ref{def:gen_vcdim}, to bound the differences between empirical and expected risks.
%
However in~\cite{AndoZ05}, Ando and et al. rely on another notion  of complexity, the Rademacher Complexity~\cite{BartlettM02}, which measures how well a family of hypothesis can approximate random noise. The Rademacher complexity, unlike the VC-dimension, is a distribution-dependent. That is the VC-dimension only uses properties of the hypothesis space $\hypspace$, while the Rademacher complexity also depends on $\distf$. In first place, they use two pseudo-metrics different to those of Definitions~\ref{def:sample_pseudometric} or~\ref{def:dist_pseudometric} and they define the corresponding covering numbers. Then they bound the covering numbers of the set of hypothesis used in~\eqref{eq:mtl_struct_learn} using the Rademacher complexity of those sets of functions. 
However, this bounds are also dependent on the dimensions $k$ and $\dimx$, which make them unfeasible for kernel extensions.





To control the complexity or VC-dimension of the hypothesis space regularization is used, so bounds for regularized methods are also of interest.

In~\cite{Maurer06}, Maurer improves the bounds from~\cite{baxter2000model} and~\cite{Ben-DavidS03,Ben-DavidB08} in the particular case of using a feature extractor $g$, i.e. $\hypf_r = f_r \circ g$.
Then, in~\cite{ArgyriouMP09} an extension of the spectral theorem to vector-valued function is shown. This is useful to compute the\comm{TODO}



In~\cite{PontilM13} bounds for trace-norm regularized methods are given.

Then, in~\cite{MaurerPR16} bounds for the Multi-Task Representation Learning (MTRL)~\cite{ArgyriouEP06} are given. That is, methods where the group sparse regularization $\norm{\mymat{W}^\intercal}_{2, 1}$ is used. Moreover, this bounds are not limited to MTL but are also valid for the Learning to Learn paradigm.


\subsection{Learning Under Privileged Information}
% The classical learning paradigm tries to minimize the expected risk by minimizing the empirical risk...
The standard machine learning paradigm tries to find the hypothesis $\hypf$ from a set of hypothesis $\hypspace$ that minimizes the expected risk $\emprisk$ given a set of training samples.
% Vapnik gives develops the theory of Statisical Learning Theory and it seems complete
Vapnik~\cite{vapnik2013nature} has developed the theory of statistical learning. In this theory several important results are provided: necessary and sufficient conditions for the consistency of learning processes and bounds for the rate of convergence, which uses the notion of VC-dimension. A new inductive principle, Structural Risk Minimization (SRM), and an algorithm, Support Vector Machine (SVM), that makes use of this notion to improve the learning process.

% However, machines need many examples to learn. What is lacking?
% An Intelligent Teacher is important in human learning, providing additional information: examples, metaphors...
Nowadays learning approaches based on Deep Neural Networks, which are not focused on controlling the capacity of the set of hypothesis, outperform the SVM approaches in many problems. However, these popular Deep Learning approaches require large amounts of data to learn good hypothesis.
It is commonly believed that machines need much more samples to learn than humans do. Vapnik~\cite{VapnikV09, VapnikI15a} reflects on this belief and states that humans typically learn under the supervision of an Intelligent Teacher.
This Teacher shares important knowledge by providing metaphors, examples or clarifications that are helpful for the students.



\subsubsection*{LUPI Paradigm}
% How can that additional information be incorporated in Machine Learning?
These additional knowledge provided by the Teacher is the Privileged Information that is available only during the training stage.
To incorporate the concept of Intelligent Teacher in the Machine Learning framework, Vapnik introduces the paradigm of Learning Under Privileged Information (LUPI).
% LUPI
% Definition and notation
In the LUPI paradigm describes the following model. Given a set of i.i.d. triplets
$$ z = \set{(x_1, x^*_1, y_1), \ldots, (x_\nsamples,x^*_\nsamples, y_\nsamples)}, \; x \in \Xspace, x^* \in \Xspace^*, y \in \Yspace $$
generated according to an unknown distribution $P(x, x^*, y)$, the goal is to find the hypothesis $\hyp{x}{\param^*}$ from a set of hypothesis $\hypspace = \set{\hyp{x}{\param}, \param \in \paramspace}$ that minimizes some expected risk 
$$ \exprisk = \int \loss{\hyp{x}{\param}}{y} dF(x, y). $$

Note that the goal is the same that in the standard paradigm, however with the LUPI approach we are provided additional information, which is available only during the training stage. This additional information is encoded in the elements $x^*$ of a space $\Xspace^*$, which is different from $\Xspace$. The goal of the Teacher is, given a pair $(x_i, y_i)$, to provide a useful information $x^* \in \Xspace^*$ given some probability $\cond{x^*}{x}$. That is, the ''intelligence'' of the Teacher is defined by the choice of the space $\Xspace^*$ and the conditional probability $\cond{x^*}{x}$. 
% Examples
To understand better this paradigm consider the following examples.
\\
% Example of doctor: images and commentaries
\textbf{Example 1.} Consider that the goal is to find a decision rule that classifies biopsy images into cancer or non-cancer. Here, $\Xspace$ is the space of images, i.e. the matrix of pixels, for example $[0, 1]^{64 \times 64}$. The label space is $\Yspace = \set{0, 1}$. An Intelligent Teacher might provide a student of medicine with commentaries about the images, for example: ''There is an area of unusual concentration of cells of Type A.'' or ''There is an aggresive proliferation of cells of Type B''. These commentaries are the elements $x^*$ of certain space $X^*$ and the Teacher also chooses the probability $\cond{x^*}{x}$.
\\
% Example of high quality and low quality images
\textbf{Example 2.} Consider the \comm{TODO} 

\subsubsection*{Analysis of convergence rates}
To get a better insight of how the Privileged Information can help in the Learning process, Vapnik provides a theoretical analysis of its influence on the learning rates.
In the standard learning paradigm, how well the expected risk $\exprisk$ can be bounded is controlled by two factors: the empirical risk $\emprisk$ and the VC-dimension of the set of hypothesis $\mathcal{\hypspace}$.
In the case of classification, where $\Yspace = \set{-1, 1}$ and the loss $\lossf(\hyp{x}{\param}, y) = \ind_{\hyp{x}{\param} y \leq 0}$, the risks can be expressed as
\begin{align*}
    \nonumber
    \exprisk (\param) &= \int \ind_{y \hyp{x}{\param} \leq 0} d\distf(x, y) =  P(\hyp{x}{\param} y \leq 0) ,\\
    \emprisk(\param) &= \sum_{i=1}^\nsamples \ind_{y_i \hyp{x_i}{\param} \leq 0} = \nu(\param) .
\end{align*}
In~\cite[Theorem~6.8]{vapnik1982estimation} the following bound for the rate of convergence is given with probability $1 - \eta$:
\begin{equation}
    \nonumber
    P(\hyp{x}{\param_\nsamples} y \leq 0) \leq \nu(\param_\nsamples) + \bigO{\frac{\vc \log\left(\frac{2\nsamples}{\vc} \right) - \log\eta}{\nsamples} \sqrt{\nu(\param_\nsamples) \frac{\nsamples}{\vc \log\left(\frac{2\nsamples}{\vc} \right) - \log\eta}} } .
\end{equation}
That is, the bound is controlled by the ratio $\vc/\nsamples$, where $\vc$ is the $\vcdim{\hypspace}$. If this $VC$-dimension is finite, the bound goes to zero as $\nsamples$ grows.
However, two different cases can be considered.
\\
% Convergence of separable hypothesis

\textbf{Separable case:} the training data can be classified in two groups without errors. That is, there exists $\param_\nsamples \in \eta$ such that $y_i \hyp{x_i}{\param_\nsamples} > 0$ for $i = 1, \ldots, \nsamples$, and thus $\nu(\param_\nsamples) = 0$.
In this case, the following bound for the rate of converge holds
\begin{equation}
    \nonumber
    P(\hyp{x}{\param_\nsamples} y \leq 0) \leq \bigO{\frac{\vc \log\left(\frac{2\nsamples}{\vc} \right) - \log\eta}{\nsamples} } .
\end{equation}
\\
% Convergence of non-separable
\textbf{Non-Separable case:} the training data cannot be classified in two groups without errors. That is, for all $\param_\nsamples \in \paramspace$, there exists $i = 1, \ldots, \nsamples$, such that $y_i \hyp{x_i}{\param_\nsamples} \leq 0$, and thus $\nu(\param_\nsamples) > 0$.
In this case, the following bound for the rate of converge holds
\begin{equation}
    \nonumber
    P(\hyp{x}{\param_\nsamples} y \leq 0) \leq \nu(\param_\nsamples) + \bigO{\sqrt{\frac{\vc \log\left(\frac{2\nsamples}{\vc} \right) - \log\eta}{\nsamples}} } .
\end{equation}
Note that there is an important difference here in the rate of convergence. The separable case has a convergence rate of $\vc/\nsamples$, while the non-separable case has a rate of $\sqrt{\vc/\nsamples}$. Vapnik tries to address the question of why there exists such difference.

% Oracle SVM: separable
\subsubsection*{Oracle SVM}
Vapnik tries to answer these question by looking at Support Vector Machines. In the separable case, one has to minimize the functional 
$$J(\hplane) = \norm{\hplane}^2$$
subject to the constraints
$$y_i \left( \hplane x_i + \bias \right) \geq 1.$$
However, in the non-separable case the functional to minimize is 
$$J(\hplane, \xi_1, \ldots, \xi_\nsamples) = \norm{\hplane}^2 + C \sum_{i=1}^\nsamples \xi_i$$
subject to the constraints
$$y_i \left( \hplane x_i + b \right) \geq 1 - \xi_i.$$
That is, in the separable case $\dimx$ parameters (of $\hplane$) have to be estimated using $\nsamples$ examples, while in the non-separable case $\dimx + \nsamples$ parameters (considering $\hplane$ and the slack variables $\xi_1, \ldots, \xi_\nsamples$) have to be estimated with $\nsamples$ examples. 

What would happen if the parameters $\xi_1, \ldots, \xi_\nsamples$ were known?
In~\cite{VapnikI15a} an \emph{Oracle} SVM is considered. Here, the learner (Student) is supplied with a set of triplets
\begin{equation}
    \nonumber
    (x_1, \xi_1^0, y_1), \ldots, (x_\nsamples, \xi_\nsamples^0, y_\nsamples)
\end{equation}
where $\xi^0_1, \ldots, \xi^0_\nsamples$ are the slack variables for the best decision rule $\hyp{x}{\param_0} = \inf_{\param \in \paramspace} \exprisk(\param) $:
\begin{equation}
    \nonumber
    \xi_i^0 = \max \left(0, 1 - \hyp{x}{\param_0}  \right), \; \forall i = 1, \ldots, \nsamples .
\end{equation}
An \emph{Oracle} SVM has to minimize the functional
$$J(\hplane) = \norm{\hplane}^2$$
subject to the constraints
$$y_i \left( \hplane x_i + \bias \right) \geq 1 - \xi_i^0 .$$
Since the slack variables $\xi_i^0$ are known in advance, it can be shown~\cite{VapnikV09} that for the \emph{Oracle} SVM the following bound holds
\begin{equation}
    \nonumber
    P(\hyp{x}{\param_\nsamples} y \leq 0) \leq P(1 - \xi^0 \leq 0) + \bigO{\frac{\vc \log\left(\frac{2\nsamples}{\vc} \right) - \log\eta}{\nsamples} } ,
\end{equation}
where $P(1 - \xi^0 \leq 0)$ is the probability error of the hypothesis $\hyp{x}{\param_0}$.
That is, we recover the rate $\vc/\nsamples$.

% Intelligent Teacher: separable but we need to control two models
\subsubsection*{From Oracle to Intelligent Teacher}
The \emph{Oracle} SVM is a theoretical construct, but we can approximate it by modelling the slack variables with the information provided by the Teacher in the LUPI paradigm.
That is, the Teacher defines a space $\Xspace^*$ and a set of functions $\set{f^*(x, \param^*) , \param^* \in \paramspace^*}$. Then model the slack variables can be approximated as
\begin{equation}
    \nonumber
    \xi^* = f^*(x^*, \param^*) .
\end{equation}
From the pairs generated by some random generator in nature, the Teacher also defines the probability $\cond{x^*}{x}$ to provide the triplets
\begin{equation}
    \nonumber
    (x_1, x^*_1, y_1), \ldots, (x_\nsamples, x^*_\nsamples, y_\nsamples) .
\end{equation}
Then, we can consider the problem where the goal is to minimize
$$ J(\param, \param^*) = \sum_{i=1}^\nsamples \max(0, f^*(x_i^*, \param^*)) $$
subject to the constraints
$$ \hyp{x_i}{\param} \geq 1 - f^*(x_i^*, \param^*).$$
Let $f(x, \param_\nsamples), h(x. \param_\nsamples)$ that minimize this problem. Then, in~\cite[Proposition~2]{VapnikV09} the following results for the bound of convergence is given
\begin{equation}
    \nonumber
    P(\hyp{x}{\param_\nsamples} y \leq 0) \leq P(1 - f^*(x^*, \param^*_\nsamples) \leq 0) + \bigO{\frac{(\vc + \vc^*) \log\left(\frac{2\nsamples}{(\vc + \vc^*)} \right) - \log\eta}{\nsamples} } ,
\end{equation}
where $d^*$ is the VC-dimension of the space of hypothesis $\set{f(x, \param^*) \in \paramspace^*}$. This result shows that, to maintain the best convergence rate $\vc/\nsamples$, we need to estimate the $P(1 - f^*(x^*, \param^*_\nsamples) \leq 0)$. Although this probability is unknown,  we can control it. Considering
\begin{equation}
    \nonumber
    \param^*_0 = \inf_{\param^* \in \paramspace^*} \int_{\Xspace^*} \max(0, f^*(x^*, \param^*)- 1) dP(x^*)
\end{equation}
and
\begin{equation}
    \nonumber
    \param^*_\nsamples = \inf_{\param^* \in \paramspace^*} \sum_{i=1}^\nsamples \max(0, f^*(x^*_i, \param^*)- 1) .
\end{equation}
Consider $\set{f^*(x^*, \param^*), \param^* \in \paramspace^*}$ such that $f^*(x^*, \alpha^*) < B, \param^* \in \paramspace^*$, then 
$$\set{\max(0, f^*(x^*, \param^*)- 1), \param^* \in \paramspace^*}$$
 is a set of totally bounded non-negative functions, then we have the standard bound~\cite{vapnik2013nature}
\begin{equation}
    \nonumber
    P(1 - f^*(x^*, \param^*_0) \leq 0) \leq P(1 - f^*(x^*, \param^*_\nsamples) \leq 0) + \bigO{\sqrt{\frac{\vc \log\left(\frac{2\nsamples}{\vc} \right) - \log\eta}{\nsamples}} } .
\end{equation}
with probability $1 - 2\eta$.
That is, to have a rate of $\vc/\nsamples$ for $\param_\nsamples$, we need to estimate $\param^*_\nsamples$, which has a rate of $\sqrt{\vc^*/\nsamples}$. However, observe that $\Xspace^*$ is the space suggested by the Teacher, which hopefully has a much lower capacity, and thus, the convergence will be faster in this space.

\subsubsection*{SVM+ }
Vapnik describes an extension of the SVM that embodies the LUPI paradigm~\cite{VapnikV09,VapnikI15a}. Given a set of triplets
\begin{equation}
    \nonumber
    (x_1, x^*_1, y_1), \ldots, (x_\nsamples, x^*_\nsamples, y_\nsamples) ,
\end{equation}
the idea is to model the slack variables of the standard SVM using the elements $x^* \in \Xspace^*$ as
$$ \xi(x^*, y) = \left[y (w^* \phi^*(x^*) + b^*) \right]_+  = \max\left( y (w^* \phi^*(x^*) + b^*), 0  \right).$$
The minimization problem is the following:
\begin{equation}
    \label{eq:svmplus_original}
    \begin{aligned}
        & \argmin_{w, w^*, b, b^*}
        & &  C \sum_{i=1}^\nsamples \left[y_i (\dotp{w^*}{\phi^*(x_i^*)} + b^*) \right]_+ + \frac{1}{2} \dotp{w}{w} + \frac{\mu}{2} \dotp{w^*}{w^*} \\
        & \text{s.t.}
        & & y_{i} ( \dotp{w}{\phi(x_{i})} + b) \geq 1 - \left[y_i (\dotp{w^*}{\phi^*(x^*_i)} + b^*) \right]_+ .
    \end{aligned}
\end{equation}
Here $\phi$ and $\phi^*$ are two transformations that can be different.
However, note that problem~\eqref{eq:svmplus_original} is not convex due to the positive part term in the objective function. Vapnik et al. propose a relaxation of this problem to obtain a convex one. The idea is to model the slack variables $\xi$ as
$$ \xi(x^*, y) = \left[y (w^* \phi^*(x^*) + b^*) \right] + \zeta(x^*, y) ,$$
where $\zeta(x^*, y) \geq 0$.
The minimization problem is then
\begin{equation}
    \label{eq:svmplus_delta_primal}
    \begin{aligned}
        & \argmin_{w, w^*, b, b^*, \zeta_i}
        & &  C \sum_{i=1}^\nsamples \left( \left[y_i (\dotp{w^*}{\phi^*(x_i^*)} + b^*) \right] + \zeta_i \right) + C \Delta \sum_{i=1}^\nsamples \zeta_i \\
        & & &\qquad + \frac{1}{2} \dotp{w}{w} + \frac{\mu}{2} \dotp{w^*}{w^*} \\
        & \text{s.t.}
        & & y_{i} ( \dotp{w}{\phi(x_{i})} + b) \geq 1 - \left[y_i (\dotp{w^*}{\phi^*(x^*_i)} + b^*) + \zeta_i \right] ,\\
        & & &y_i (\dotp{w^*}{\phi^*(x^*_i)} + b^*) + \zeta_i \geq 0 ,\\
        & & &\zeta_i \geq 0, \\
        & \text{for } & & i=1, \ldots, \nsamples.
    \end{aligned}
\end{equation}
Problem~\eqref{eq:svmplus_delta_primal} is convex and the corresponding dual problem is
\begin{equation}\label{eq:svmplus_delta_dual}
    \begin{aligned}
        & \argmin_{\alpha_i, \delta_i} 
        & & \frac{1}{2} \sum_{i, j=1}^\nsamples y_i y_j \alpha_i \alpha_j k(x_i, x_j) +\frac{1}{2 \mu} \sum_{i, j=1}^\nsamples y_i y_j (\alpha_i - \delta_i) (\alpha_j - \delta_i) k^*(x^*_i, x^*_j)  - \sum_{i=1}^\nsamples \alpha_i \\
        & \text{s.t.}
        & & 0 \leq \delta_i \leq C \\
        & & & 0 \leq \alpha_i \leq C + \delta_i, \\
        & & & \sum_{i=1}^{\nsamples}{\delta_i y_i} = 0, \; 
        \sum_{i=1}^{\nsamples}{\alpha_i y_i} = 0, \\
        & \text{for } & & i=1, \ldots, \nsamples.
        \end{aligned}
\end{equation}
where we use the kernel functions
$$k(x_i, x_j) = \dotp{\phi(x_i)}{\phi(x_j)}, \; k^*(x^*_i, x^*_j) = \dotp{\phi^*(x^*_i)}{\phi^*(x^*_j)}.$$ 
We can observe in Problem~\eqref{eq:svmplus_delta_dual} that the LUPI paradigm exerts a similarity control, correcting the similarity in space $\Xspace$ with the similarity in the privileged space $\Xspace^*$. For that reason, $\Xspace$ and $\Xspace^*$ are named Decision Space and Correction Space, respectively.


\subsubsection*{Connection between SVM+ and MTLSVM}
% Liang and Cherkassky
In~\cite{LiangC08} the connection between SVM+ and Multi-Task Learning SVM (MTLSVM) is discussed. 
% Definition
The MTLSVM proposed in~\cite{LiangC08} is a Multi-Task Learning model based on the SVM. It solves the primal problem
\begin{equation}
    \label{eq:mtlsvm_primal}
    \begin{aligned}
        & \argmin_{w, b, v_r, b_r, \xi_i^r}
        & & C \sum_{r=1}^\ntasks \sum_{i=1}^{\npertask_r} \xi_i^r + \frac{1}{2} \dotp{w}{w} + \sum_{r=1}^\ntasks \frac{\mu}{2} \dotp{v_r}{v_r} \\
        & \text{s.t.}
        & & y_{i}^r ( \dotp{w}{\phi(x_{i}^r)} + b + \dotp{v_r}{\phi_r(x_{i}^r)} + b_r) \geq 1 - \xi_i^r ,\\
        & & &\xi_i^r \geq 0, \\
        & \text{for } & & r=1, \ldots, \ntasks; \; i=1, \ldots, \npertask_r.
    \end{aligned}
\end{equation}
Here, a combination of a common model for all tasks
$$ \dotp{w}{\phi(x_{i})} + b $$
and a task-specific model
$$ \dotp{v_r}{\phi_r(x_{i}^r)} + b_r $$
is used. Here, the common transformation $\phi$ and the task-independent ones $\phi_r$ can be different 
The dual problem corresponding to~\eqref{eq:mtlsvm_primal} is
\begin{equation}\label{eq:mtlsvm_dual}
    \begin{aligned}
        & \argmin_{\alpha_i} 
        & & \frac{1}{2} \sum_{r, s=1}^\ntasks \sum_{i, j=1}^{\npertask_r} y_i^r y_j^s \alpha_i^r \alpha_j^s k(x_i^r, x_j^s) + \frac{1}{2 \mu} \sum_{r, s=1}^\ntasks  \sum_{i, j=1}^{\npertask_r} y_i^r y_j^s \alpha_i^r \alpha_j^s \delta_{rs} k_r(x^r_i, x^s_j) \\
        & & & \qquad - \sum_{r=1}^\ntasks \sum_{i=1}^{\npertask_r} \alpha_i^r \\
        & \text{s.t.}
        & & 0 \leq \alpha_i^r \leq C \\
        & & & \sum_{i=1}^{\npertask_r}{\alpha_i^r y_i^r} = 0, \\
        & \text{for } & & r=1, \ldots, \ntasks; \; i=1, \ldots, \npertask_r.
        \end{aligned}
\end{equation}
% Similarities and differences
In~\cite{LiangC08} some similarities between MTLSVM and SVM+ are pointed out. Problem~\eqref{eq:mtlsvm_primal} can be regarded as an adaptation of~\eqref{eq:svmplus_delta_primal} to solve MTL problems, where the $\Xspace = \Xspace^*$ but different tasks are incorporated, which is reflected on the different transformations $\phi_r$.
If we consider the problem~\eqref{eq:mtlsvm_primal} with a single task, it is a modification of SVM+ where the slack variables are modeled as
$$ \xi(x^*, y) = y (w^* \phi^*(x^*) + b^*)  $$
and $x^* = x$. That is, it is a relaxation of the original problem~\eqref{eq:svmplus_original} where the positive part is ignored.
This relaxation gives place to some important differences between both models. Since the auxiliary primal variables $\zeta_i$ are no longer required, this is reflected in a simpler dual form~\eqref{eq:mtlsvm_dual}, where only $\nsamples$ dual variables have to be estimated, instead of the $2\nsamples$ dual variables of~\eqref{eq:svmplus_delta_dual}.
The Multi-Task element, is reflected on~\eqref{eq:mtlsvm_dual} through the $\delta_{rs}$ function, which makes the correction of similarity only possible between elements of the same task.
%
A major remark can be make about the differences between MTLSVM and SVM+. The results for the improved rate of convergence with an Intelligent Teacher may not be valid with MTLSVM, since we are no modelling the slack variables $\xi$ adequately. 
It is still a work in progress to study the rate of convergence of MTLSVM and to establish more clear links with SVM+.
% Relation with the simplified approach?


%%%%%%%%%%%%%%%%%%%%%%%%%%%%%%%%%%%%%%%%%%%%%%%%%%%%%%%%%%%%%%%%%%%%%%%%%%%%%%%%%%%%%
  %%%%%%%%%%%%%%%%%%%%%%%%%%             SECTION         %%%%%%%%%%%%%%%%%%%%%%%%%%
%%%%%%%%%%%%%%%%%%%%%%%%%%%%%%%%%%%%%%%%%%%%%%%%%%%%%%%%%%%%%%%%%%%%%%%%%%%%%%%%%%%%%

\section{Multi-Task Learning with Kernel Methods} % Evgeniou
% Most multi-task methods are linear models, which may not be flexible enough to capture certain dependencies.
% Deep Learning is a very popular and cost-effective way of overcoming this problem. The final linear models are substituted by the neural network output and the parameters are learned together using back propagation.
% However, one of the main problems of deep learning is the lack of theoretical results and the non-convexity of the problems.
% Other alternative to extend the MTL models non-linearly is by using the kernels.
% Kernel trick.

% Kernels for Multi--task Learning. Michelli and Pontil

% Joint Learning
% Leveraging common and specific information
% Common + specific model which is equivalent to penalizing individual norm and variance
% Evgeniou, T. and Pontil, M. (2004). Regularized multi-task learning.
% Evgeniou, T., Micchelli, C. A., and Pontil, M. (2005).  Learning 

% Teorema de Evgeniou

% When is there a representer theorem? Vector versus matrix regularizers.

% Multi-task least-squares support vector machines. Shuo

% Multi-task Gaussian process prediction. Bonilla

% Sparse coding for multitask and transfer learning

\section{Conclusions}\label{sec-conclusions-2}

In this chapter, we covered\dots
