% Chapter 3

\chapter{A Convex Formulation for Multi-Task Learning} % Write in your own chapter title
\label{Chapter4}
\lhead{Chapter \ref{Chapter4}. 
\emph{A Convex Formulation for Multi-Task Learning}} % Write in your own chapter title to set the page header

{\bf \small{

}}

\section{Introduction}
% Multi-Task Learning paradigms can be divided in: Feature-Based, Parameter-Based, Joint-Learning



% The Feature-Based approach is restricted to use the same representation for all tasks

% The Parameter-Based approach usually involves a lot of hyperparameters and problems that are not convex

% The Joint-Learning approach is a very natural way to combine models, combining only the final prediction
% flexible
% easy to express: one single hyperparameter
% preserver convexity

% General formulation of joint learning

% Convex formulation is more natural

% General formulation of convex joint learning

% hyperparameter lambda


\section{Convex Multi-Task Learning with Kernel Models}
% Kernel Models are powerful: Kernel trick, high dimensionality representation
As explained in the previous chapters, kernel models offer many good properties such as implicit transformation to a possibly infinite-dimensional space and convexity. In thise models a regularized risk problem is solved.
% General formulation for a kernel model
A general formulation for kernel models is 
\begin{equation}
    \label{eq:regrisk_kernel}
    \emprisk({w}) \defeq \sum_{i=1}^{\nsamples} \lossf({w}^\intercal {\phi(\fv{x}_i)}+ b, y_i) + \mu \Omega({w}) ,
\end{equation}
where $\sample$ is the sample $\{(x_1, y_1), \ldots, (x_\nsamples, y_\nsamples)\}$ and $\Omega(w)$ is a regularizer for ${w}$, tipically the $L_2$ norm: $\norm{w}^2$. Observe that $b$, the bias term, is not regularized since it does not affect the capacity of the hypothesis space.
In~\eqref{eq:regrisk_kernel} $\phi$ is a fixed transformation function such that there exists a ``kernel trick'', that is a kernel function $k$ for which
\begin{equation}
    \nonumber
    \dotp{\phi(x)}{\phi(y)} = k(x, y).
\end{equation}
These models embrace the Structural Risk Minimization paradigm by limiting the capacity of the space of hypothesis. This is done by using the $L_2$ norm of $w$ as regularizer, which is equivalent to limiting our space of candidates to vectors inside a ball of some radius.

% One of the first multi-task learning approach with kernel models is Regularized MTL
Multi-Task Learning with kernel models require imposing some kind of coupling between models in the learning process. The feature learning or feature sharing approach, which is usually adopted with neural networks, is not feasible when using kernel models, since the (implicit) transformation functions used are not learned but fixed. Therefore, other strategies have to be developed. 
One of the first approaches to MTL with kernel models was developed in~\cite{EvgeniouP04}, where the models for each task are defined as:
\begin{equation}
    \nonumber
    {w}_r = {w} + {v}_r,
\end{equation} 
where $w$ is a common part, shared by all models, and $v_r$ is a task-specific part. With this approximation, the transfer of information is performed by the common part ${w}$.
The regularized risk that is minimized is 
\begin{equation}
    \label{eq:additive_regrisk}
    \risk_{\bsample} ({w}) \defeq \sum_{r=1}^{\ntasks} \sum_{i=1}^{\npertask_r} \lossf({w}^\intercal {\phi(\fv{x}_i)} + {v}_r^\intercal {\phi_r(\fv{x}_i)}, y_i) + \mu_c \norm{{w}}^2 + \mu_s \sum_{r=1}^\ntasks \norm{{v}_r}^2,
\end{equation}
where $\mu_c$ and $\mu_s$ are the hyperparameters to control the common and specific parts regularization, respectively. Here $\bsample$ is a MTL sample $$\bsample = \bigcup_{r=1}^\ntasks \{(x_1^r, y_1^r), \ldots, (x_{\npertask}^r, y_{\npertask}^r)\}.$$
Observe also that in~\eqref{eq:additive_regrisk} different transformations are used. The transformation $\phi$ corresponds to common part of the model, while $\phi_r$ is task-specific.
%
This is a joint learning approach that is developed for the L1-SVM, which Evgeniou et al. give the name of \emph{Regularized MTL} but we will refer to as {additive} approach.
%
Observe that as $\frac{\mu_c}{\mu_s} \tendsto{}{\infty} $ we would have a common part ${w}$ that tends to zero, which would results in independent models for each task, i.e. ${w}_r \eqsim {v}_r$. On the contrary, when $\frac{\mu_c}{\mu_s} \tendsto{}{0} $, the task-specific parts tend to zero and every model is the common part, i.e. ${w}_r \eqsim {w}$.
There are two asymptotical behaviours: the first one tends to an ITL approach, while the second one tends to a CTL one. The MTL formulation is one strategy that lies between those two approaches, CTL and ITL, combining them to achieve a more flexible model.

% General formulation for a convex MTL Kernel 
The asymptotical properties of this approach offer an interpretation to understand the influence of each hyperparameter, but they are not applicable in practice.
In~\cite{RuizAD19} an alternative formulation for this joint learning approach is proposed. The models for each task are defined as a convex combination of the common and task specific parts:
\begin{equation}
    \nonumber
    {w}_r = \lambda {w} + (1 - \lambda) {v}_r,
\end{equation}
where $\lambda$ is a hyperparameter such that $\lambda \in \left[0, 1\right]$.
The corresponding regularized risk is 
\begin{equation}
    \nonumber
    \risk_{\bsample} ({w}) \defeq \sum_{r=1}^{\ntasks} \sum_{i=1}^{\npertask_r} \lossf({w}^\intercal {\phi(\fv{x}_i)} + {v}_r^\intercal {\phi_r(\fv{x}_i)}, y_i) + \mu_c \norm{{w}}^2 + \mu_s \sum_{r=1}^\ntasks \norm{{v}_r}^2,
\end{equation}
We will name this approach {convex}, in contrast to the {additive} approach of the original formulation.
% Discussion?
With this formulation, the interpretation of $\lambda$ is straight-forward. The model with $\lambda = 1$ is equivalent to learning a single common model for all tasks, that is ${w}_r = {w}$. When $\lambda=0$, the models for each task are completely independent: ${w}_r = {v}_r$.
The convex formulation also define an MTL model that is between a CTL approach and an ITL one, but it presents two advantages: the values of $\lambda$ that recover the CTL and ITL approaches are known, and these values are specific, not an asymptotical behaviour as in the original formulation. 
%
Moreover, it is shown in the paper that the two formulations, the {additive} and {convex}, are equivalent with an L1-SVM setting.
%
The proposal of~\cite{RuizAD19} is made only for L1-SVM, but it is extended to L2 and LS-SVMs in~\cite{RuizAD21}. In this chapter we present the convex formulation for L1, L2 and LS-SVM, as well as the equivalence results between the {additive} and {convex} formulations.




\subsection{L1 Support Vector Machine}
The L1-SVM~\cite{Vapnik00} is the original and most popular variant of the SVMs and is also the basis of the MTL formulation in~\cite{EvgeniouP04}.
I will present the development for the {additive} approach and the one for the {convex} using an L1-SVM setting. Then I will show the equivalence between the two approaches and discuss its differences.

\subsubsection{{additive} MTL L1-SVM}
% Additive Approach
%   Primal
The {additive} MTL primal problem formulation, presented in~\cite{EvgeniouP04} 
and extended for task-specific biases in~\cite{CaiC12}, is
\begin{equation}\label{eq:svmmtl_primal_add}
    \begin{aligned}
    & \argmin_{w, {v}_r, \xi}
    & & {J({w}, {v}_r, \xi) = C \sum_{r= 1}^T \sum_{i=1}^{m_r} {\xi_{i}^r} + \frac{1}{2} \sum_{r= 1}^T{\norm{{v}_r}^2} + \frac{\mu}{2} {\norm{{w}}}^2} \\
    & \text{s.t.}
    & & y_{i}^r ({w} \cdot \phi(x_{i}^r) + b + {v}_r \cdot \phi_r(x_{i}^r) + d_r) \geq p_{i}^r - \xi_{i}^r ,  \\
    & & & \xi_{i}^r \geq 0; \;  i=1 , \dotsc , m_r, \;  r= 1,\dotsc, T  . \\
    \end{aligned}
\end{equation}
The prediction model is then 
\begin{equation}
    \nonumber
    h_r(\fv{x}) = {w} \cdot \phi(x_{i}^r) + b + {v}_r \cdot \phi_r(x_{i}^r) + d_r
\end{equation}
for regression and 
\begin{equation}
    \nonumber
    h_r(\fv{x}) = \sign{ \left({w} \cdot \phi(x_{i}^r) + b + {v}_r \cdot \phi_r(x_{i}^r) + d_r \right)}
\end{equation}
for classification.
Observe again that the transformation $\phi$ is used for the common part and is shared by all tasks, while the transformation $\phi_r$ is task-specific.

%
In~\eqref{eq:svmmtl_primal_add} there are two kind of hyperparameters: $C$ and $\mu$, which, in combination, balance the different parts of the objective function. 
%
Hyperparameter $C$ plays the same role than in the standard L1-SVM : it balances the tradeoff between the loss incurred by the model, represented by the hinge variables $\xi_i^r$ and complexity of the models, represented by the norms $\norm{{w}}$ and $\norm{{v}_r}$. Large values of $C$ highly penalize the loss, so the resulting models are more complex because they have to adapt to the training sample distribution, but these models generalize worse. Small values of $C$ penalize more the norms of $w$ and $v_r$ so the resulting models are simpler but not so dependent on the training sample.

%
Hyperparameter $\mu$, in combination with $C$, balances the specifity of our models. 
Large values of $\mu$, penalize the common part, resulting in more specific models; while small values of $\mu$, alongside large values of $C$, result in a vanishing regularization of the specific parts which leads to common models.
We can find the following cases:
\begin{itemize}
    \item Reduction to an ITL approach:
    $$\mu \tendsto{}{\infty} \implies h_r(\fv{x}_i^r) = {v}_r \cdot \phi_r(x_{i}^r) + d_r .$$
    That is, the models are learned independently because the common part vanishes.
    \item Reduction to a CTL approach: 
    $$C \tendsto{}{0}, \mu \tendsto{}{0} \implies h_r(\fv{x}_i^r) = {w} \cdot \phi(x_{i}^r) + b .$$
    That is, the model for all tasks is common because the specific parts disappear.
    \item Pure MTL approach:
    $$ \mu_\text{inf} < \mu < \mu_\text{sup} \implies h_r(\fv{x}_i^r) =({w} \cdot \phi(x_{i}^r) + b) + ({v}_r \cdot \phi_r(x_{i}^r) + d_r) .$$
    There is a range of $\mu$, which is unknown, in which the models combine a common and task-specific part.
\end{itemize}
%   Lagrangian/KKT

Observe that~\eqref{eq:svmmtl_primal_add} is a convex problem. As in the standard case of the L1-SVM, the corresponding dual problem is solved. To obtain the dual problem, it is necessary to express the Lagrangian of problem~\eqref{eq:svmmtl_primal_add}:
\begin{equation}\label{eq:svmmtl_lagrangian_add}
    \begin{aligned}
        & \lagr({w}, \fv{v}_1, \ldots, \fv{v}_\ntasks, b, d_1, \ldots, d_\ntasks, \fv{\xi}, \fv{\alpha}, \fv{\beta}) \\
        &= C \sum_{r= 1}^T \sum_{i=1}^{m_r} {\xi_{i}^r} + \frac{1}{2} \sum_{r= 1}^T{\norm{{v}_r}^2} + \frac{\mu}{2} {\norm{{w}}}^2 \\
        &\quad -  \sum_{r= 1}^T \sum_{i=1}^{m_r} \alpha_i^r \left\lbrace y_{i}^r \left[ {w} \cdot \phi(x_{i}^r) + b + {v}_r \cdot \phi_r(x_{i}^r) + d_r \right] - p_{i}^r + \xi_{i}^r  \right\rbrace \\
        &\quad -  \sum_{r= 1}^T \sum_{i=1}^{m_r} \beta_i^r \xi_i^r ,
    \end{aligned}
\end{equation}
where $\alpha_i^r, \beta_i^r \geq 0$ are the Lagrange multipliers. Here $\fv{\xi}$ represents the vector $$(\xi_1^1, \ldots, \xi_{m_1}^1, \ldots, \xi_1^\ntasks, \ldots, \xi_{m_\ntasks}^\ntasks)^\intercal$$ and analogously we define $\fv{\alpha}$ and $\fv{\beta}$.
Recall that the dual objective function is defined as 
\begin{equation}\nonumber
    \begin{aligned}
         \Theta(\fv{\alpha}, \fv{\beta}) &=  \min_{{w}, \fv{v}_1, \ldots, \fv{v}_\ntasks, b, d_1, \ldots, d_\ntasks, \fv{\xi}} \lagr({w}, \fv{v}_1, \ldots, \fv{v}_\ntasks, b, d_1, \ldots, d_\ntasks, \fv{\xi}, \fv{\alpha}, \fv{\beta}) \\
         &= \lagr(\optim{{w}}, \optim{\fv{v}_1}, \ldots, \optim{\fv{v}_\ntasks}, \optim{b}, \optim{d_1}, \ldots, \optim{d_\ntasks}, \optim{\fv{\xi}}, \fv{\alpha}, \fv{\beta})
    \end{aligned}    
\end{equation}
Since $\lagr$ is convex with respect to the primal variables, it is just necessary to compute the corresponding gradients
\begin{align}
    \grad_{{w}} \lagr \vert_{\optim{{w}}, \optim{\fv{v}_1}, \ldots, \optim{\fv{v}_\ntasks}, \optim{b}, \optim{d_1}, \ldots, \optim{d_\ntasks}, \optim{\fv{\xi}}, {\fv{\alpha}}, {\fv{\beta}}} = 0  &\implies \mu \optim{{w}} - \sum_{r= 1}^T \sum_{i=1}^{m_r} {\alpha_i^r} \left\lbrace y_i^r \phi(x_i^r) \right\rbrace = 0 , \label{eq:common_repr_add} \\
    \grad_{{v}_r} \lagr \vert_{\optim{{w}}, \optim{\fv{v}_1}, \ldots, \optim{\fv{v}_\ntasks}, \optim{b}, \optim{d_1}, \ldots, \optim{d_\ntasks}, \optim{\fv{\xi}}, {\fv{\alpha}}, {\fv{\beta}}} = 0 &\implies \optim{{v}_r} - \sum_{i=1}^{m_r} {\alpha_i^r} \left\lbrace y_i^r \phi_r(x_i^r) \right\rbrace = 0 , \label{eq:specific_repr_add} \\
    \grad_{b} \lagr \vert_{\optim{{w}}, \optim{\fv{v}_1}, \ldots, \optim{\fv{v}_\ntasks}, \optim{b}, \optim{d_1}, \ldots, \optim{d_\ntasks}, \optim{\fv{\xi}}, {\fv{\alpha}}, {\fv{\beta}}} = 0  &\implies \sum_{r= 1}^T \sum_{i=1}^{m_r} {\alpha_i^r} y_i^r = 0 , \label{eq:specific_eqconstr_add}  \\
    \grad_{{d}_r} \lagr \vert_{\optim{{w}}, \optim{\fv{v}_1}, \ldots, \optim{\fv{v}_\ntasks}, \optim{b}, \optim{d_1}, \ldots, \optim{d_\ntasks}, \optim{\fv{\xi}}, {\fv{\alpha}}, {\fv{\beta}}} = 0 &\implies \sum_{i=1}^{m_r} {\alpha_i^r} y_i^r = 0 , \label{eq:common_eqconstr_add} \\
    \grad_{\xi_i^r} \lagr \vert_{\optim{{w}}, \optim{\fv{v}_1}, \ldots, \optim{\fv{v}_\ntasks}, \optim{b}, \optim{d_1}, \ldots, \optim{d_\ntasks}, \optim{\fv{\xi}}, {\fv{\alpha}}, {\fv{\beta}}} = 0 &\implies C - {\alpha_i^r} - {\beta_i^r} = 0 \label{eq:xi_feas_add}
\end{align}
% \begin{align}
%     \grad_{{w}} \lagr \vert_{\optim{{w}}, \optim{\fv{v}_1}, \ldots, \optim{\fv{v}_\ntasks}, \optim{b}, \optim{d_1}, \ldots, \optim{d_\ntasks}, \optim{\fv{\xi}}, \optim{\fv{\alpha}}, \optim{\fv{\beta}}} = 0  &\implies \mu \optim{{w}} - \sum_{r= 1}^T \sum_{i=1}^{m_r} \optim{\alpha_i^r} \left\lbrace y_i^r \phi(x_i^r) \right\rbrace = 0 \\
%     \grad_{{v}_r} \lagr \vert_{\optim{{w}}, \optim{\fv{v}_1}, \ldots, \optim{\fv{v}_\ntasks}, \optim{b}, \optim{d_1}, \ldots, \optim{d_\ntasks}, \optim{\fv{\xi}}, \optim{\fv{\alpha}}, \optim{\fv{\beta}}} = 0 &\implies \optim{{v}_r} - \sum_{i=1}^{m_r} \optim{\alpha_i^r} \left\lbrace y_i^r \phi_r(x_i^r) \right\rbrace = 0 \\
%     \grad_{b} \lagr \vert_{\optim{{w}}, \optim{\fv{v}_1}, \ldots, \optim{\fv{v}_\ntasks}, \optim{b}, \optim{d_1}, \ldots, \optim{d_\ntasks}, \optim{\fv{\xi}}, \optim{\fv{\alpha}}, \optim{\fv{\beta}}} = 0  &\implies \sum_{r= 1}^T \sum_{i=1}^{m_r} \optim{\alpha_i^r} y_i^r = 0  \\
%     \grad_{{v}_r} \lagr \vert_{\optim{{w}}, \optim{\fv{v}_1}, \ldots, \optim{\fv{v}_\ntasks}, \optim{b}, \optim{d_1}, \ldots, \optim{d_\ntasks}, \optim{\fv{\xi}}, \optim{\fv{\alpha}}, \optim{\fv{\beta}}} = 0 &\implies \sum_{r= 1}^T \sum_{i=1}^{m_r} \optim{\alpha_i^r} y_i^r = 0 \\
%     \grad_{\xi_i^r} \lagr \vert_{\optim{{w}}, \optim{\fv{v}_1}, \ldots, \optim{\fv{v}_\ntasks}, \optim{b}, \optim{d_1}, \ldots, \optim{d_\ntasks}, \optim{\fv{\xi}}, \optim{\fv{\alpha}}, \optim{\fv{\beta}}} = 0 &\implies C - \optim{\alpha_i^r} - \optim{\beta_i^r} = 0
% \end{align}

Using these results and substituting back in the Lagrangian we obtain
\begin{equation}\nonumber
    \begin{aligned}
        &  \lagr(\optim{{w}}, \optim{\fv{v}_1}, \ldots, \optim{\fv{v}_\ntasks}, \optim{b}, \optim{d_1}, \ldots, \optim{d_\ntasks}, \optim{\fv{\xi}}, \fv{\alpha}, \fv{\beta}) =\\
        &\quad =  \frac{1}{2} \sum_{r= 1}^T{\norm{\sum_{i=1}^{m_r} {\alpha_i^r} \left\lbrace y_i^r \phi_r(\fv{x}_i^r) \right\rbrace}^2} + \frac{\mu}{2} {\norm{ \frac{1}{\mu}\sum_{r= 1}^T \sum_{i=1}^{m_r} {\alpha_i^r} \left\lbrace y_i^r \phi(\fv{x}_i^r) \right\rbrace}}^2 \\
        &\qquad -  \sum_{r= 1}^T \sum_{i=1}^{m_r} \alpha_i^r \left\lbrace y_{i}^r \left[ \left(\frac{1}{\mu} \sum_{s= 1}^T \sum_{j=1}^{m_s} {\alpha_j^s} \left\lbrace y_j^s \phi(\fv{x}_j^s) \right\rbrace \right) \cdot \phi(\fv{x}_{i}^r) \right]  \right\rbrace \\
        &\qquad -  \sum_{r= 1}^T \sum_{i=1}^{m_r} \alpha_i^r \left\lbrace y_{i}^r \left[  \left( \sum_{j=1}^{m_r} {\alpha_j^r} \left\lbrace y_j^r \phi_r(\fv{x}_j^r) \right\rbrace \right) \cdot \phi_r(\fv{x}_{i}^r)  \right]  \right\rbrace \\
        &\qquad -  \sum_{r= 1}^T \sum_{i=1}^{m_r} \alpha_i^r \left\lbrace - p_{i}^r  \right\rbrace \\
        &\quad =  \frac{1}{2} \sum_{r= 1}^T{\dotp{\sum_{i=1}^{m_r} {\alpha_i^r} \left\lbrace y_i^r \phi_r(\fv{x}_i^r) \right\rbrace}{\sum_{i=1}^{m_r} {\alpha_i^r} \left\lbrace y_i^r \phi_r(\fv{x}_i^r) \right\rbrace}} \\
        &\qquad + \frac{\mu}{2} {\dotp{ \frac{1}{\mu}\sum_{r= 1}^T \sum_{i=1}^{m_r} {\alpha_i^r} \left\lbrace y_i^r \phi(\fv{x}_i^r) \right\rbrace}{ \frac{1}{\mu}\sum_{r= 1}^T \sum_{i=1}^{m_r} {\alpha_i^r} \left\lbrace y_i^r \phi(\fv{x}_i^r) \right\rbrace}} \\
        &\qquad - \frac{1}{\mu} \dotp{ \sum_{r= 1}^T \sum_{i=1}^{m_r} {\alpha_i^r} \left\lbrace y_i^r \phi(\fv{x}_i^r) \right\rbrace}{ \sum_{r= 1}^T \sum_{i=1}^{m_r} {\alpha_i^r} \left\lbrace y_i^r \phi(\fv{x}_i^r) \right\rbrace} \\
        &\qquad -  \sum_{r= 1}^T {\dotp{\sum_{i=1}^{m_r} {\alpha_i^r} \left\lbrace y_i^r \phi_r(\fv{x}_i^r) \right\rbrace}{\sum_{i=1}^{m_r} {\alpha_i^r} \left\lbrace y_i^r \phi_r(\fv{x}_i^r) \right\rbrace}} \\
        &\qquad -  \sum_{r= 1}^T \sum_{i=1}^{m_r} \alpha_i^r \left\lbrace - p_{i}^r  \right\rbrace \\
        &\quad = - \frac{1}{2\mu} \dotp{ \sum_{r= 1}^T \sum_{i=1}^{m_r} {\alpha_i^r} \left\lbrace y_i^r \phi(\fv{x}_i^r) \right\rbrace}{ \sum_{r= 1}^T \sum_{i=1}^{m_r} {\alpha_i^r} \left\lbrace y_i^r \phi(\fv{x}_i^r) \right\rbrace} \\
        &\qquad - \frac{1}{2} \sum_{r= 1}^T {\dotp{\sum_{i=1}^{m_r} {\alpha_i^r} \left\lbrace y_i^r \phi_r(\fv{x}_i^r) \right\rbrace}{\sum_{i=1}^{m_r} {\alpha_i^r} \left\lbrace y_i^r \phi_r(\fv{x}_i^r) \right\rbrace}} \\
        &\qquad +  \sum_{r= 1}^T \sum_{i=1}^{m_r} \alpha_i^r  p_{i}^r 
    \end{aligned}
\end{equation}
Observe that $\fv{\beta}$ has disappeared from the Lagrangian.
%   Dual
Then, the dual problem can be defined as $\min_{\fv{\alpha}} \Theta(\fv{\alpha})$ where 
$$ \Theta(\fv{\alpha}) = - \lagr(\optim{{w}}, \optim{\fv{v}_1}, \ldots, \optim{\fv{v}_\ntasks}, \optim{b}, \optim{d_1}, \ldots, \optim{d_\ntasks}, \optim{\fv{\xi}}, \fv{\alpha}, \fv{\beta}), $$
and $\fv{\alpha}$ must fulfill the KKT conditions. The condition~\eqref{eq:xi_feas_add}, using that $\alpha_i^r , \beta_i^r \geq 0$, implies $0 \leq \alpha_i^r \leq C$. Also, observe that the equality constraint~\eqref{eq:common_eqconstr_add} is included in the task-specific equality constraints~\eqref{eq:specific_eqconstr_add}. Taking into account these KKT conditions, the dual problem can be expressed as
\begin{equation}\label{eq:svmmtl_dual_add}
    \begin{aligned}
    & \min_{\alpha} & \Theta(\alpha) &=  \frac{1}{2\mu} \dotp{ \sum_{r= 1}^T \sum_{i=1}^{m_r} {\alpha_i^r} \left\lbrace y_i^r \phi(\fv{x}_i^r) \right\rbrace}{ \sum_{r= 1}^T \sum_{i=1}^{m_r} {\alpha_i^r} \left\lbrace y_i^r \phi(\fv{x}_i^r) \right\rbrace} \\
    & & &\quad + \frac{1}{2} \sum_{r= 1}^T {\dotp{\sum_{i=1}^{m_r} {\alpha_i^r} \left\lbrace y_i^r \phi_r(\fv{x}_i^r) \right\rbrace}{\sum_{i=1}^{m_r} {\alpha_i^r} \left\lbrace y_i^r \phi_r(\fv{x}_i^r) \right\rbrace}}  - \sum_{r=1}^\ntasks \sum_{i=1}^{\npertask_r} \alpha_i^r p_i^r \\
    & \text{s.t.}
    & & 0 \leq \alpha_i^r \leq C ; \; i=1, \ldots, m_r; r=1, \ldots, \ntasks \\
    & & & \sum_{i=1}^{m_r} \alpha_i^r y_i^r = 0;  r=1, \ldots, \ntasks . \\
    \end{aligned}
\end{equation}
Using the kernel trick, we can write the dual problem using a vector formulation
\begin{equation}\label{eq:svmmtl_dualvec_add}
    \begin{aligned}
    & \min_{\alpha} && \Theta(\alpha) = \frac{1}{2} \fv{\alpha}^\intercal \left(\frac{1}{\mu} \fm{Q} + \fm{K} \right) \fv{\alpha} - \fv{p} \fv{\alpha} \\
    & \text{s.t.}
    & & 0 \leq \alpha_i^r \leq C ; \; i=1, \ldots, m_r; r=1, \ldots, \ntasks \\
    & & & \sum_{i=1}^{m_r} \alpha_i^r y_i^r = 0;  r=1, \ldots, \ntasks . \\
    \end{aligned}
\end{equation}
Here $Q$ and $K$ are the common and specific kernel matrices, respectively.
The matrix $Q$ is generated using the common kernel, defined as 
\begin{equation}
    \nonumber
    k(x_i^r, x_j^s) = \dotp{\phi(x_i^r)}{\phi(x_j^s)} 
\end{equation}
and $K$ is the block-diagonal matrix built using the kernel
\begin{equation}
    \nonumber
    k_r(x_i^r, x_j^s) = \delta_{rs} \dotp{\phi_r(x_i^r)}{\phi_s(x_j^s)} 
\end{equation}
that is 
\begin{equation}
    \nonumber
    \fm{Q} = 
    \begin{pmatrix}
    \underbrace{Q_{1, 1}}_{m_1 \times m_1} & \underbrace{\fm{Q}_{1, 2}}_{m_1 \times m_2} & \underbrace{\fm{Q}_{1, 3}}_{m_1 \times m_3} & \cdots & \underbrace{\fm{Q}_{1, T}}_{m_1 \times m_T} \\
    \underbrace{\fm{Q}_{2, 1}}_{m_2 \times m_1} & \underbrace{Q_{2, 2}}_{m_2 \times m_2} & \underbrace{\fm{Q}_{2, 1}}_{m_2 \times m_1} & \cdots & \underbrace{\fm{Q}_{2, T}}_{m_2 \times m_T} \\
    \vdots      & \vdots &\vdots    & \ddots & \vdots \\
    \underbrace{\fm{Q}_{T, 1}}_{m_T \times m_1} & \underbrace{\fm{Q}_{T, 2}}_{m_T \times m_2} & \underbrace{\fm{Q}_{T, 3}}_{m_T \times m_3} & \cdots & \underbrace{\fm{Q}_{\ntasks, \ntasks}}_{m_\ntasks \times m_\ntasks}
    \end{pmatrix} ,
\end{equation}
where each block $\fm{Q}_{r,s}$ 
\begin{equation}
    \label{eq:kernelmatrix_common}
    \fm{Q}_{r, s} = \begin{pmatrix}
    k(\fv{x}_1^r, \fv{x}_1^s) & k(\fv{x}_1^r, \fv{x}_2^s) & \cdots & k(\fv{x}_1^r, \fv{x}_{m_s}^s) \\
    k(\fv{x}_2^r, \fv{x}_1^s) & k(\fv{x}_2^r, \fv{x}_2^s) & \cdots & k(\fv{x}_2^r, \fv{x}_{m_s}^s) \\
    \vdots & \vdots & \ddots & \vdots \\
    k(\fv{x}_{m_r}^r, \fv{x}_1^s) & k(\fv{x}_{m_r}^r, \fv{x}_2^s) & \cdots & k(\fv{x}_{m_r}^r, \fv{x}_{m_s}^s) \\
    \end{pmatrix} ;
\end{equation}
and 
\begin{equation}
    \nonumber
    \fm{K} = 
    \begin{pmatrix}
    \underbrace{K_{1, 1}}_{m_1 \times m_1} & \underbrace{\fm{0}}_{m_1 \times m_2} & \underbrace{\fm{0}}_{m_1 \times m_3} & \cdots & \underbrace{\fm{0}}_{m_1 \times m_T} \\
    \underbrace{\fm{0}}_{m_2 \times m_1} & \underbrace{K_{2, 2}}_{m_2 \times m_2} & \underbrace{\fm{0}}_{m_2 \times m_1} & \cdots & \underbrace{\fm{0}}_{m_2 \times m_T} \\
    \vdots      & \vdots &\vdots    & \ddots & \vdots \\
    \underbrace{\fm{0}}_{m_T \times m_1} & \underbrace{\fm{0}}_{m_T \times m_2} & \underbrace{\fm{0}}_{m_T \times m_3} & \cdots & \underbrace{K_{\ntasks, \ntasks}}_{m_\ntasks \times \ntasks}
    \end{pmatrix} ,
\end{equation}
where for each task block $\fm{K}_r$ we have 
\begin{equation}
    \label{eq:kernelmatrix_specific}
    \fm{K}_{r, r} = \begin{pmatrix}
    k_r(\fv{x}_1^r, \fv{x}_1^r) & k_r(\fv{x}_1^r, \fv{x}_2^r) & \cdots & k_r(\fv{x}_1^r, \fv{x}_{m_r}^r) \\
    k_r(\fv{x}_2^r, \fv{x}_1^r) & k_r(\fv{x}_2^r, \fv{x}_2^r) & \cdots & k_r(\fv{x}_2^r, \fv{x}_{m_r}^r) \\
    \vdots & \vdots & \ddots & \vdots \\
    k_r(\fv{x}_{m_r}^r, \fv{x}_1^r) & k_r(\fv{x}_{m_r}^r, \fv{x}_2^r) & \cdots & k_r(\fv{x}_{m_r}^r, \fv{x}_{m_r}^r) \\
    \end{pmatrix} .
\end{equation}
Combined, we have a multi-task kernel matrix $\widehat{\fm{Q}} = (1/\mu) \fm{Q} + \fm{K}$, whose corresponding multi-task kernel function can be expressed as 
\begin{equation}
    \nonumber
    \widehat{k}(\fv{x}_i^r, \fv{x}_j^s) = \frac{1}{\mu} k(\fv{x}_i^r, \fv{x}_j^s) + \delta_{rs} k_r(\fv{x}_i^r, \fv{x}_j^s) .
\end{equation}
%   Discussion: Primal vs Dual
The dual problem~\eqref{eq:svmmtl_dualvec_add} is very similar to the standard one but we have two major differences: the use of the multi-task kernel matrix $\widehat{\fm{Q}}$ and the multiple equality constraints. These constraints, which appear in~\eqref{eq:specific_eqconstr_add} are consequence of the specific biases used in the primal problem~\eqref{eq:svmmtl_primal_add}. In~\cite{CaiC12} the authors develop a Generalized SMO algorithm to account for these multiple equality constraints.
Hyperparameter $C$ is an upper bound for the dual coefficients, as in the standard case, but with a different bound for each task. The hyperparameter of interest for this MTL formulation is $\mu$, which, in the dual problem, scales the common matrix $\fm{Q}$. As with the primal formulation, we can define three different cases:
\begin{itemize}
    \item Reduction to an ITL approach:
    $$\mu \tendsto{}{\infty} \implies  \Theta(\alpha) = \frac{1}{2} \fv{\alpha}^\intercal \left(\fm{K} \right) \fv{\alpha} - \fv{p} \fv{\alpha} .$$
    That is the matrix is block-diagonal and optimizing the dual problem is equivalent to optimizing a specific dual problem for each task.
    \item Reduction to a CTL approach: 
    $$C \tendsto{}{0}, \mu \tendsto{}{0} \implies  \Theta(\alpha) = \frac{1}{2} \fv{\alpha}^\intercal \left(\frac{1}{\mu} \fm{Q} \right) \fv{\alpha} - \fv{p} \fv{\alpha}.$$
    That is the dual objective function is the standard one for common task learning.
    \item Pure MTL approach:
    $$ \mu_\text{inf} < \mu < \mu_\text{sup} \implies \Theta(\alpha) = \frac{1}{2} \fv{\alpha}^\intercal \left(\frac{1}{\mu} \fm{Q} + \fm{K} \right) \fv{\alpha} - \fv{p} \fv{\alpha}. $$
    There is a range of $\mu$, which is unknown, in which the kernel matrix combines the common and specific matrices.
\end{itemize}



% Convex Approach
\subsubsection{{convex} MTL L1-SVM}
% Additive Approach
%   Primal
The {convex} MTL primal problem formulation, presented in~\cite{RuizAD19}, changes the formulation of the {additive} MTL SVM but changes its formulation for a convex one that is more interpretable one. The primal problem is
\begin{equation}\label{eq:svmmtl_primal_convex}
    \begin{aligned}
    & \argmin_{w, {v}_r, \xi}
    & & {J({w}, {v}_r, \xi) = C \sum_{r= 1}^T \sum_{i=1}^{m_r} {\xi_{i}^r} + \frac{1}{2} \sum_{r= 1}^T{\norm{{v}_r}^2} + \frac{1}{2} {\norm{{w}}}^2} \\
    & \text{s.t.}
    & & y_{i}^r \left(\lambda_r \left\lbrace {w} \cdot \phi(x_{i}^r) + b \right\rbrace + (1 - \lambda_r) \left\lbrace {v}_r \cdot \phi_r(x_{i}^r) + d_r \right\rbrace  \right) \geq p_{i}^r - \xi_{i}^r ,  \\
    & & & \xi_{i}^r \geq 0; \;  i=1 , \dotsc , m_r, \;  r= 1,\dotsc, T  . \\
    \end{aligned}
\end{equation}
Here, the former hyperparameter $\mu$ used in the regularization is replaced by $\lambda_r$, which is used in the model definition. The prediction model is
\begin{equation}
    \nonumber
    h_r(\fv{x}) = \lambda_r \left\lbrace {w} \cdot \phi(x_{i}^r) + b \right\rbrace + (1 - \lambda_r) \left\lbrace {v}_r \cdot \phi_r(x_{i}^r) + d_r \right\rbrace
\end{equation}
for regression and 
\begin{equation}
    \nonumber
    h_r(\fv{x}) = \sign{ \left(\lambda_r \left\lbrace {w} \cdot \phi(x_{i}^r) + b \right\rbrace + (1 - \lambda_r) \left\lbrace {v}_r \cdot \phi_r(x_{i}^r) + d_r \right\rbrace \right)}
\end{equation}
for classification.
%

With this convex formulation, the roles of hyperparameters $C$ and $\lambda_r$ are independent. Hyperparameter $C$ regulates the trade-off between the loss and the margin size of each task-specialized model $h_r$, while $\lambda_r$ indicates how specific or common these models are in the range of $[0, 1]$. With $\lambda_r=0$ we have independent models for each task and for $\lambda_r=1$ we have a common model for all tasks. 
%
In~\eqref{eq:svmmtl_primal_convex}, depending on the value of hyperparameters $C$, $\lambda_1, \ldots, \lambda_\ntasks$, we can highlight the following situations:
%
\begin{itemize}
    \item Reduction to an ITL approach:
    $$\lambda_r = 0 ; r=1, \ldots, \ntasks \implies h_r(\fv{x}_i^r) = {v}_r \cdot \phi_r(x_{i}^r) + d_r .$$
    \item Reduction to a CTL approach: 
    $$ \lambda_r = 1 ; r=1, \ldots, \ntasks \implies h_r(\fv{x}_i^r) = {w} \cdot \phi(x_{i}^r) + b .$$
    \item Pure MTL approach:
    $$ 0 < \lambda_r < 1 ; r=1, \ldots, \ntasks \implies h_r(\fv{x}_i^r) = \lambda_r ({w} \cdot \phi(x_{i}^r) + b) + (1 - \lambda_r) ({v}_r \cdot \phi_r(x_{i}^r) + d_r) .$$
\end{itemize}
Observe that now the cases are not asymptotical but have clear values, 0 for ITL and 1 for MTL, while all the values in the open $(0, 1)$ yield pure MTL models. Also, the parameter $C$ no longer interferes with these cases and only $\lambda_r$ calibrates the specifity of the models.
%   Lagrangian/KKT
 To obtain the dual problem the Lagrangian of problem~\eqref{eq:svmmtl_primal_convex}:
\begin{equation}\label{eq:svmmtl_lagrangian_convex}
    \begin{aligned}
        & \lagr({w}, \fv{v}_1, \ldots, \fv{v}_\ntasks, b, d_1, \ldots, d_\ntasks, \fv{\xi}, \fv{\alpha}, \fv{\beta}) \\
        &=  C \sum_{r= 1}^T \sum_{i=1}^{m_r} {\xi_{i}^r} + \frac{1}{2} \sum_{r= 1}^T{\norm{{v}_r}^2} + \frac{1}{2} {\norm{{w}}}^2 \\
        &\quad -  \sum_{r= 1}^T \sum_{i=1}^{m_r} \alpha_i^r \left\lbrace y_{i}^r  \left(\lambda_r \left\lbrace {w} \cdot \phi(x_{i}^r) + b \right\rbrace + (1 - \lambda_r) \left\lbrace {v}_r \cdot \phi_r(x_{i}^r) + d_r \right\rbrace  \right) - p_{i}^r + \xi_{i}^r  \right\rbrace \\
        &\quad -  \sum_{r= 1}^T \sum_{i=1}^{m_r} \beta_i^r \xi_i^r ,
    \end{aligned}
\end{equation}
where $\alpha_i^r, \beta_i^r \geq 0$ are the Lagrange multipliers. Again, $\fv{\xi}$ represents the vector $$(\xi_1^1, \ldots, \xi_{m_1}^1, \ldots, \xi_1^\ntasks, \ldots, \xi_{m_\ntasks}^\ntasks)^\intercal$$ and analogously for $\fv{\alpha}$ and $\fv{\beta}$.The dual objective function is defined as 
\begin{equation}\nonumber
    \begin{aligned}
         \Theta(\fv{\alpha}, \fv{\beta}) &=  \min_{{w}, \fv{v}_1, \ldots, \fv{v}_\ntasks, b, d_1, \ldots, d_\ntasks, \fv{\xi}} \lagr({w}, \fv{v}_1, \ldots, \fv{v}_\ntasks, b, d_1, \ldots, d_\ntasks, \fv{\xi}, \fv{\alpha}, \fv{\beta}) \\
         &= \lagr(\optim{{w}}, \optim{\fv{v}_1}, \ldots, \optim{\fv{v}_\ntasks}, \optim{b}, \optim{d_1}, \ldots, \optim{d_\ntasks}, \optim{\fv{\xi}}, \fv{\alpha}, \fv{\beta})
    \end{aligned}    
\end{equation}
The gradients with respect to the primal variables are
\begin{align}
    \grad_{{w}} \lagr \vert_{\optim{{w}}, \optim{\fv{v}_1}, \ldots, \optim{\fv{v}_\ntasks}, \optim{b}, \optim{d_1}, \ldots, \optim{d_\ntasks}, \optim{\fv{\xi}}, {\fv{\alpha}}, {\fv{\beta}}} = 0  &\implies \optim{{w}} - \sum_{r= 1}^\ntasks \lambda_r \sum_{i=1}^{m_r} {\alpha_i^r} \left\lbrace y_i^r \phi(x_i^r) \right\rbrace = 0 , \label{eq:common_repr_conv} \\
    \grad_{{v}_r} \lagr \vert_{\optim{{w}}, \optim{\fv{v}_1}, \ldots, \optim{\fv{v}_\ntasks}, \optim{b}, \optim{d_1}, \ldots, \optim{d_\ntasks}, \optim{\fv{\xi}}, {\fv{\alpha}}, {\fv{\beta}}} = 0 &\implies \optim{{v}_r} - (1 - \lambda_r) \sum_{i=1}^{m_r} {\alpha_i^r} \left\lbrace y_i^r \phi_r(x_i^r) \right\rbrace = 0 , \label{eq:specific_repr_conv} \\
    \grad_{b} \lagr \vert_{\optim{{w}}, \optim{\fv{v}_1}, \ldots, \optim{\fv{v}_\ntasks}, \optim{b}, \optim{d_1}, \ldots, \optim{d_\ntasks}, \optim{\fv{\xi}}, {\fv{\alpha}}, {\fv{\beta}}} = 0  &\implies \sum_{r= 1}^T \sum_{i=1}^{m_r} {\alpha_i^r} y_i^r = 0 , \label{eq:specific_eqconstr_conv}  \\
    \grad_{{d}_r} \lagr \vert_{\optim{{w}}, \optim{\fv{v}_1}, \ldots, \optim{\fv{v}_\ntasks}, \optim{b}, \optim{d_1}, \ldots, \optim{d_\ntasks}, \optim{\fv{\xi}}, {\fv{\alpha}}, {\fv{\beta}}} = 0 &\implies \sum_{i=1}^{m_r} {\alpha_i^r} y_i^r = 0 , \label{eq:common_eqconstr_conv} \\
    \grad_{\xi_i^r} \lagr \vert_{\optim{{w}}, \optim{\fv{v}_1}, \ldots, \optim{\fv{v}_\ntasks}, \optim{b}, \optim{d_1}, \ldots, \optim{d_\ntasks}, \optim{\fv{\xi}}, {\fv{\alpha}}, {\fv{\beta}}} = 0 &\implies C - {\alpha_i^r} - {\beta_i^r} = 0 \label{eq:xi_feas_conv}
\end{align}
% \begin{align}
%     \grad_{{w}} \lagr \vert_{\optim{{w}}, \optim{\fv{v}_1}, \ldots, \optim{\fv{v}_\ntasks}, \optim{b}, \optim{d_1}, \ldots, \optim{d_\ntasks}, \optim{\fv{\xi}}, \optim{\fv{\alpha}}, \optim{\fv{\beta}}} = 0  &\implies \mu \optim{{w}} - \sum_{r= 1}^T \sum_{i=1}^{m_r} \optim{\alpha_i^r} \left\lbrace y_i^r \phi(x_i^r) \right\rbrace = 0 \\
%     \grad_{{v}_r} \lagr \vert_{\optim{{w}}, \optim{\fv{v}_1}, \ldots, \optim{\fv{v}_\ntasks}, \optim{b}, \optim{d_1}, \ldots, \optim{d_\ntasks}, \optim{\fv{\xi}}, \optim{\fv{\alpha}}, \optim{\fv{\beta}}} = 0 &\implies \optim{{v}_r} - \sum_{i=1}^{m_r} \optim{\alpha_i^r} \left\lbrace y_i^r \phi_r(x_i^r) \right\rbrace = 0 \\
%     \grad_{b} \lagr \vert_{\optim{{w}}, \optim{\fv{v}_1}, \ldots, \optim{\fv{v}_\ntasks}, \optim{b}, \optim{d_1}, \ldots, \optim{d_\ntasks}, \optim{\fv{\xi}}, \optim{\fv{\alpha}}, \optim{\fv{\beta}}} = 0  &\implies \sum_{r= 1}^T \sum_{i=1}^{m_r} \optim{\alpha_i^r} y_i^r = 0  \\
%     \grad_{{v}_r} \lagr \vert_{\optim{{w}}, \optim{\fv{v}_1}, \ldots, \optim{\fv{v}_\ntasks}, \optim{b}, \optim{d_1}, \ldots, \optim{d_\ntasks}, \optim{\fv{\xi}}, \optim{\fv{\alpha}}, \optim{\fv{\beta}}} = 0 &\implies \sum_{r= 1}^T \sum_{i=1}^{m_r} \optim{\alpha_i^r} y_i^r = 0 \\
%     \grad_{\xi_i^r} \lagr \vert_{\optim{{w}}, \optim{\fv{v}_1}, \ldots, \optim{\fv{v}_\ntasks}, \optim{b}, \optim{d_1}, \ldots, \optim{d_\ntasks}, \optim{\fv{\xi}}, \optim{\fv{\alpha}}, \optim{\fv{\beta}}} = 0 &\implies C - \optim{\alpha_i^r} - \optim{\beta_i^r} = 0
% \end{align}

Using these results and substituting back in the Lagrangian we obtain
\begin{equation}\nonumber
    \begin{aligned}
        & \lagr(\optim{{w}}, \optim{\fv{v}_1}, \ldots, \optim{\fv{v}_\ntasks}, \optim{b}, \optim{d_1}, \ldots, \optim{d_\ntasks}, \optim{\fv{\xi}}, \fv{\alpha}, \fv{\beta})\\
        &\quad=  \frac{1}{2} \sum_{r= 1}^T{\norm{ (1 - \lambda_r) \sum_{i=1}^{m_r}  {\alpha_i^r} \left\lbrace y_i^r \phi_r(\fv{x}_i^r) \right\rbrace}^2} + \frac{1}{2} {\norm{\sum_{r= 1}^T \lambda_r \sum_{i=1}^{m_r} {\alpha_i^r} \left\lbrace y_i^r \phi(\fv{x}_i^r) \right\rbrace}}^2 \\
        &\qquad - \sum_{r= 1}^T \lambda_r \sum_{i=1}^{m_r} \alpha_i^r \left\lbrace y_{i}^r \left[ \left(\sum_{s= 1}^T \lambda_r \sum_{j=1}^{m_s} {\alpha_j^s} \left\lbrace y_j^s \phi(\fv{x}_j^s) \right\rbrace \right) \cdot \phi(\fv{x}_{i}^r) \right]  \right\rbrace \\
        &\qquad -  \sum_{r= 1}^T (1-\lambda_r) \sum_{i=1}^{m_r} \alpha_i^r \left\lbrace y_{i}^r \left[  \left((1-\lambda_r) \sum_{j=1}^{m_r} {\alpha_j^r} \left\lbrace y_j^r \phi_r(\fv{x}_j^r) \right\rbrace \right) \cdot \phi_r(\fv{x}_{i}^r)  \right]  \right\rbrace \\
        &\qquad -  \sum_{r= 1}^T \sum_{i=1}^{m_r} \alpha_i^r \left\lbrace - p_{i}^r  \right\rbrace \\
        &\quad=  \frac{1}{2} \sum_{r= 1}^T{\dotp{(1-\lambda_r)\sum_{i=1}^{m_r} {\alpha_i^r} \left\lbrace y_i^r \phi_r(\fv{x}_i^r) \right\rbrace}{(1-\lambda_r)\sum_{i=1}^{m_r} {\alpha_i^r} \left\lbrace y_i^r \phi_r(\fv{x}_i^r) \right\rbrace}} \\
        &\qquad + \frac{1}{2} {\dotp{\sum_{r= 1}^\ntasks \lambda_r \sum_{i=1}^{m_r} {\alpha_i^r} \left\lbrace y_i^r \phi(\fv{x}_i^r) \right\rbrace}{\sum_{r= 1}^\ntasks \lambda_r \sum_{i=1}^{m_r} {\alpha_i^r} \left\lbrace y_i^r \phi(\fv{x}_i^r) \right\rbrace}} \\
        &\qquad - \dotp{ \sum_{r= 1}^T \lambda_r \sum_{i=1}^{m_r} {\alpha_i^r} \left\lbrace y_i^r \phi(\fv{x}_i^r) \right\rbrace}{ \sum_{r= 1}^T \lambda_r \sum_{i=1}^{m_r} {\alpha_i^r} \left\lbrace y_i^r \phi(\fv{x}_i^r) \right\rbrace} \\
        &\qquad -  \sum_{r= 1}^T {\dotp{(1-\lambda_r) \sum_{i=1}^{m_r} {\alpha_i^r} \left\lbrace y_i^r \phi_r(\fv{x}_i^r) \right\rbrace}{(1-\lambda_r) \sum_{i=1}^{m_r} {\alpha_i^r} \left\lbrace y_i^r \phi_r(\fv{x}_i^r) \right\rbrace}} \\
        &\qquad -  \sum_{r= 1}^T \sum_{i=1}^{m_r} \alpha_i^r \left\lbrace - p_{i}^r  \right\rbrace \\
        &\quad= - \frac{1}{2} \dotp{ \sum_{r= 1}^T  \lambda_r \sum_{i=1}^{m_r} {\alpha_i^r} \left\lbrace y_i^r \phi(\fv{x}_i^r) \right\rbrace}{ \sum_{r= 1}^T \lambda_r \sum_{i=1}^{m_r} {\alpha_i^r} \left\lbrace y_i^r \phi(\fv{x}_i^r) \right\rbrace} \\
        &\qquad - \frac{1}{2} \sum_{r= 1}^T {(1-\lambda_r) \dotp{\sum_{i=1}^{m_r} {\alpha_i^r} \left\lbrace y_i^r \phi_r(\fv{x}_i^r) \right\rbrace}{(1-\lambda_r) \sum_{i=1}^{m_r} {\alpha_i^r} \left\lbrace y_i^r \phi_r(\fv{x}_i^r) \right\rbrace}} \\
        &\qquad +  \sum_{r= 1}^T \sum_{i=1}^{m_r} \alpha_i^r  p_{i}^r 
    \end{aligned}
\end{equation}
%   Dual
As with the additive formulation, the dual problem can then be defined as 
\begin{equation}\label{eq:svmmtl_dual_conv}
    \begin{aligned}
    & \min_{\alpha} & \Theta(\alpha) &=  \frac{1}{2} \dotp{\sum_{r= 1}^\ntasks \lambda_r \sum_{i=1}^{m_r} {\alpha_i^r} \left\lbrace y_i^r \phi(\fv{x}_i^r) \right\rbrace}{\sum_{r= 1}^\ntasks \lambda_r \sum_{i=1}^{m_r} {\alpha_i^r} \left\lbrace y_i^r \phi(\fv{x}_i^r) \right\rbrace} \\
    & & &\quad + \frac{1}{2} \sum_{r= 1}^T {\dotp{(1 - \lambda_r)\sum_{i=1}^{m_r} {\alpha_i^r} \left\lbrace y_i^r \phi_r(\fv{x}_i^r) \right\rbrace}{(1 - \lambda_r) \sum_{i=1}^{m_r} {\alpha_i^r} \left\lbrace y_i^r \phi_r(\fv{x}_i^r) \right\rbrace}} \\
    & & &\quad - \sum_{r=1}^\ntasks \sum_{i=1}^{\npertask_r} \alpha_i^r p_i^r   \\
    & \text{s.t.}
    & & 0 \leq \alpha_i^r \leq C ; \; i=1, \ldots, m_r; r=1, \ldots, \ntasks \\
    & & & \sum_{i=1}^{m_r} \alpha_i^r y_i^r = 0;  r=1, \ldots, \ntasks . \\
    \end{aligned}
\end{equation}
And the vector formulation using the kernel matrices is
\begin{equation}\label{eq:svmmtl_dualvec_conv}
    \begin{aligned}
    & \min_{\alpha} && \Theta(\alpha) = \frac{1}{2} \fv{\alpha}^\intercal \left(\Lambda \fm{Q} \Lambda + \left(\fm{I}_{\nsamples} - \Lambda \right) \fm{K} \left(\fm{I}_{\nsamples} - \Lambda \right) \right) \fv{\alpha} - \fv{p} \fv{\alpha} \\
    & \text{s.t.}
    & & 0 \leq \alpha_i^r \leq C ; \; i=1, \ldots, m_r; r=1, \ldots, \ntasks \\
    & & & \sum_{i=1}^{m_r} \alpha_i^r y_i^r = 0;  r=1, \ldots, \ntasks . \\
    \end{aligned}
\end{equation}
where
$$
\Lambda = \Diag(\lambda_1, \ldots, \lambda_\ntasks)
$$
and $\fm{I}_{\nsamples}$ is the $\nsamples \times \nsamples$ identity matrix.
%
Here $Q$ and $K$ are the common and specific kernel matrices, respectively.
Combined, we have a multi-task kernel matrix $\widehat{\fm{Q}} = \Lambda \fm{Q} \Lambda + \left(\fm{I}_{\nsamples} - \Lambda \right) \fm{K} \left(\fm{I}_{\nsamples} - \Lambda \right)$, whose corresponding multi-task kernel function can be expressed as 
\begin{equation}
    \nonumber
    \widehat{k}(\fv{x}_i^r, \fv{x}_j^s) = \lambda_r^2 k(\fv{x}_i^r, \fv{x}_j^s) +  \delta_{rs} (1-\lambda_r)^2 k_r(\fv{x}_i^r, \fv{x}_j^s) .
\end{equation}
%   Discussion: Primal vs Dual
This dual problem is very similar to the one shown in~\eqref{eq:svmmtl_dualvec_add} where there are also $\ntasks$ equality constraints, but the multi-task kernel matrix $\widehat{\fm{Q}}$ is defined differently, dropping the $\mu$ hyperparameter and incorporating the $\lambda_r$ ones.
Studying the influence of $\lambda_r$ hyperparameters in the dual problem we can describe the following cases:
\begin{itemize}
    \item Reduction to an ITL approach:
    $$\lambda_r = 0; r=1, \ldots, \ntasks  \implies  \Theta(\alpha) = \frac{1}{2} \fv{\alpha}^\intercal \left(\fm{K} \right) \fv{\alpha} - \fv{p} \fv{\alpha} .$$
    \item Reduction to a CTL approach: 
    $$\lambda_r = 1; r=1, \ldots, \ntasks \implies  \Theta(\alpha) = \frac{1}{2} \fv{\alpha}^\intercal \left(\fm{Q} \right) \fv{\alpha} - \fv{p} \fv{\alpha}.$$
    \item Pure MTL approach:
    $$ 0 < \lambda_r < 1; r=1, \ldots, \ntasks \implies \Theta(\alpha) = \frac{1}{2} \fv{\alpha}^\intercal \left(\Lambda \fm{Q} \Lambda + \left(\fm{I}_{\nsamples} - \Lambda \right) \fm{K} \left(\fm{I}_{\nsamples} - \Lambda \right) \right) \fv{\alpha} - \fv{p} \fv{\alpha}. $$
\end{itemize}
The properties that are found in the primal formulation are also present in the dual one. All the values of $\lambda_r$ in the open interval $(0, 1)$ correspond to pure MTL approaches while the extreme values $\lambda_r=1$ and $\lambda_r=0$ correspond to CTL and ITL approaches, respectively.

\subsubsection{Equivalence results}
% Equivalence results
Both the {additive} and {convex} MTL SVM approaches solve a similar problem, but there is a change in the formulation to get rid of a regularization hyperparameter $\mu$ in favor of those defining convex combination of models, hyperparameters $\lambda_r$.
Both approaches offer similar properties: ranging the value of their hyperparameters to go from completely common to completely independent models, and passing through pure multi-task models.
However, it is not totally trivial what is the relation between those two approaches.
% Lemmas
In~\cite{RuizAD19} we provide two propositions to show the equivalence between {additive} and {convex} MTL SVM formulations.
\begin{prop}\label{prop:add_conv_equiv}
    The {additive} MTL-SVM primal problem with parameters $C_\text{add}$ and $\mu$ (and possibly $\epsilon$), that is,
    \begin{equation}\label{eq:svmmtl_primal_add_equiv}
        \begin{aligned}
        & \argmin_{w, {v}_r, \xi}
        & & {J({w}, {v}_r, \xi) = C_\text{add} \sum_{r= 1}^T \sum_{i=1}^{m_r} {\xi_{i}^r} + \frac{1}{2} \sum_{r= 1}^T{\norm{{v}_r}^2} + \frac{\mu}{2} {\norm{{w}}}^2} \\
        & \text{s.t.}
        & & y_{i}^r ({w} \cdot \phi(x_{i}^r) + b + {v}_r \cdot \phi_r(x_{i}^r) + d_r) \geq p_{i}^r - \xi_{i}^r ,  \\
        & & & \xi_{i}^r \geq 0; \;  i=1 , \dotsc , m_r, \;  r= 1,\dotsc, T  , \\
        \end{aligned}
    \end{equation}
    and the {convex} MTL-SVM primal problem with parameters $C_\text{conv}$ and $\lambda_1=\ldots=\lambda_\ntasks = \lambda$ (and possibly $\epsilon$), that is,
    \begin{equation}\label{eq:svmmtl_primal_convex_equiv}
        \begin{aligned}
        & \argmin_{u, {u}_r, \xi}
        & & {J({u}, {u}_r, \xi) = C_\text{conv} \sum_{r= 1}^T \sum_{i=1}^{m_r} {\xi_{i}^r} + \frac{1}{2} \sum_{r= 1}^T{\norm{{u}_r}^2} + \frac{1}{2} {\norm{{u}}}^2} \\
        & \text{s.t.}
        & & y_{i}^r \left(\lambda \left\lbrace {u} \cdot \phi(x_{i}^r) + \hat{b} \right\rbrace + (1 - \lambda) \left\lbrace {u}_r \cdot \phi_r(x_{i}^r) + \hat{d}_r \right\rbrace  \right) \geq p_{i}^r - \xi_{i}^r ,  \\
        & & & \xi_{i}^r \geq 0; \;  i=1 , \dotsc , m_r, \;  r= 1,\dotsc, T  . \\
        \end{aligned}
    \end{equation}
    where $\lambda \in (0, 1)$, are equivalent when $C_\text{add} = (1 - \lambda)^2 C_\text{conv}$ and $\mu = (1 - \lambda)^2 / \lambda^2$.
    \label{thm_equiv}
\end{prop}
\begin{proof}
    Making the change of variables $w = \lambda u$, $v_r = (1 - \lambda)u_r$, $b = \lambda \hat{b}$ and $d_r = (1 - \lambda) \hat{d_r}$ in the convex primal problem~\eqref{eq:svmmtl_primal_convex_equiv}, we can write it as 
        \begin{equation*}
            \begin{aligned}
            & \argmin_{w, v_r, \xi}
            & & {J(w, v_r, \xi) = C_\text{conv} \sum_{t= 1}^T \sum_{i=1}^{m_r}{\xi_{i}^r} + \frac{1}{2 (1-\lambda)^2} \sum_{t= 1}^T{\norm{v_r}^2} + \frac{1}{2 \lambda^2} {\norm{w}}^2} \\
            & \text{s.t.}
            & & y_{i}^r (w \cdot \phi(x_{i}^r) + b + v_r \cdot \phi_r(x_{i}^r) + d_r) \geq p_{i}^r - \xi_{i}^r , \\
            & & & \xi_{i}^r \geq 0,  \\
            & & & i=1,\dotsc,m_r, \;  t= 1,\dotsc,T ,
            \end{aligned}
        \end{equation*}
    Multiplying now the objective function by $(1 - \lambda)^2$ we obtain the additive MTL-SVM primal problem~\eqref{eq:svmmtl_primal_add_equiv} with $\mu =(1 - \lambda)^2 / \lambda^2$ and $C_\text{add} = (1-\lambda)^2 C_\text{conv}$.
    Conversely, we can start at the primal additive problem and make the inverse changes to arrive now to the primal convex problem.
\end{proof}

The previous proposition shows the equivalence between the primal problems, but this result can also be obtained from the dual problems. 
Consider the dual problem of the convex formulation when $\lambda_1 = \ldots = \lambda_\ntasks = \lambda$, and multiplying the objective function by $\frac{1}{(1 - \lambda)^2} > 0$ we get
\begin{equation}\label{eq:svmmtl_dualvec_conv_equiv}
    \begin{aligned}
    & \min_{\fv{\alpha}} && \Theta(\fv{\alpha}) = \frac{1}{2} \fv{\alpha}^\intercal \left(\frac{\lambda^2}{(1 - \lambda)^2} \fm{Q} + \frac{(1 - \lambda)^2}{(1 - \lambda)^2} \fm{K} \right) \fv{\alpha} - \frac{1}{(1 - \lambda)^2} \fv{p} \fv{\alpha} \\
    & \text{s.t.}
    & & 0 \leq \alpha_i^r \leq C_\text{conv} ; \; i=1, \ldots, m_r; r=1, \ldots, \ntasks \\
    & & & \sum_{i=1}^{m_r} \alpha_i^r y_i^r = 0;  r=1, \ldots, \ntasks , \\
    \end{aligned}
\end{equation}
and consider the change of variables 
$ \fv{\beta} = (1 - \lambda)^2 \fv{\alpha}$, then we obtain the problem
\begin{equation}\nonumber
    \begin{aligned}
    & \min_{\fv{\beta}} && \Theta(\fv{\beta}) = \frac{1}{2} \frac{1}{(1 - \lambda)^2}\fv{\beta}^\intercal \left(\frac{\lambda^2}{(1 - \lambda)^2} \fm{Q} + \frac{(1 - \lambda)^2}{(1 - \lambda)^2} \fm{K} \right) \frac{1}{(1 - \lambda)^2} \fv{\beta} - \frac{1}{(1 - \lambda)^4} \fv{p}  \fv{\beta} \\
    & \text{s.t.}
    & & 0 \leq \beta_i^r \leq (1 - \lambda)^2 C_\text{conv} ; \; i=1, \ldots, m_r; r=1, \ldots, \ntasks \\
    & & & \sum_{i=1}^{m_r} \alpha_i^r y_i^r = 0;  r=1, \ldots, \ntasks , \\
    \end{aligned}
\end{equation}
if we consider again $\mu =(1 - \lambda)^2 / \lambda^2$ and $C_\text{add} = (1-\lambda)^2 C_\text{conv}$ and multiply the objective function by $(1 - \lambda)^4$ we get the equivalent problem
% Experiments
\begin{equation}\nonumber
    \begin{aligned}
    & \min_{\fv{\beta}} && \Theta(\fv{\beta}) = \frac{1}{2} \fv{\beta}^\intercal \left(\frac{1}{\mu} \fm{Q} + \fm{K} \right) \fv{\beta} - \fv{p}  \fv{\beta} \\
    & \text{s.t.}
    & & 0 \leq \beta_i^r \leq  C_\text{add} ; \; i=1, \ldots, m_r; r=1, \ldots, \ntasks \\
    & & & \sum_{i=1}^{m_r} \alpha_i^r y_i^r = 0;  r=1, \ldots, \ntasks , \\
    \end{aligned}
\end{equation}
which is the dual problem of the additive formulation.

Finally, observe that the results in Proposition~\ref{prop:add_conv_equiv} is only valid for $\lambda$ in the open set $(0, 1)$. The results in the extremes of the interval, i.e. $\lambda=0, 1$, have been already exposed. When $\lambda=0$, we obtain an ITL approach, where an independent model is learned for each task. When $\lambda=1$, the convex formulation is equivalent to a CTL approach, where a single, common model is trained using all tasks.

% TODO: Equivalence results for \lambda=0,1

\subsection{L2 Support Vector Machine}
The L2-SVM~\cite{Burges98} is a variant of the standard SVM where the hinge loss is replaced by the squared hinge loss in the case of a classification setting, and the epsilon-insensitive loss by the corresponding squared version in the case of regression problems. In both settings a margin where the errors are not penalized is kept, but the errors which are larger than such margin are penalized using its squared value.
In this subsection I present directly an MTL approach using L2-SVMs with a convex formulation.

% Convex Approach
\subsubsection{{Convex} MTL L2-SVM}
% Additive Approach
%   Primal
The primal problem for the convex MTL L2-SVM, presented in~\cite{RuizAD21}, is
\begin{equation}\label{eq:svmmtl_primal_convex_l2}
    \begin{aligned}
    & \argmin_{w, {v}_r, \xi}
    & & {J({w}, {v}_r, \xi) = \frac{C}{2} \sum_{r= 1}^T \sum_{i=1}^{m_r} {\xi_{i}^r}^2 + \frac{1}{2} \sum_{r= 1}^T{\norm{{v}_r}^2} + \frac{1}{2} {\norm{{w}}}^2} \\
    & \text{s.t.}
    & & y_{i}^r \left(\lambda_r \left\lbrace {w} \cdot \phi(x_{i}^r) + b \right\rbrace + (1 - \lambda_r) \left\lbrace {v}_r \cdot \phi_r(x_{i}^r) + d_r \right\rbrace  \right) \geq p_{i}^r - \xi_{i}^r .  \\
    \end{aligned}
\end{equation}
The prediction model is again
\begin{equation}
    \nonumber
    h_r(\fv{x}) = \lambda_r \left\lbrace {w} \cdot \phi(x_{i}^r) + b \right\rbrace + (1 - \lambda_r) \left\lbrace {v}_r \cdot \phi_r(x_{i}^r) + d_r \right\rbrace
\end{equation}
for regression and 
\begin{equation}
    \nonumber
    h_r(\fv{x}) = \sign{ \left(\lambda_r \left\lbrace {w} \cdot \phi(x_{i}^r) + b \right\rbrace + (1 - \lambda_r) \left\lbrace {v}_r \cdot \phi_r(x_{i}^r) + d_r \right\rbrace \right)}
\end{equation}
for classification.
%

As in the standard variant, hyperparameter $C$ regulates the trade-off between the loss and the margin size of each task-specialized model $h_r$, while $\lambda_r$ indicates how specific or common these models are in the range of $[0, 1]$. With $\lambda_1, \ldots, \lambda_\ntasks=0$ we have independent models for each task and for $\lambda_1, \ldots, \lambda_\ntasks=1$ we have a common model for all tasks. 
 


%   Lagrangian/KKT
 To obtain the dual problem the Lagrangian of problem~\eqref{eq:svmmtl_primal_convex_l2}:
\begin{equation}\label{eq:svmmtl_lagrangian_convex_l2}
    \begin{aligned}
        & \lagr({w}, \fv{v}_1, \ldots, \fv{v}_\ntasks, b, d_1, \ldots, d_\ntasks, \fv{\xi}, \fv{\alpha}) \\
        &=  \frac{C}{2} \sum_{r= 1}^T \sum_{i=1}^{m_r} ({\xi_{i}^r})^2 + \frac{1}{2} \sum_{r= 1}^T{\norm{{v}_r}^2} + \frac{1}{2} {\norm{{w}}}^2 \\
        &\quad -  \sum_{r= 1}^T \sum_{i=1}^{m_r} \alpha_i^r \left\lbrace y_{i}^r  \left(\lambda_r \left\lbrace {w} \cdot \phi(x_{i}^r) + b \right\rbrace + (1 - \lambda_r) \left\lbrace {v}_r \cdot \phi_r(x_{i}^r) + d_r \right\rbrace  \right) - p_{i}^r + \xi_{i}^r  \right\rbrace ,
    \end{aligned}
\end{equation}
where $\alpha_i^r \geq 0$ are the Lagrange multipliers. Again, $\fv{\xi}$ represents the vector $$(\xi_1^1, \ldots, \xi_{m_1}^1, \ldots, \xi_1^\ntasks, \ldots, \xi_{m_\ntasks}^\ntasks)^\intercal$$ and analogously for $\fv{\alpha}$. The dual objective function is defined as 
\begin{equation}\nonumber
    \begin{aligned}
         \Theta(\fv{\alpha}) &=  \min_{{w}, \fv{v}_1, \ldots, \fv{v}_\ntasks, b, d_1, \ldots, d_\ntasks, \fv{\xi}} \lagr({w}, \fv{v}_1, \ldots, \fv{v}_\ntasks, b, d_1, \ldots, d_\ntasks, \fv{\xi}, \fv{\alpha}) \\
         &= \lagr(\optim{{w}}, \optim{\fv{v}_1}, \ldots, \optim{\fv{v}_\ntasks}, \optim{b}, \optim{d_1}, \ldots, \optim{d_\ntasks}, \optim{\fv{\xi}}, \fv{\alpha})
    \end{aligned}    
\end{equation}
The gradients with respect to the primal variables are
\begin{align}
    \grad_{{w}} \lagr \vert_{\optim{{w}}, \optim{\fv{v}_1}, \ldots, \optim{\fv{v}_\ntasks}, \optim{b}, \optim{d_1}, \ldots, \optim{d_\ntasks}, \optim{\fv{\xi}}, {\fv{\alpha}}} = 0  &\implies \optim{{w}} - \sum_{r= 1}^\ntasks \lambda_r \sum_{i=1}^{m_r} {\alpha_i^r} \left\lbrace y_i^r \phi(x_i^r) \right\rbrace = 0 , \label{eq:common_repr_conv_l2} \\
    \grad_{{v}_r} \lagr \vert_{\optim{{w}}, \optim{\fv{v}_1}, \ldots, \optim{\fv{v}_\ntasks}, \optim{b}, \optim{d_1}, \ldots, \optim{d_\ntasks}, \optim{\fv{\xi}}, {\fv{\alpha}}} = 0 &\implies \optim{{v}_r} - (1 - \lambda_r) \sum_{i=1}^{m_r} {\alpha_i^r} \left\lbrace y_i^r \phi_r(x_i^r) \right\rbrace = 0 , \label{eq:specific_repr_conv_l2} \\
    \grad_{b} \lagr \vert_{\optim{{w}}, \optim{\fv{v}_1}, \ldots, \optim{\fv{v}_\ntasks}, \optim{b}, \optim{d_1}, \ldots, \optim{d_\ntasks}, \optim{\fv{\xi}}, {\fv{\alpha}}} = 0  &\implies \sum_{r= 1}^T \sum_{i=1}^{m_r} {\alpha_i^r} y_i^r = 0 , \label{eq:specific_eqconstr_conv_l2}  \\
    \grad_{{d}_r} \lagr \vert_{\optim{{w}}, \optim{\fv{v}_1}, \ldots, \optim{\fv{v}_\ntasks}, \optim{b}, \optim{d_1}, \ldots, \optim{d_\ntasks}, \optim{\fv{\xi}}, {\fv{\alpha}}} = 0 &\implies \sum_{i=1}^{m_r} {\alpha_i^r} y_i^r = 0 , \label{eq:common_eqconstr_conv_l2} \\
    \grad_{\xi_i^r} \lagr \vert_{\optim{{w}}, \optim{\fv{v}_1}, \ldots, \optim{\fv{v}_\ntasks}, \optim{b}, \optim{d_1}, \ldots, \optim{d_\ntasks}, \optim{\fv{\xi}}, {\fv{\alpha}}} = 0 &\implies C \xi_i^r - {\alpha_i^r} = 0 . \label{eq:xi_feas_conv_l2}
\end{align}
% \begin{align}
%     \grad_{{w}} \lagr \vert_{\optim{{w}}, \optim{\fv{v}_1}, \ldots, \optim{\fv{v}_\ntasks}, \optim{b}, \optim{d_1}, \ldots, \optim{d_\ntasks}, \optim{\fv{\xi}}, \optim{\fv{\alpha}}, \optim{\fv{\beta}}} = 0  &\implies \mu \optim{{w}} - \sum_{r= 1}^T \sum_{i=1}^{m_r} \optim{\alpha_i^r} \left\lbrace y_i^r \phi(x_i^r) \right\rbrace = 0 \\
%     \grad_{{v}_r} \lagr \vert_{\optim{{w}}, \optim{\fv{v}_1}, \ldots, \optim{\fv{v}_\ntasks}, \optim{b}, \optim{d_1}, \ldots, \optim{d_\ntasks}, \optim{\fv{\xi}}, \optim{\fv{\alpha}}, \optim{\fv{\beta}}} = 0 &\implies \optim{{v}_r} - \sum_{i=1}^{m_r} \optim{\alpha_i^r} \left\lbrace y_i^r \phi_r(x_i^r) \right\rbrace = 0 \\
%     \grad_{b} \lagr \vert_{\optim{{w}}, \optim{\fv{v}_1}, \ldots, \optim{\fv{v}_\ntasks}, \optim{b}, \optim{d_1}, \ldots, \optim{d_\ntasks}, \optim{\fv{\xi}}, \optim{\fv{\alpha}}, \optim{\fv{\beta}}} = 0  &\implies \sum_{r= 1}^T \sum_{i=1}^{m_r} \optim{\alpha_i^r} y_i^r = 0  \\
%     \grad_{{v}_r} \lagr \vert_{\optim{{w}}, \optim{\fv{v}_1}, \ldots, \optim{\fv{v}_\ntasks}, \optim{b}, \optim{d_1}, \ldots, \optim{d_\ntasks}, \optim{\fv{\xi}}, \optim{\fv{\alpha}}, \optim{\fv{\beta}}} = 0 &\implies \sum_{r= 1}^T \sum_{i=1}^{m_r} \optim{\alpha_i^r} y_i^r = 0 \\
%     \grad_{\xi_i^r} \lagr \vert_{\optim{{w}}, \optim{\fv{v}_1}, \ldots, \optim{\fv{v}_\ntasks}, \optim{b}, \optim{d_1}, \ldots, \optim{d_\ntasks}, \optim{\fv{\xi}}, \optim{\fv{\alpha}}, \optim{\fv{\beta}}} = 0 &\implies C - \optim{\alpha_i^r} - \optim{\beta_i^r} = 0
% \end{align}

Using these results and substituting back in the Lagrangian we obtain
\begin{equation}\nonumber
    \begin{aligned}
         &\lagr(\optim{{w}}, \optim{\fv{v}_1}, \ldots, \optim{\fv{v}_\ntasks}, \optim{b}, \optim{d_1}, \ldots, \optim{d_\ntasks}, \optim{\fv{\xi}}, \fv{\alpha}) \\
        &\quad= \frac{C}{2} \frac{1}{C^2} \sum_{r=1}^\ntasks \sum_{i=1}^{\npertask_r} (\alpha_i^r)^2 \\
        &\qquad +  \frac{1}{2} \sum_{r= 1}^T{\norm{ (1 - \lambda_r) \sum_{i=1}^{m_r}  {\alpha_i^r} \left\lbrace y_i^r \phi_r(\fv{x}_i^r) \right\rbrace}^2} + \frac{1}{2} {\norm{\sum_{r= 1}^T \lambda_r \sum_{i=1}^{m_r} {\alpha_i^r} \left\lbrace y_i^r \phi(\fv{x}_i^r) \right\rbrace}}^2 \\
        &\qquad - \sum_{r= 1}^T \lambda_r \sum_{i=1}^{m_r} \alpha_i^r \left\lbrace y_{i}^r \left[ \left(\sum_{s= 1}^T \lambda_r \sum_{j=1}^{m_s} {\alpha_j^s} \left\lbrace y_j^s \phi(\fv{x}_j^s) \right\rbrace \right) \cdot \phi(\fv{x}_{i}^r) \right]  \right\rbrace \\
        &\qquad -  \sum_{r= 1}^T (1-\lambda_r) \sum_{i=1}^{m_r} \alpha_i^r \left\lbrace y_{i}^r \left[  \left((1-\lambda_r) \sum_{j=1}^{m_r} {\alpha_j^r} \left\lbrace y_j^r \phi_r(\fv{x}_j^r) \right\rbrace \right) \cdot \phi_r(\fv{x}_{i}^r)  \right]  \right\rbrace \\
        &\qquad -  \sum_{r= 1}^T \sum_{i=1}^{m_r} \alpha_i^r \left\lbrace - p_{i}^r  \right\rbrace \\
        &\quad=  \frac{1}{2} \sum_{r= 1}^T{\dotp{(1-\lambda_r)\sum_{i=1}^{m_r} {\alpha_i^r} \left\lbrace y_i^r \phi_r(\fv{x}_i^r) \right\rbrace}{(1-\lambda_r)\sum_{i=1}^{m_r} {\alpha_i^r} \left\lbrace y_i^r \phi_r(\fv{x}_i^r) \right\rbrace}} \\
        &\qquad + \frac{1}{2} {\dotp{\sum_{r= 1}^\ntasks \lambda_r \sum_{i=1}^{m_r} {\alpha_i^r} \left\lbrace y_i^r \phi(\fv{x}_i^r) \right\rbrace}{\sum_{r= 1}^\ntasks \lambda_r \sum_{i=1}^{m_r} {\alpha_i^r} \left\lbrace y_i^r \phi(\fv{x}_i^r) \right\rbrace}} \\
        &\qquad - \dotp{ \sum_{r= 1}^T \lambda_r \sum_{i=1}^{m_r} {\alpha_i^r} \left\lbrace y_i^r \phi(\fv{x}_i^r) \right\rbrace}{ \sum_{r= 1}^T \lambda_r \sum_{i=1}^{m_r} {\alpha_i^r} \left\lbrace y_i^r \phi(\fv{x}_i^r) \right\rbrace} \\
        &\qquad -  \sum_{r= 1}^T {\dotp{(1-\lambda_r) \sum_{i=1}^{m_r} {\alpha_i^r} \left\lbrace y_i^r \phi_r(\fv{x}_i^r) \right\rbrace}{(1-\lambda_r) \sum_{i=1}^{m_r} {\alpha_i^r} \left\lbrace y_i^r \phi_r(\fv{x}_i^r) \right\rbrace}} \\
        &\qquad -  \sum_{r= 1}^T \sum_{i=1}^{m_r} \alpha_i^r \left\lbrace - p_{i}^r  \right\rbrace \\
        &\quad= - \frac{1}{2} \dotp{ \sum_{r= 1}^T  \lambda_r \sum_{i=1}^{m_r} {\alpha_i^r} \left\lbrace y_i^r \phi(\fv{x}_i^r) \right\rbrace}{ \sum_{r= 1}^T \lambda_r \sum_{i=1}^{m_r} {\alpha_i^r} \left\lbrace y_i^r \phi(\fv{x}_i^r) \right\rbrace} \\
        &\qquad - \frac{1}{2} \sum_{r= 1}^T {(1-\lambda_r) \dotp{\sum_{i=1}^{m_r} {\alpha_i^r} \left\lbrace y_i^r \phi_r(\fv{x}_i^r) \right\rbrace}{(1-\lambda_r) \sum_{i=1}^{m_r} {\alpha_i^r} \left\lbrace y_i^r \phi_r(\fv{x}_i^r) \right\rbrace}} \\
        &\qquad +  \sum_{r= 1}^T \sum_{i=1}^{m_r} \alpha_i^r  p_{i}^r .
    \end{aligned}
\end{equation}
%   Dual
The corresponding dual problem can be then expressed as
\begin{equation}\label{eq:svmmtl_dual_conv_l2}
    \begin{aligned}
    & \min_{\alpha} & \Theta(\alpha) &= \frac{1}{2} \frac{1}{C} \sum_{r=1}^\ntasks \sum_{i=1}^{\npertask_r} (\alpha_i^r)^2 \\
    & & &\quad+\frac{1}{2} \dotp{\sum_{r= 1}^\ntasks \lambda_r \sum_{i=1}^{m_r} {\alpha_i^r} \left\lbrace y_i^r \phi(\fv{x}_i^r) \right\rbrace}{\sum_{r= 1}^\ntasks \lambda_r \sum_{i=1}^{m_r} {\alpha_i^r} \left\lbrace y_i^r \phi(\fv{x}_i^r) \right\rbrace} \\
    & & &\quad+  \frac{1}{2} \sum_{r= 1}^T {\dotp{(1 - \lambda_r)\sum_{i=1}^{m_r} {\alpha_i^r} \left\lbrace y_i^r \phi_r(\fv{x}_i^r) \right\rbrace}{(1 - \lambda_r) \sum_{i=1}^{m_r} {\alpha_i^r} \left\lbrace y_i^r \phi_r(\fv{x}_i^r) \right\rbrace}} \\
    & & &\quad - \sum_{r=1}^\ntasks \sum_{i=1}^{\npertask_r} \alpha_i^r p_i^r \\
    & \text{s.t.}
    & & 0 \leq \alpha_i^r ; \; i=1, \ldots, m_r; r=1, \ldots, \ntasks \\
    & & & \sum_{i=1}^{m_r} \alpha_i^r y_i^r = 0;  r=1, \ldots, \ntasks . \\
    \end{aligned}
\end{equation}
Using the kernel trick, we can write the dual problem using a vector formulation
\begin{equation}\label{eq:svmmtl_dualvec_conv_l2}
    \begin{aligned}
    & \min_{\alpha} && \Theta(\alpha) = \frac{1}{2} \fv{\alpha}^\intercal \left( \left\lbrace \Lambda \fm{Q} \Lambda + \left(\fm{I}_{\nsamples} - \Lambda \right) \fm{K} \left(\fm{I}_{\nsamples} - \Lambda \right) \right\rbrace + \frac{1}{C} \fm{I} \right) \fv{\alpha} - \fv{p} \fv{\alpha} \\
    & \text{s.t.}
    & & 0 \leq \alpha_i^r ; \; i=1, \ldots, m_r; r=1, \ldots, \ntasks \\
    & & & \sum_{i=1}^{m_r} \alpha_i^r y_i^r = 0;  r=1, \ldots, \ntasks . \\
    \end{aligned}
\end{equation}
As in the convex MTL L1-SVM, we are using the matrix
$$
\Lambda = \Diag(\lambda_1, \ldots, \lambda_\ntasks)
$$
and the $\nsamples \times \nsamples$ identity matrix $\fm{I}_{\nsamples}$.
%
Note that instead of having the box constraints for the dual coefficients $\alpha_i^r$ we add a diagonal term to the MTL kernel matrix  $$\widehat{\fm{Q}} = \Lambda \fm{Q} \Lambda + \left(\fm{I}_{\nsamples} - \Lambda \right) \fm{K} \left(\fm{I}_{\nsamples} - \Lambda \right),$$ so the matrix that is used is $\widehat{Q} + \frac{1}{C} \fm{I}_{\nsamples}$.

%   Discussion: Primal vs Dual
The difference between the convex MTL based on the L1-SVM and this one, based on the L2-SVM, can be seen in the primal formulation, where the square of the errors is penalized, and, therefore, no inequality constraint is needed for the $\xi_i^r$ variables. This is reflected in the dual problem, where there is no longer an upper bound for the dual coefficients, but a diagonal term is added to the kernel matrix, which acts as a soft constraint for the size of these dual coefficients.


\subsection{LS Support Vector Machine}
The LS-SVM~\citep{SuykensV99} is a variant of the standard SVM where the epsilon-insensitive loss is replaced by the squared loss in the case of regression problems, and for classification a regression is made for the  negative and positive class.
In this subsection I present directly an MTL approach using L2-SVMs with a convex formulation.

% Convex Approach
\subsubsection{{Convex} MTL LS-SVM}
% Additive Approach
%   Primal
The primal problem for the convex MTL L2-SVM, presented in~\cite{RuizAD21}, is
\begin{equation}\label{eq:svmmtl_primal_convex_ls}
    \begin{aligned}
    & \argmin_{w, {v}_r, \xi}
    & & {J({w}, {v}_r, \xi) = \frac{C}{2} \sum_{r= 1}^T \sum_{i=1}^{m_r} {\xi_{i}^r}^2 + \frac{1}{2} \sum_{r= 1}^T{\norm{{v}_r}^2} + \frac{1}{2} {\norm{{w}}}^2} \\
    & \text{s.t.}
    & & y_{i}^r \left(\lambda_r \left\lbrace {w} \cdot \phi(x_{i}^r) + b \right\rbrace + (1 - \lambda_r) \left\lbrace {v}_r \cdot \phi_r(x_{i}^r) + d_r \right\rbrace  \right) = p_{i}^r - \xi_{i}^r .  \\
    \end{aligned}
\end{equation}
The prediction model is again
\begin{equation}
    \nonumber
    h_r(\fv{x}) = \lambda_r \left\lbrace {w} \cdot \phi(x_{i}^r) + b \right\rbrace + (1 - \lambda_r) \left\lbrace {v}_r \cdot \phi_r(x_{i}^r) + d_r \right\rbrace
\end{equation}
for regression and 
\begin{equation}
    \nonumber
    h_r(\fv{x}) = \sign{ \left(\lambda_r \left\lbrace {w} \cdot \phi(x_{i}^r) + b \right\rbrace + (1 - \lambda_r) \left\lbrace {v}_r \cdot \phi_r(x_{i}^r) + d_r \right\rbrace \right)}
\end{equation}
for classification.
%

As in the standard variant, hyperparameter $C$ regulates the trade-off between the loss and the margin size of each task-specialized model $h_r$, while $\lambda_r$ indicates how specific or common these models are in the range of $[0, 1]$. With $\lambda_1, \ldots, \lambda_\ntasks=0$ we have independent models for each task and for $\lambda_1, \ldots, \lambda_\ntasks=1$ we have a common model for all tasks. 
 


%   Lagrangian/KKT
 To obtain the dual problem the Lagrangian of problem~\eqref{eq:svmmtl_primal_convex_ls}:
\begin{equation}\label{eq:svmmtl_lagrangian_convex_ls}
    \begin{aligned}
        & \lagr({w}, \fv{v}_1, \ldots, \fv{v}_\ntasks, b, d_1, \ldots, d_\ntasks, \fv{\xi}, \fv{\alpha}) \\
        &=  \frac{C}{2} \sum_{r= 1}^T \sum_{i=1}^{m_r} ({\xi_{i}^r})^2 + \frac{1}{2} \sum_{r= 1}^T{\norm{{v}_r}^2} + \frac{1}{2} {\norm{{w}}}^2 \\
        &\quad -  \sum_{r= 1}^T \sum_{i=1}^{m_r} \alpha_i^r \left\lbrace y_{i}^r  \left(\lambda_r \left\lbrace {w} \cdot \phi(x_{i}^r) + b \right\rbrace + (1 - \lambda_r) \left\lbrace {v}_r \cdot \phi_r(x_{i}^r) + d_r \right\rbrace  \right) - p_{i}^r + \xi_{i}^r  \right\rbrace ,
    \end{aligned}
\end{equation}
where $\alpha_i^r \geq \in reals$ are the Lagrange multipliers. Observe that, although the Lagrangian is identical to the one of the L2-SVM, the non-negativity condition is no longer required for the dual coefficients.  Again, $\fv{\xi}$ represents the vector $$(\xi_1^1, \ldots, \xi_{m_1}^1, \ldots, \xi_1^\ntasks, \ldots, \xi_{m_\ntasks}^\ntasks)^\intercal$$ and analogously for $\fv{\alpha}$. The dual objective function is defined as 
\begin{equation}\nonumber
    \begin{aligned}
          &  \min_{{w}, \fv{v}_1, \ldots, \fv{v}_\ntasks, b, d_1, \ldots, d_\ntasks, \fv{\xi}} \lagr({w}, \fv{v}_1, \ldots, \fv{v}_\ntasks, b, d_1, \ldots, d_\ntasks, \fv{\xi}, \fv{\alpha}) \\
         &\quad = \lagr(\optim{{w}}, \optim{\fv{v}_1}, \ldots, \optim{\fv{v}_\ntasks}, \optim{b}, \optim{d_1}, \ldots, \optim{d_\ntasks}, \optim{\fv{\xi}}, \fv{\alpha})
    \end{aligned}    
\end{equation}
The gradients with respect to the primal variables are
\begin{align}
    \grad_{{w}} \lagr \vert_{\optim{{w}}, \optim{\fv{v}_1}, \ldots, \optim{\fv{v}_\ntasks}, \optim{b}, \optim{d_1}, \ldots, \optim{d_\ntasks}, \optim{\fv{\xi}}, {\fv{\alpha}}} = 0  &\implies \optim{{w}} - \sum_{r= 1}^\ntasks \lambda_r \sum_{i=1}^{m_r} {\alpha_i^r} \left\lbrace y_i^r \phi(x_i^r) \right\rbrace = 0 , \label{eq:common_repr_conv_ls} \\
    \grad_{{v}_r} \lagr \vert_{\optim{{w}}, \optim{\fv{v}_1}, \ldots, \optim{\fv{v}_\ntasks}, \optim{b}, \optim{d_1}, \ldots, \optim{d_\ntasks}, \optim{\fv{\xi}}, {\fv{\alpha}}} = 0 &\implies \optim{{v}_r} - (1 - \lambda_r) \sum_{i=1}^{m_r} {\alpha_i^r} \left\lbrace y_i^r \phi_r(x_i^r) \right\rbrace = 0 , \label{eq:specific_repr_conv_ls} \\
    \grad_{b} \lagr \vert_{\optim{{w}}, \optim{\fv{v}_1}, \ldots, \optim{\fv{v}_\ntasks}, \optim{b}, \optim{d_1}, \ldots, \optim{d_\ntasks}, \optim{\fv{\xi}}, {\fv{\alpha}}} = 0  &\implies \sum_{r= 1}^T \sum_{i=1}^{m_r} {\alpha_i^r} y_i^r = 0 , \label{eq:specific_eqconstr_conv_ls}  \\
    \grad_{{d}_r} \lagr \vert_{\optim{{w}}, \optim{\fv{v}_1}, \ldots, \optim{\fv{v}_\ntasks}, \optim{b}, \optim{d_1}, \ldots, \optim{d_\ntasks}, \optim{\fv{\xi}}, {\fv{\alpha}}} = 0 &\implies \sum_{i=1}^{m_r} {\alpha_i^r} y_i^r = 0 , \label{eq:common_eqconstr_conv_ls} \\
    \grad_{\xi_i^r} \lagr \vert_{\optim{{w}}, \optim{\fv{v}_1}, \ldots, \optim{\fv{v}_\ntasks}, \optim{b}, \optim{d_1}, \ldots, \optim{d_\ntasks}, \optim{\fv{\xi}}, {\fv{\alpha}}} = 0 &\implies C \xi_i^r - {\alpha_i^r} = 0 . \label{eq:xi_feas_conv_ls}
\end{align}
Since all $\alpha_i^r \in \reals$ are feasible, we can also write the corresponding KKT conditions as
\begin{equation}
    \nonumber
    \diffp{\lagr}{\alpha_i^r} = 0 \implies y_{i}^r \left(\lambda_r \left\lbrace {w} \cdot \phi(x_{i}^r) + b \right\rbrace + (1 - \lambda_r) \left\lbrace {v}_r \cdot \phi_r(x_{i}^r) + d_r \right\rbrace  \right) = p_{i}^r - \xi_{i}^r  .
\end{equation}
Using Equations~\eqref{eq:common_repr_conv_ls} and~\eqref{eq:specific_repr_conv_ls} to substitute $w$ and $v_r$, and~\eqref{eq:xi_feas_conv_ls} to replace $\xi_i^r$, we can express these conditions as 
\begin{equation}
    \begin{aligned}
        &y_{i}^r \left(\lambda_r \left\lbrace \left\langle\sum_{s= 1}^\ntasks \lambda_s \sum_{j=1}^{m_s} {\alpha_j^s} y_j^s \phi(x_j^s) , \phi(x_{i}^r) \right\rangle + b \right\rbrace \right) \\
        &\qquad y_{i}^r \left( (1 - \lambda_r) \left\lbrace \left\langle (1 - \lambda_r) \sum_{j=1}^{m_r} {\alpha_j^r}  y_j^r \phi_r(x_j^r) , \phi_r(x_{i}^r) \right\rangle + d_r \right\rbrace  \right) = p_{i}^r - \frac{1}{C}\alpha_{i}^r  \\
        &\quad\implies  \left(\lambda_r \left\lbrace \sum_{s= 1}^\ntasks \lambda_s \sum_{j=1}^{m_s} {\alpha_j^s} y_{i}^r y_j^s k(x_j^s, x_{i}^r)  + b \right\rbrace \right) \\
        &\qquad  \left( (1 - \lambda_r) \left\lbrace (1 - \lambda_r) \sum_{j=1}^{m_r} {\alpha_j^r} y_{i}^r  y_j^r k_r(x_j^r, x_{i}^r)  + d_r \right\rbrace  \right) + \frac{1}{C}\alpha_{i}^r  = p_{i}^r 
    \end{aligned}
\end{equation}
The corresponding system of equations can be written as
\begin{equation}
    \nonumber
    \left\lbrace \Lambda \fm{Q} \Lambda + \left(\fm{I}_{\nsamples} - \Lambda \right) \fm{K} \left(\fm{I}_{\nsamples} - \Lambda \right) \right\rbrace \fv{\alpha} + \fv{d} + \frac{1}{C} \fm{I}_{\nsamples} \fv{\alpha}   = \fv{p} 
\end{equation}
These equations, alongside those corresponding to conditions~\eqref{eq:specific_eqconstr_conv_ls}
Using these results and substituting back in the Lagrangian we obtain
\begin{equation}\nonumber
    \begin{aligned}
         &\lagr(\optim{{w}}, \optim{\fv{v}_1}, \ldots, \optim{\fv{v}_\ntasks}, \optim{b}, \optim{d_1}, \ldots, \optim{d_\ntasks}, \optim{\fv{\xi}}, \fv{\alpha}) \\
        &\quad= \frac{C}{2} \frac{1}{C^2} \sum_{r=1}^\ntasks \sum_{i=1}^{\npertask_r} (\alpha_i^r)^2 \\
        &\qquad +  \frac{1}{2} \sum_{r= 1}^T{\norm{ (1 - \lambda_r) \sum_{i=1}^{m_r}  {\alpha_i^r} \left\lbrace y_i^r \phi_r(\fv{x}_i^r) \right\rbrace}^2} + \frac{1}{2} {\norm{\sum_{r= 1}^T \lambda_r \sum_{i=1}^{m_r} {\alpha_i^r} \left\lbrace y_i^r \phi(\fv{x}_i^r) \right\rbrace}}^2 \\
        &\qquad - \sum_{r= 1}^T \lambda_r \sum_{i=1}^{m_r} \alpha_i^r \left\lbrace y_{i}^r \left[ \left(\sum_{s= 1}^T \lambda_r \sum_{j=1}^{m_s} {\alpha_j^s} \left\lbrace y_j^s \phi(\fv{x}_j^s) \right\rbrace \right) \cdot \phi(\fv{x}_{i}^r) \right]  \right\rbrace \\
        &\qquad -  \sum_{r= 1}^T (1-\lambda_r) \sum_{i=1}^{m_r} \alpha_i^r \left\lbrace y_{i}^r \left[  \left((1-\lambda_r) \sum_{j=1}^{m_r} {\alpha_j^r} \left\lbrace y_j^r \phi_r(\fv{x}_j^r) \right\rbrace \right) \cdot \phi_r(\fv{x}_{i}^r)  \right]  \right\rbrace \\
        &\qquad -  \sum_{r= 1}^T \sum_{i=1}^{m_r} \alpha_i^r \left\lbrace - p_{i}^r  \right\rbrace \\
        &\quad=  \frac{1}{2} \sum_{r= 1}^T{\dotp{(1-\lambda_r)\sum_{i=1}^{m_r} {\alpha_i^r} \left\lbrace y_i^r \phi_r(\fv{x}_i^r) \right\rbrace}{(1-\lambda_r)\sum_{i=1}^{m_r} {\alpha_i^r} \left\lbrace y_i^r \phi_r(\fv{x}_i^r) \right\rbrace}} \\
        &\qquad + \frac{1}{2} {\dotp{\sum_{r= 1}^\ntasks \lambda_r \sum_{i=1}^{m_r} {\alpha_i^r} \left\lbrace y_i^r \phi(\fv{x}_i^r) \right\rbrace}{\sum_{r= 1}^\ntasks \lambda_r \sum_{i=1}^{m_r} {\alpha_i^r} \left\lbrace y_i^r \phi(\fv{x}_i^r) \right\rbrace}} \\
        &\qquad - \dotp{ \sum_{r= 1}^T \lambda_r \sum_{i=1}^{m_r} {\alpha_i^r} \left\lbrace y_i^r \phi(\fv{x}_i^r) \right\rbrace}{ \sum_{r= 1}^T \lambda_r \sum_{i=1}^{m_r} {\alpha_i^r} \left\lbrace y_i^r \phi(\fv{x}_i^r) \right\rbrace} \\
        &\qquad -  \sum_{r= 1}^T {\dotp{(1-\lambda_r) \sum_{i=1}^{m_r} {\alpha_i^r} \left\lbrace y_i^r \phi_r(\fv{x}_i^r) \right\rbrace}{(1-\lambda_r) \sum_{i=1}^{m_r} {\alpha_i^r} \left\lbrace y_i^r \phi_r(\fv{x}_i^r) \right\rbrace}} \\
        &\qquad -  \sum_{r= 1}^T \sum_{i=1}^{m_r} \alpha_i^r \left\lbrace - p_{i}^r  \right\rbrace \\
        &\quad= - \frac{1}{2} \dotp{ \sum_{r= 1}^T  \lambda_r \sum_{i=1}^{m_r} {\alpha_i^r} \left\lbrace y_i^r \phi(\fv{x}_i^r) \right\rbrace}{ \sum_{r= 1}^T \lambda_r \sum_{i=1}^{m_r} {\alpha_i^r} \left\lbrace y_i^r \phi(\fv{x}_i^r) \right\rbrace} \\
        &\qquad - \frac{1}{2} \sum_{r= 1}^T {(1-\lambda_r) \dotp{\sum_{i=1}^{m_r} {\alpha_i^r} \left\lbrace y_i^r \phi_r(\fv{x}_i^r) \right\rbrace}{(1-\lambda_r) \sum_{i=1}^{m_r} {\alpha_i^r} \left\lbrace y_i^r \phi_r(\fv{x}_i^r) \right\rbrace}} \\
        &\qquad +  \sum_{r= 1}^T \sum_{i=1}^{m_r} \alpha_i^r  p_{i}^r .
    \end{aligned}
\end{equation}
%   Dual
The corresponding dual problem can be then expressed as
\begin{equation}\label{eq:svmmtl_dual_conv_ls}
    \begin{aligned}
    & \min_{\alpha} & \Theta(\alpha) &= \frac{1}{2} \frac{1}{C} \sum_{r=1}^\ntasks \sum_{i=1}^{\npertask_r} (\alpha_i^r)^2 \\
    & & &\quad+\frac{1}{2} \dotp{\sum_{r= 1}^\ntasks \lambda_r \sum_{i=1}^{m_r} {\alpha_i^r} \left\lbrace y_i^r \phi(\fv{x}_i^r) \right\rbrace}{\sum_{r= 1}^\ntasks \lambda_r \sum_{i=1}^{m_r} {\alpha_i^r} \left\lbrace y_i^r \phi(\fv{x}_i^r) \right\rbrace} \\
    & & &\quad+  \frac{1}{2} \sum_{r= 1}^T {\dotp{(1 - \lambda_r)\sum_{i=1}^{m_r} {\alpha_i^r} \left\lbrace y_i^r \phi_r(\fv{x}_i^r) \right\rbrace}{(1 - \lambda_r) \sum_{i=1}^{m_r} {\alpha_i^r} \left\lbrace y_i^r \phi_r(\fv{x}_i^r) \right\rbrace}} \\
    & & &\quad - \sum_{r=1}^\ntasks \sum_{i=1}^{\npertask_r} \alpha_i^r p_i^r \\
    & \text{s.t.}
    & & 0 \leq \alpha_i^r ; \; i=1, \ldots, m_r; r=1, \ldots, \ntasks \\
    & & & \sum_{i=1}^{m_r} \alpha_i^r y_i^r = 0;  r=1, \ldots, \ntasks . \\
    \end{aligned}
\end{equation}
Using the kernel trick, we can write the dual problem using a vector formulation
\begin{equation}\label{eq:svmmtl_dualvec_conv_ls}
    \begin{aligned}
    & \min_{\alpha} && \Theta(\alpha) = \frac{1}{2} \fv{\alpha}^\intercal \left( \left\lbrace \Lambda \fm{Q} \Lambda + \left(\fm{I}_{\nsamples} - \Lambda \right) \fm{K} \left(\fm{I}_{\nsamples} - \Lambda \right) \right\rbrace + \frac{1}{C} \fm{I} \right) \fv{\alpha} - \fv{p} \fv{\alpha} \\
    & \text{s.t.}
    & & 0 \leq \alpha_i^r ; \; i=1, \ldots, m_r; r=1, \ldots, \ntasks \\
    & & & \sum_{i=1}^{m_r} \alpha_i^r y_i^r = 0;  r=1, \ldots, \ntasks . \\
    \end{aligned}
\end{equation}
As in the convex MTL L1-SVM, we are using the matrix
$$
\Lambda = \Diag(\lambda_1, \ldots, \lambda_\ntasks)
$$
and the $\nsamples \times \nsamples$ identity matrix $\fm{I}_{\nsamples}$.
%
Note that instead of having the box constraints for the dual coefficients $\alpha_i^r$ we add a diagonal term to the MTL kernel matrix  $$\widehat{\fm{Q}} = \Lambda \fm{Q} \Lambda + \left(\fm{I}_{\nsamples} - \Lambda \right) \fm{K} \left(\fm{I}_{\nsamples} - \Lambda \right),$$ so the matrix that is used is $\widehat{Q} + \frac{1}{C} \fm{I}_{\nsamples}$.

%   Discussion: Primal vs Dual
The difference between the convex MTL based on the L1-SVM and this one, based on the L2-SVM, can be seen in the primal formulation, where the square of the errors is penalized, and, therefore, no inequality constraint is needed for the $\xi_i^r$ variables. This is reflected in the dual problem, where there is no longer an upper bound for the dual coefficients, but a diagonal term is added to the kernel matrix, which acts as a soft constraint for the size of these dual coefficients.


\section{Optimal Convex Combination of fitted Models}
% A natural alternative to convexMTL is to direclty combine models



\section{Convex Multi-Task Learning with Neural Networks}

\section{Experiments}

\section{Conclusions}\label{sec-conclusions-3}

In this chapter, we have\dots
