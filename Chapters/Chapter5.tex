% Chapter 4

\chapter{Adaptive Graph Laplacian for Multi-Task Learning} % Write in your own chapter title
\label{Chapter5}
\lhead{Chapter \ref{Chapter5}. 
\emph{Adaptive Graph Laplacian Multi-Task Support Vector Machine}} % Write in your own chapter title to set the page header

{\bf \small{

}}

\section{Introduction}
In Chapter~\ref{Chapter3} the MTL strategies are divided into feature-based, parameter-based and joint learning approaches.
The feature-based approaches, which try to find a representation shared by all tasks to obtain leverage in the learning process, rely on the assumption that all tasks can indeed share a common latent representation.
In the case of joint learning, which combines a common part and task-specific parts in the models, a similar belief is held. Although this approach has many good properties, it relies on the assumption that all tasks can share the same common information, which is captured by the common part of the model.
However, this might not be the case in some MTL scenarios, where there can exist groups of tasks that share some information, but are unrelated to the rest.
The parameter-based approaches are usually reliant on enforcing low-rank matrices, thus, assuming that all tasks parameters belong to the same subspace; however, the graph Laplacian based approaches offer a different perspective.

In the graph Laplacian strategies, the idea is based on penalizing the distance between the parameters of different tasks. It is more natural for linear or kernel approaches, where the models for each task $r=1, \ldots, \ntasks$ are defined as
\begin{equation}
    \nonumber
    f_r(x) = w_r \cdot \phi({x}) + b_r
\end{equation}
where $\phi(x)$ is a transformation, which can be the identity $\phi(x)=x$ in linear models, or an implicit transformation to an \acrshort{rkhs} in kernel models.
Here, the models are defined by the parameter $w_r$, so pushing together these parameters enforces the models to be similar. 
The idea is to assume that the relationship between tasks can be modelled using a graph, then the adjacency matrix $\fm{A}$ of such graph is used to define the regularization
\begin{equation}
    \label{eq:gl_regularization}
    \sum_{r=1}^\ntasks \sum_{s=1}^\ntasks A_{rs} \norm{w_r - w_s}^2
\end{equation}
where $A_{rs}$ are positive scalars that weight the pairwise distances.
Although the meaning of~\eqref{eq:gl_regularization} is clear for the linear case, it is not that direct in the case of kernel models where, as shown in the Representer Theorem, the optimal parameters $w_r^*$ are elements of an \acrshort{rkhs}.
In this Chapter, a framework to use the graph Laplacian regularization with kernel models is presented. This is implemented with L1, L2 and LS-SVMs. Moreover, the graph Laplacian strategy is combined with the convex joint learning one presented in Chapter~\ref{Chapter3}.
%

Besides, the definition of the weights $A_{rs}$ is not trivial, and it is crucial for the good performance of this strategy. 
First, the values $A_{rs}$ have to be bounded, otherwise its interpretability is lost and, moreover, one term can dominate the sum, so only two tasks models would be enforced to be similar.
%
Even with bounded weights, in absence of expert knowledge to select them, it is necessary to find a procedure that finds a set of weights $A_{rs}$ that reflects the real relations between tasks.
In this chapter, a data-driven procedure to select the weights for a Laplacian regularization is presented.


\section{Graph Laplacian Multi-Task Learning with Kernel Methods}
\subsection{Linear Case and Kernel Extension}
\paragraph*{Linear Case.\\}
Observe that we can express the Laplacian regularization as
\begin{equation}
    \nonumber
    \Omega(w_1, \ldots, w_\ntasks) = \sum_{r=1}^T \sum_{s=1}^T A_{rs} \norm{w_r - w_s}^2 =  \sum_{r=1}^T \sum_{s=1}^T A_{rs} \left\{ \norm{w_r}^2 + \norm{w_s}^2 - 2 \langle w_r, w_s \rangle \right\}.
\end{equation}
Here, only the distance between model parameters is penalized, and in the extreme case where all the tasks can use the same model, i.e. $w_r = w$, there would be no regularization for the complexity of such model $w$.
This problem can be solved by adding the individual model regularization aside from the Laplacian one.
To illustrate this, we first use a linear L1-SVM, whose primal problem is
\begin{equation}\label{eq:linear_gl_primal}
\begin{aligned}
& \argmin_{\substack{w_1, \ldots, w_\ntasks ; \\ \substack{b_1, \ldots, b_\ntasks } ; \\ \substack{\fv{\xi}}; }}
& & {\sum_{r=1}^T C_r \sum_{i=1}^m{\xi_{i}^r} + \frac{\nu}{4} \sum_{r=1}^T \sum_{s=1}^T A_{rs} \norm{w_r - w_s}^2 + \frac{1}{2} \sum_r \norm{w_r}^2 }\\
& \text{s.t.}
& & y_{i}^r ( w_r \cdot x_{i}^r + b_r) \geq p_{i}^r - \xi_{i}^r , \;  i=1,\ldots,m_r; \;  r=1,\ldots,T,\\
& & & \xi_{i}^r \geq 0, \;  i=1,\ldots,m_r; \;  r=1,\ldots,T \; .
\end{aligned}
\end{equation}
The corresponding Lagrangian is
\begin{equation}\label{eq:svmmtl_lagr_GL_addreg}
\begin{aligned}
        \mathcal{L}&(w_1, \ldots, w_\ntasks, b_r, \fv{\alpha}, \fv{\beta}) = C \sum_{r=1}^T \sum_{i=1}^{m_r}{\xi_{i}^r} + \frac{\nu}{2} \sum_{r=1}^T \sum_{s=1}^T A_{rs} \norm{w_r - w_s}^2 + \frac{1}{2} \sum_r \norm{w_r}^2 \\
        & - \sum_{r=1}^T \sum_{i=1}^{m_r}{ \alpha_i^r [y_{i}^r (w_r \cdot x_{i}^r + b_r) - p_{i}^r + \xi_{i}^r]   } - \sum_{r=1}^T \sum_{i=1}^{m_r}{ \beta_i^r \xi_i^r } \; .
\end{aligned}
\end{equation}
And the derivatives of the Lagrangian with respect the primal variables are
\begin{align}
& \frac{\partial \mathcal{L}}{\partial w_r} = 0 \implies  w_r^* + \frac{\nu}{2} \sum_{s=1}^T (A_{rs} + A_{sr}) (w_r^* - w_s^*)= \sum_{i=1}^{m_r}{\alpha_i^r y_i^r x_i^r} \label{eq:partial_w_r_addreg} \; , \\
& \frac{\partial \mathcal{L}}{\partial b_r} = 0 \implies  \sum_{i=1}^{m_r}{\alpha_i^r y_i^r } = 0 \label{eq:partial_b_r_addreg} \; ,\\
& \frac{\partial \mathcal{L}}{\partial \xi_i^r} = 0 \implies C_r - \alpha_i^r - \beta_i^r = 0 \; \label{eq:partial_xi_addreg}\; .
\end{align}
Consider the following definitions of vectors
\begin{equation}
    \nonumber
    \underset{1 \times Td}{\fv{w}^\intercal} = (w_1^\intercal \ldots w_T^\intercal)
    , \; 
    \underset{1 \times (\sum_r m_r)}{\fv{\alpha}^T} = (\fv{\alpha}_1^\intercal \ldots \fv{\alpha}_T^\intercal)
    , \; 
    \underset{1 \times m_r}{\fv{\alpha}_r^\intercal} =  (\alpha_1^r \ldots \alpha_{m_r}^r) ;
\end{equation}
and consider also the matrices
\begin{equation*}
    \underset{\ntasks \dimx \times \ntasks \dimx}{E} = \left\lbrace I_T + \nu L \right\rbrace \otimes I_d, \;
    \underset{\ntasks \times \ntasks}{L} =
    \begin{bmatrix}
        \sum_{s \neq 1} A_{1s} & - A_{12} & \ldots & - A_{1T} \\
        \vdots & \vdots & \ddots & \vdots \\
        A_{\ntasks 1} & A_{\ntasks 2} & \ldots & \sum_{s \neq \ntasks} A_{\ntasks s}
    \end{bmatrix}
\end{equation*}
and
\begin{equation*}
    \underset{(\sum_r m_r) \times Td}{\Phi} =
    \begin{bmatrix}
        X_1 & 0 & \ldots & 0 \\
        0 & X_2 & \ldots & 0 \\
        \vdots & \vdots & \ddots & \vdots \\
        0 & 0 & \ldots & X_T
    \end{bmatrix} .
\end{equation*}
where $\fm{X}_r$ is the matrix with examples from task $r$.
Then, we can write~\eqref{eq:partial_w_r_addreg} with a matrix formulation as
\begin{equation}
    \nonumber
    E \fv{w} = \Phi^\intercal \fv{\alpha} \implies \fv{w} = E^{-1} \Phi^\intercal \fv{\alpha}
\end{equation}
With these definitions and using the derivatives to substitute in the Lagrangian, the result is
\begin{align*}
    \mathcal{L}(\fv{w}, \fv{\alpha}) &= \frac{1}{2} \fv{w}^\intercal E \fv{w} - \fv{\alpha}^\intercal \Phi \fv{w} + p^\intercal \fv{\alpha} \\
    &= \frac{1}{2} (E^{-1} \Phi^\intercal \fv{\alpha})^\intercal E E^{-1} \Phi^\intercal \fv{\alpha} - \fv{\alpha}^\intercal \Phi E^{-1} \Phi^\intercal \fv{\alpha} + p^\intercal \fv{\alpha} \\
    &= \frac{1}{2} \fv{\alpha}^\intercal \Phi ({E^\intercal})^{-1} \Phi^\intercal \fv{\alpha}  - \fv{\alpha}^\intercal \Phi E^{-1} \Phi^\intercal \fv{\alpha} + p^\intercal \fv{\alpha} \\
    &= -\frac{1}{2}  \fv{\alpha}^\intercal \Phi E^{-1} \Phi^\intercal \fv{\alpha} + p^\intercal \fv{\alpha}.
\end{align*}
Here, the block structure of $\Phi$ and $E$ allow to write the following result 
\begin{equation}
    \label{eq:laplacian_block_linear}
    \Phi E^{-1} \Phi^\intercal = 
    \begin{bmatrix}
        E^{-1}_{11} X_1 X_1^\intercal & E^{-1}_{12} X_1 X_2^\intercal & \ldots & E^{-1}_{1\ntasks} X_1 X_\ntasks^\intercal \\
        E^{-1}_{21} X_2 X_1^\intercal & E^{-1}_{22} X_2 X_2^\intercal & \ldots & E^{-1}_{2\ntasks} X_2 X_\ntasks^\intercal \\
        \vdots & \vdots & \ddots & \vdots \\
        E^{-1}_{\ntasks1} X_\ntasks X_1^\intercal & E^{-1}_{\ntasks 2} X_\ntasks X_2^\intercal & \ldots & E^{-1}_{\ntasks \ntasks} X_\ntasks X_\ntasks^\intercal \\
    \end{bmatrix} .
\end{equation}
That is, $\Phi E^{-1} \Phi^\intercal = \widetilde{Q}$, where $ \widetilde{Q}$ is the kernel matrix corresponding to the kernel function 
\begin{equation}
    \label{eq:kernelfun_gl}
    \widetilde{k}(x_i^r, y_j^s) =  (I_T + \nu L)^{-1}_{rs} \dotp{x_i^r}{x_j^s}.
\end{equation}
Here, the individual regularization of the primal problem leads to the diagonal term added, which ensures the invertibility of the matrix $(I_T + \nu L)$.
Thus, this additional regularization ultimately makes the solution of the problem more numerically stable.
Using this result, the corresponding dual problem is
\begin{equation}\label{eq:linear_gl_dual}
    \begin{aligned}
        & \argmin_{\fv{\alpha}} 
        & & \Theta(\fv{\alpha}) = \frac{1}{2} \fv{\alpha}^t \widetilde{Q} \fv{\alpha} - p \fv{\alpha} \\
        & \text{s.t.}
        & & 0 \leq \alpha_i^r \leq C, \;  i=1,\ldots,m_r; r=1,\ldots,T\; \text{(box constraints)} \; , \\
        & & & \sum_{i=1}^{n_r}{\alpha_i^r y_i^r} = 0, \; r=1,\ldots,T\; \text{(equality constraints)} \; .
        \end{aligned}
\end{equation}


\paragraph*{Non-Linear Case.\\}
When we use an implicit kernel transformation, the result~\eqref{eq:laplacian_block_linear} is not direct. However, we can use the result from Lemma~\ref{lemma:regproblems_kernel}.
Observe that
\begin{align*}
    \myvec{v}^\intercal (I_T \otimes I_d) \myvec{v} &= \sum_{r=1}^T \norm{v_r}^2 , \\
    \myvec{v}^\intercal (L \otimes I_d) \myvec{v} &= \frac{1}{2} \sum_{r=1}^T \sum_{s=1}^T A_{rs} \norm{v_r - v_s}^2 .
\end{align*}
To check the second equation, see
\begin{align*}
    \myvec{v} (\mymat{L} \otimes \mymat{I}_d) \myvec{v} &= \myvec{v} (\mymat{D} \otimes \mymat{I}_d) \myvec{v} - \myvec{v} (\mymat{A} \otimes \mymat{I}_d) \myvec{v} \\
    &= \sum_{r=1}^\ntasks \sum_{s=1}^\ntasks D_{rs} v_r^\intercal v_s - \sum_{r=1}^\ntasks \sum_{s=1}^\ntasks A_{rs} v_r^\intercal v_s \\
    &= \sum_{r=1}^\ntasks D_{rr} v_r^\intercal v_r - \sum_{r=1}^\ntasks \sum_{s=1}^\ntasks A_{rs} v_r^\intercal v_s \\
    &= \sum_{r=1}^\ntasks \sum_{s=1}^\ntasks A_{rs} v_r^\intercal v_r - \sum_{r=1}^\ntasks \sum_{s=1}^\ntasks A_{rs} v_r^\intercal v_s \\
    &= \sum_{r=1}^\ntasks \sum_{s=1}^\ntasks A_{rs} (v_r^\intercal v_r - v_r^\intercal v_s)  \\
    &= \sum_{r=1}^\ntasks \sum_{s=r}^\ntasks \lbrace A_{rs}  (v_r^\intercal v_r - v_r^\intercal v_s) + A_{sr} (v_s^\intercal v_s - v_s^\intercal v_r) \rbrace \\
    &=
    \sum_{r=1}^\ntasks \sum_{s=r}^\ntasks \lbrace (A_{rs} + A_{sr})  (v_r^\intercal v_r + v_s^\intercal v_s - 2v_r^\intercal v_s) \rbrace \\
    &=
    \sum_{r=1}^\ntasks \sum_{s=r}^\ntasks \lbrace (A_{rs} + A_{sr})  \norm{v_r - v_s}^2 \rbrace \\
    &=
    \frac{1}{2} \sum_{r=1}^\ntasks \sum_{s=1}^\ntasks \lbrace (A_{rs} + A_{sr})  \norm{v_r - v_s}^2 \rbrace \\
    &=
     \sum_{r=1}^\ntasks \sum_{s=1}^\ntasks \lbrace A_{rs}  \norm{v_r - v_s}^2 \rbrace .\\
\end{align*}
where
$$ \myvec{v} = \sum_{t=1}^T e_r \otimes v_r ,$$
and $e_1, \ldots, e_\ntasks$ are the canonical base of $\reals^\ntasks$.
Then, we can write a general primal problem with Laplacian regularization using kernels as 
\begin{equation}
    \begin{aligned}
        &R(\myvec{v}) = C \sum_{r=1}^{\ntasks} \sum_{i=1}^\npertask \lossf(y_i^r, \dotp{\myvec{v}}{e_r \otimes \phi(x_i^r)}) + \left(  \myvec{v}^\intercal ((\nu L + I_\ntasks ) \otimes I) \myvec{v} \right),\\
    \end{aligned}
\end{equation}
As shown in Lemma~\ref{lemma:regproblems_kernel}, this is equivalent to solving a dual problem where the kernel function is
\begin{equation}
    \nonumber
    \widetilde{k}(x_i^r, x_j^s) = \left( \nu \fm{L} + \fm{I}_\ntasks \right)^{-1}_{rs} k(x_i^r, x_j^s)
\end{equation}
where $k(\cdot, \cdot)$ is the reproducing kernel induced by the implicit transformation $\phi(\cdot)$.
One example, using the L1-SVM would be the primal problem shown in~\eqref{eq:linear_gl_primal} where an implicit transformation $\phi(\cdot)$ is used.
The corresponding dual problem is, therefore, the problem shown in~\eqref{eq:linear_gl_dual}, but where the kernel matrix $\widetilde{Q}$ is defined using the kernel function~\eqref{eq:kernelfun_gl}.
% If $A$ is symmetric, that is $A_{rs} = A_{sr}$, then
% \begin{align*}
%     \sum_{r=1}^\ntasks \sum_{s=1}^\ntasks A_{rs} (v_r^\intercal v_r - v_r^\intercal v_s) &=
%     \sum_{r=1}^\ntasks \sum_{s=r}^\ntasks \lbrace A_{rs}  (v_r^\intercal v_r - v_r^\intercal v_s) + A_{sr} (v_s^\intercal v_s - v_s^\intercal v_r) \rbrace \\
%     &=
%     \sum_{r=1}^\ntasks \sum_{s=r}^\ntasks \lbrace (A_{rs} + A_{sr})  (v_r^\intercal v_r + v_s^\intercal v_s - 2v_r^\intercal v_s) \rbrace \\
%     &=
%     \sum_{r=1}^\ntasks \sum_{s=r}^\ntasks \lbrace (A_{rs} + A_{sr})  \norm{w_r - w_s}^2 \rbrace \\
%     &=
%     \frac{1}{2} \sum_{r=1}^\ntasks \sum_{s=1}^\ntasks \lbrace (A_{rs} + A_{sr})  \norm{w_r - w_s}^2 \rbrace \\
%     &=
%      \sum_{r=1}^\ntasks \sum_{s=1}^\ntasks \lbrace A_{rs}  \norm{w_r - w_s}^2 \rbrace .\\
% \end{align*}



\subsection{Convex Graph Laplacian MTL}



In~\cite{RuizAD21_hais} we proposed a convex formulation of the Graph Laplacian MTL SVM which includes a convex combination of a common part and task-specific parts that are coupled through a Laplacian regularization.
That is, the models for each task are
\begin{equation}
    \nonumber
    h_r(x) = \lambda_r w \cdot \phi(x) + \lambda_r v_r \cdot \psi(x) ,
\end{equation}
where $\phi(\cdot)$ and $\psi(\cdot)$ are the implicit transformations for the common part and task-specific parts, and can be possibly distinct, trying to capture different properties of the data.
That is, $\phi: \Xspace \to \hilbertspace_\phi$ and $\psi: \Xspace \to \hilbertspace_\psi$, where $\hilbertspace_\phi$ and $\hilbertspace_\psi$ are RKHS's with reproducing kernels $k_\phi(\cdot, \cdot)$ and $k_\psi(\cdot, \cdot)$, respectively.
Unlike the model definition for Convex MTL in~\eqref{eq:convexmtl_modeldef}, where each task-specific part can use a different Hilbert space, here all tasks must use the same two transformations. The common $\phi(\cdot)$ must be obviously equal for all tasks, but also the specific one $\psi(\cdot)$ must be the same for all tasks because to impose a Laplacian regularization, all the parameters $v_r$ have to be elements from the same space.
Again, as with the Convex MTL approach, parameters $\lambda_r \in [0, 1]$ define how relevant the common part for each task. When $\lambda=1$, only the common part is present, while $\lambda=0$ results in task-specific models that, now, are coupled through the Laplacian regularization.

\paragraph*{A general result for Convex Graph Laplacian MTL.\\}
In general, we can use the Representer Theorem with the tensor kernels defined in Chapter~\ref{Chapter3} to find the result of kernel problems using this approach.
Using the same approach used in Section~\ref{sec:ch3_mtl_kernelmethods}, we define
$$ \fv{v} = \sum_{t=1}^T e_r \otimes v_r \in \reals^\ntasks \otimes \hilbertspace_\psi, $$
such that
$\dotp{\fv{v}}{e_r \otimes \psi(x_i^r)} = \dotp{v_r}{\psi(x_i^r)}$.
If we consider then the product space $\hilbertspace_\phi \times (\reals^\ntasks \otimes \hilbertspace_\psi)$, where the norm of $(w, \fv{v})$ is 
$$\dotp{(w, \fv{v})}{(w, \fv{v})} =  \dotp{w}{w} + \dotp{\fv{v}}{\fv{v}} = \norm{w}^2 + \norm{\fv{v}^2},$$
then a general kernelized problem for convex graph Laplacian MTL can be expressed as
\begin{equation}\label{eq:cvxgl_general_problem}
    \begin{aligned}
    \argmin_{(w, \fv{v}) \in \hilbertspace_\phi \times (\reals^\ntasks \otimes \hilbertspace_\psi)} R((w, \fv{v})) &= \sum_{r=1}^\ntasks \sum_{i=1}^{\npertask_r} \ell(y_i^r, (\dotp{(w, \fv{v})}{(\lambda_r \phi(x_i^r), (1 - \lambda_r) (e_r \otimes \psi({x}_i^r)))}))\\
    &\quad + \dotp{(w, \fv{v})}{(I_{\rkhs_\phi} \times (E \otimes I_{\rkhs_\psi})) (w, \fv{v})}  ,
    \end{aligned}
\end{equation}
where $\ell$ is a loss function, $\rkhs_\phi$, in our case it is the identity operator in $\rkhs_\phi$, and $E$ is a positive definite operator in $\reals^\ntasks$ in our case, $E = ((\nu \fm{L} + I))$, with $L$ being a graph Laplacian matrix.
% With both $A$ and $E$ are positive definite, the function $ \dotp{(w, \fv{v})}{(A \times E) (w, \fv{v})}$ is strictly increasing in $\norm{(w, \fv{v})}$; then, we can apply the Representer Theorem to problem~\eqref{eq:cvxgl_general_problem}, that is, the solutions of this problem can be expressed as
% \begin{equation}
%     \nonumber
%     (w^*, \fv{v}^*) = \sum_{r=1}^\ntasks \sum_{i=1}^\npertask \alpha_i^r y_i^r (\lambda_r \phi(x_i^r), (1 - \lambda_r) (e_r \otimes \psi({x}_i^r)))
% \end{equation}
Using a modification of Lemma~\ref{lemma:regproblems_kernel}, we can state the following one 
\begin{lemma}\label{lemma:regproblems_kernel_convex}
    The predictions $\dotp{(w^*, \fv{v}^*)}{(\phi(x_i^r), e_r \otimes \psi(x_i^r))}$ of the solution $(w^*, \fv{v}^*)$ from the Multi-Task optimization problem~\eqref{eq:cvxgl_general_problem} can be obtained solving the problem
    \begin{equation}
        \label{eq:cvxgl_general_problem_alt}
        \begin{aligned}
            &S((w, \fv{u})) = \sum_{r=1}^{\ntasks} \sum_{i=1}^{\npertask_r} \lossf(y_i^r, \dotp{(w, \fv{u})}{(\lambda_r \phi(x_i^r), (1 - \lambda_r) (B_r \otimes \psi(x_i^r)))}) + \mu \left( \dotp{w}{w} + \dotp{\fv{u}}{\fv{u}}\right),\\
        \end{aligned}
    \end{equation}
    where $w \in \rkhs_\phi$ and $\fv{u} \in \reals^p \otimes \hilbertspace_\psi$ with $p \geq \ntasks$ and $B_r$ are the columns of a full rank matrix $B \in \reals^{p \times \ntasks}$ such that $\mymat{E}^{-1} = \mymat{B}^\intercal \mymat{B}$.
\end{lemma}
\begin{proof}
Again, since ${E} \in \reals^{\ntasks \times \ntasks}$ is positive definite, we can find $B \in \reals^{p \times \ntasks}, p \geq \ntasks$ and $\rank{B} = \ntasks$ such that $E^{-1} = 
{B^\intercal} {B}$, using for example the SVD;
% In the case that ${E} \in \reals^{\ntasks \times \ntasks}$ is a semipositive definite matrix with rank $r$ we can find $B \in \reals^{p \times r}, p \geq \ntasks$ and $\rank{B} = r$ such that $E^{+} = 
% {B^\intercal} {B}$.
% For the rest of the analysis we will consider a positive definite matrix $\mymat{E}$ but the results are also valid for positive semidefinite matrices.
% 
then,  
$$ E^{-1} \otimes I = (B^\intercal B) \otimes I = (B^\intercal \otimes I_{\rkhs_\psi}) (B \otimes I),$$
%
Consider the change of variable $\fv{v} = (B^\intercal \otimes I_{\rkhs_\psi}) \fv{u}$, where $\fv{u} \in \reals^p \otimes \hilbertspace$, which can always be done because 
%the columns of $B$ generates $\reals^T$, so we can always find $\hat{w}$ such that $e_r = B w$   .The condition of full rank for $B$ is necessary in this step.
$B$ is full rank, then
\begin{equation}
    \nonumber
    \begin{aligned}
        R(w, \fv{u}) &= \sum_{r=1}^{\ntasks} \sum_{i=1}^{\npertask_r} \lossf(y_i^r, \dotp{(w, (B^\intercal \otimes I_{\rkhs_\psi}) \fv{v})}{ (\lambda_r \phi(x_i^r), (1 - \lambda_r) (e_r \otimes \psi(x_i^r)))}) \\
        &\quad + \mu \dotp{(w, (B^\intercal \otimes I_{\rkhs_\psi})\fv{v})}{(I_{\rkhs_\phi} \times (E \otimes I_{\rkhs_\psi})) (w, (B^\intercal \otimes I_{\rkhs_\psi}) \fv{v}) } \\
    \end{aligned}
\end{equation}
This is equivalent to problem~\eqref{eq:cvxgl_general_problem_alt}, where the regularizer is increasing in $\norm{(w, \fv{u})}^2$, so can apply the Representer theorem, which states that the minimizer of $S(w, \fv{u})$ has the form
\begin{equation}
    \nonumber
    %\label{eq:representer_tensor}
    (w^*, \fv{u}^*) = \sum_{r=1}^\ntasks \sum_{i=1}^{\npertask_r} \alpha_i^r (\lambda_r \phi(x_i^r), (1-\lambda_r) (B_r \otimes \psi(x_i^r))) ,
\end{equation}
Using the correspondence between $\fv{u}^*$ and $\fv{v}^*$, 
\begin{equation}\nonumber
    \begin{aligned}
        (w^*, \fv{v}^*) 
         &= (w^*, (B^\intercal \otimes I_{\rkhs_\psi}) \fv{u}^*) \\
         &=  \sum_{r=1}^\ntasks \sum_{i=1}^{\npertask_r} \alpha_i^r (\lambda_r \phi(x_i^r), (1 - \lambda_r)\vect{(\dotp{B_1}{B_r}, \ldots, \dotp{B_\ntasks}{B_r})} \otimes \psi(x_i^r)).
    \end{aligned}
\end{equation}
Then, we can recover the predictions corresponding to the solutions $(w^*, \fv{v}^*)$ as
\begin{equation}
    \nonumber
    \begin{aligned}
        &\dotp{(w^*, \fv{v}^*)}{(\lambda_t \phi(\hat{x}^t), (1 - \lambda_t) e_t \otimes \psi(\hat{x}^t))} \\
        &= \dotp{\sum_{r=1}^\ntasks \sum_{i=1}^{\npertask_r} \alpha_i^r (w, \vect{(\dotp{B_1}{B_r}, \ldots, \dotp{B_\ntasks}{B_r})} \otimes \phi(x_i^r))}{(\phi(\hat{x}^t), e_t \otimes \psi(\hat{x}^t))} \\
        &= \sum_{r=1}^\ntasks \sum_{i=1}^{\npertask_r} \alpha_i^r  (\lambda_r \lambda_t \dotp{\phi(x_i^r)}{\phi(x_i^r)} + (1-\lambda_r) (1 - \lambda_t) \dotp{B_s}{B_r} \dotp{\psi(x_i^r)}{\psi(\hat{x}^t)}) \\
        &= \sum_{r=1}^\ntasks \sum_{i=1}^{\npertask_r} \alpha_i^r  (\lambda_r \lambda_t \dotp{\phi(x_i^r)}{\phi(x_i^r)} + (1-\lambda_r) (1 - \lambda_t) E_{rs}^{-1} \dotp{\psi(x_i^r)}{\psi(\hat{x}^t)}) .
    \end{aligned}
\end{equation}
\end{proof}
As with Lemma~\ref{lemma:regproblems_kernel}, the Multi-Task problem~\ref{eq:cvxgl_general_problem} can be expressed as a Single-Task problem with the Multi-Task kernel function
\begin{equation}
    \label{eq:dual_cvxgl_kernel_function}
    \widehat{\widetilde{k}}(x_i^r, x_j^s) = \lambda_r \lambda_s k_\phi(x_i^r, x_j^s) + (1 - \lambda_r) (1 - \lambda_s) (\nu \fm{L} + \fm{I}_\ntasks)^{-1}_{rs} k_\psi(x_i^r, x_j^s) ,
\end{equation}
where $k_\phi(\cdot, \cdot)$ and $k_\psi(\cdot, \cdot)$ are the reproducing kernels corresponding to transformations $\phi$ and $\psi$, respectively.
%



\paragraph*{Convex Graph Laplacian L1-SVM.\\}
%
The primal problem for the linear L1-SVM using this approach, as shown in~\citep*{RuizAD21_hais}, is the following one
%
\begin{equation}\label{eq:primal_cvxgl_l1_linear}
  \begin{aligned}
  & \argmin_{\substack{v_1, \ldots, v_\ntasks ; \\ \substack{b_1, \ldots, b_\ntasks } ; \\ \substack{\fv{\xi}}, w; }}
  & & { C \sum_{r=1}^\ntasks \sum_{i=1}^{\npertask_r} {\xi_i^r}  + \frac{\nu}{2} \sum_{r=1}^\ntasks \sum_{s=1}^T A_{rs} {\| {v}_r - {v}_s \|}^2 + \frac{1}{2} \sum_r \norm{{v}_r}^2 + \frac{1}{2} \norm{{w}}^2} \\
  & \text{s.t.}
  & & y_i^r (\lambda_r ({w} \cdot {x}_i^r) + (1 - \lambda_r) ({v}_r \cdot {x}_i^r) + b_r) \geq p_i^r - \xi_i^r  ,\\
  & & & \xi_i^r \geq 0,  \;  i = 1, \dotsc, \npertask_r, \; r=1, \dotsc, \ntasks .
  \end{aligned}
\end{equation}
%
Note that with $\mu=0$, this problem is equivalent to the one shown in~\eqref{eq:svmmtl_primal_convex}.
%
Extending this problem to the kernel case requires using a different formulation, that is
\begin{equation}\label{eq:primal_cvxgl_l1_kernel}
    \begin{aligned}
    & \argmin_{\substack{\fv{v} ; \\ \substack{b_1, \ldots, b_\ntasks } ; \\ \substack{\fv{\xi}}, w; }}
    & & { C \sum_{r=1}^\ntasks \sum_{i=1}^{\npertask_r} {\xi_i^r}  + \frac{\nu}{2} \fv{v}^\intercal (\fm{L} \otimes I_\rkhs) \fv{v} + \frac{1}{2} \fv{v}^\intercal (\fm{I}_\ntasks \otimes I_\rkhs) \fv{v} + \frac{1}{2} \norm{{w}}^2} \\
    & \text{s.t.}
    & & y_i^r (\lambda_r \dotp{w}{\phi(x_i^r)} + (1 - \lambda_r) (\dotp{\fv{v}}{e_r \otimes \psi({x}_i^r)}) + b_r) \geq p_i^r - \xi_i^r  ,\\
    & & & \xi_i^r \geq 0,  \;  i = 1, \dotsc, \npertask_r, \; r=1, \dotsc, \ntasks .
    \end{aligned}
  \end{equation}
%
Although the result from Lemma~\ref{lemma:regproblems_kernel_convex} can be applied for this problem, for illustration purposes the whole procedure is shown for this particular case.
%
The Lagrangian corresponding to~\eqref{eq:primal_cvxgl_l1_kernel} is
\begin{equation}\label{eq:lagr_cvxgl_l1_kernel}
\begin{aligned}
        \mathcal{L}&({w}, \fv{v}, b_r, \xi_i^r, {\fv{\alpha}}, \fv{\beta}) \\
        &= C \sum_{r=1}^T \sum_{i=1}^{\npertask_r}{\xi_{i}^r} + \frac{\nu}{2} \fv{v}^\intercal (\fm{L} \otimes I_\rkhs) \fv{v} + \frac{1}{2} \fv{v}^\intercal (\fm{I}_\ntasks \otimes I_\rkhs) \fv{v} + \frac{1}{2} \norm{{w}}^2
        \\ &\quad  - \sum_{r=1}^T \sum_{i=1}^{\npertask_r}{ \alpha_i^r [y_{i}^r (\lambda_r \dotp{w}{\phi(x_i^r)} + (1 - \lambda_r) (\dotp{\fv{v}}{e_r \otimes \psi({x}_i^r)}) + b_r) - p_{i}^r + \xi_{i}^r]   } \\
        &\quad - \sum_{r=1}^T \sum_{i=1}^{\npertask_r}{ \beta_i^r \xi_i^r },
\end{aligned}
\end{equation}
where $\alpha_i^r, \beta_i^r \geq 0$.
Taking derivatives
% \begin{align*}
%     & \frac{\partial \mathcal{L}}{\partial {w}} = 0 \implies {w} = \sum_{r=1}^\ntasks \lambda_r \sum_{i=1}^{\npertask_r}{\alpha_i^r y_i^r \phi(x_i^r)}  \; , \\
%     & \frac{\partial \mathcal{L}}{\partial {v}_r} = 0 \implies \left((\fm{I}_\ntasks + \nu \fm{L}) \otimes I_\rkhs \right) \fv{v} = \sum_{r=1}^\ntasks (1 - \lambda_r) \sum_{i=1}^{\npertask_r}{\alpha_i^r y_i^r (e_r \otimes \psi({x}_i^r))}  \; , \\
%     & \frac{\partial \mathcal{L}}{\partial b_r} = 0 \implies  \sum_{i=1}^{m_r}{\alpha_i^r y_i^r } = 0  \; ,\\
%     & \frac{\partial \mathcal{L}}{\partial \xi_i^r} = 0 \implies C - \alpha_i^r - \beta_i^r = 0 \; .
% \end{align*}
\begin{align}
    \grad_{{w}} \lagr = 0  &\implies \optim{{w}} = \sum_{r= 1}^\ntasks \lambda_r \sum_{i=1}^{m_r} {\alpha_i^r} \left\lbrace y_i^r \phi(x_i^r) \right\rbrace , \label{eq:common_repr_cvxgl_l1} \\
    \grad_{\fv{v}} \lagr = 0 &\implies  \left((\fm{I}_\ntasks + \nu \fm{L}) \otimes I_\rkhs \right) \fv{v} = \sum_{r=1}^\ntasks (1 - \lambda_r) \sum_{i=1}^{\npertask_r}{\alpha_i^r y_i^r (e_r \otimes \psi({x}_i^r))}, \label{eq:specific_repr_cvxgl_l1} \\
    \grad_{{b}_r} \lagr = 0 &\implies \sum_{i=1}^{m_r} {\alpha_i^r} y_i^r = 0 , \label{eq:specific_eqconstr_cvxgl_l1} \\
    \grad_{\xi_i^r} \lagr = 0 &\implies C - \alpha_i^r - \beta_i^r = 0 . \label{eq:xi_feas_cvxgl_l1}
\end{align}
With 
$$E =  \left((\fm{I}_\ntasks + \nu \fm{L}) \otimes I_\rkhs \right),$$ 
and using these results to substitute in the Lagrangian, the result is
\begin{equation}\nonumber
    \begin{aligned}
            \mathcal{L}&({w}, \fv{v}, b_r, \xi_i^r, {\fv{\alpha}}, \fv{\beta}) \\
            &= \frac{1}{2} \dotp{\fv{v}}{E\fv{v}} + \frac{1}{2} \dotp{w}{w}
            \\ &\quad  - \sum_{r=1}^T \sum_{i=1}^{\npertask_r}{ \alpha_i^r [y_{i}^r (\lambda_r \dotp{w}{\phi(x_i^r)} + (1 - \lambda_r) (\dotp{\fv{v}}{e_r \otimes \psi({x}_i^r)}) + b_r) - p_{i}^r]   } \\
            &= \frac{1}{2} \dotp{E^{-1} \sum_{r=1}^\ntasks (1 - \lambda_r) \sum_{i=1}^{\npertask_r}{\alpha_i^r y_i^r (e_r \otimes \psi({x}_i^r))}}{E E^{-1} \sum_{r=1}^\ntasks (1 - \lambda_r) \sum_{i=1}^{\npertask_r}{\alpha_i^r y_i^r (e_r \otimes \psi({x}_i^r))}} \\ 
            &\quad  \frac{1}{2} + \dotp{\sum_{r=1}^\ntasks \lambda_r \sum_{i=1}^{\npertask_r}{\alpha_i^r y_i^r \phi(x_i^r)}}{\sum_{r=1}^\ntasks \lambda_r \sum_{i=1}^{\npertask_r}{\alpha_i^r y_i^r \phi(x_i^r)}} \\
            &\quad - \sum_{r=1}^T (1 - \lambda_r) \sum_{i=1}^{\npertask_r}{ \alpha_i^r y_{i}^r \left\lbrace  \dotp{E^{-1} \sum_{s=1}^\ntasks (1 - \lambda_s) \sum_{j=1}^{\npertask_s}{\alpha_j^s y_j^s (e_s \otimes \psi({x}_j^s))}}{e_r \otimes \psi(x_i^r)} \right\rbrace   } \\
            &\quad - \sum_{r=1}^T \lambda_r \sum_{i=1}^{\npertask_r}{ \alpha_i^r y_{i}^r \left\lbrace  \dotp{\sum_{s=1}^\ntasks \lambda_s \sum_{j=1}^{\npertask_s}{\alpha_j^s y_j^s \phi(x_j^s)}}{\phi(x_i^r)} \right\rbrace   } - \sum_{r=1}^\ntasks \sum_{i=1}^{\npertask_r} \alpha_i^r p_i^r \\
            &= \frac{1}{2} \sum_{r, s=1}^\ntasks (1-\lambda_r)(1-\lambda_s) \sum_{i, j=1}^{\npertask_r, \npertask_s} \alpha_j^s \alpha_i^r y_j^s y_i^r \dotp{(e_s \otimes \psi(x_j^s))}{E^{-1} (e_r \otimes \psi(x_i^r))} \\ 
            &\quad +\frac{1}{2} \sum_{r,s=1}^\ntasks \lambda_r \lambda_s \sum_{i, j=1}^{\npertask_r, \npertask_s} \alpha_j^s \alpha_i^r y_j^s y_i^r \dotp{\phi(x_j^s)}{\phi(x_i^r)} \\
            &\quad - \sum_{r, s=1}^\ntasks (1-\lambda_r)(1-\lambda_s) \sum_{i, j=1}^{\npertask_r, \npertask_s} \alpha_j^s \alpha_i^r y_j^s y_i^r \dotp{(e_s \otimes \psi(x_j^s))}{E^{-1} (e_r \otimes \psi(x_i^r))} \\ 
            &\quad - \sum_{r,s=1}^\ntasks \lambda_r \lambda_s \sum_{i, j=1}^{\npertask_r, \npertask_s} \alpha_j^s \alpha_i^r y_j^s y_i^r \dotp{\phi(x_j^s)}{\phi(x_i^r)} - \sum_{r=1}^\ntasks \sum_{i=1}^{\npertask_r} \alpha_i^r p_i^r \\
            &= -\frac{1}{2} \sum_{r, s=1}^\ntasks (1-\lambda_r)(1-\lambda_s) \sum_{i, j=1}^{\npertask_r, \npertask_s} \alpha_j^s \alpha_i^r y_j^s y_i^r (\nu \fm{L} + \fm{I}_\ntasks)^{-1}_{rs} \dotp{\psi(x_j^s)}{\psi(x_i^r)} \\ 
            &\quad -\frac{1}{2} \sum_{r,s=1}^\ntasks \lambda_r \lambda_s \sum_{i, j=1}^{\npertask_r, \npertask_s} \alpha_j^s \alpha_i^r y_j^s y_i^r \dotp{\phi(x_j^s)}{\phi(x_i^r)} - \sum_{r=1}^\ntasks \sum_{i=1}^{\npertask_r} \alpha_i^r p_i^r .
    \end{aligned}
\end{equation}
Then, the dual problem is 
\begin{equation}\label{eq:dual_cvxgl_l1_kernel}
    \begin{aligned}
        & \argmin_{\fv{\alpha}} 
        & & \Theta(\fv{\alpha}) = \frac{1}{2} \fv{\alpha}^t \left( \Lambda \fm{Q} \Lambda + \left(\fm{I}_{\nsamples} - \Lambda \right) \fm{\widetilde{Q}} \left(\fm{I}_{\nsamples} - \Lambda \right) \right) \fv{\alpha} - p \fv{\alpha} \\
        & \text{s.t.}
        & & 0 \leq \alpha_i^r \leq C, \;  i=1,\ldots,m_r; r=1,\ldots,T\; \text{(box constraints)} \; , \\
        & & & \sum_{i=1}^{n_r}{\alpha_i^r y_i^r} = 0, \; r=1,\ldots,T\; \text{(equality constraints)} \; .
        \end{aligned}
\end{equation}
where
$
\Lambda = \Diag(\lambda_1, \ldots, \lambda_\ntasks)
$
and $\fm{I}_{\nsamples}$ is the $\nsamples \times \nsamples$ identity matrix.
%
Here $Q$ is the common, standard, kernel matrix, and $\widetilde{Q}$ is the kernel matrix with the graph Laplacian information. We can define the kernel matrix
\begin{equation}
    \label{eq:dual_cvxgl_kernel_matrix}
    \widehat{\widetilde{\fm{Q}}} = \Lambda \fm{Q} \Lambda + \left(\fm{I}_{\nsamples} - \Lambda \right) \widetilde{\fm{Q}} \left(\fm{I}_{\nsamples} - \Lambda \right),
\end{equation}
as the matrix that is generated using the kernel function~\eqref{eq:dual_cvxgl_kernel_function}.


\paragraph*{Convex Graph Laplacian L2-SVM.\\}
The primal problem for convex graph Laplacian MTL based on the L2-SVM is 
\begin{equation}\label{eq:primal_cvxgl_l2_kernel}
    \begin{aligned}
    & \argmin_{\substack{\fv{v} ; \\ \substack{b_1, \ldots, b_\ntasks } ; \\ \substack{\fv{\xi}}, w; }}
    & & { C \sum_{r=1}^\ntasks \sum_{i=1}^{\npertask_r} {(\xi_i^r)^2}  + \frac{\nu}{2} \fv{v}^\intercal (\fm{L} \otimes I_\rkhs) \fv{v} + \frac{1}{2} \fv{v}^\intercal (\fm{I}_\ntasks \otimes I_\rkhs) \fv{v} + \frac{1}{2} \norm{{w}}^2} \\
    & \text{s.t.}
    & & y_i^r (\lambda_r \dotp{w}{\phi(x_i^r)} + (1 - \lambda_r) (\dotp{\fv{v}}{e_r \otimes \psi({x}_i^r)}) + b_r) \geq p_i^r - \xi_i^r  ,\\
    \end{aligned}
\end{equation}
Its corresponding Lagrangian is 
\begin{equation}\label{eq:lagr_cvxgl_l2_kernel}
    \begin{aligned}
            \mathcal{L}&({w}, \fv{v}, b_r, \xi_i^r, {\fv{\alpha}}) \\
            &= C \sum_{r=1}^T \sum_{i=1}^{\npertask_r}{(\xi_{i}^r)^2} + \frac{\nu}{2} \fv{v}^\intercal (\fm{L} \otimes I_\rkhs) \fv{v} + \frac{1}{2} \fv{v}^\intercal (\fm{I}_\ntasks \otimes I_\rkhs) \fv{v} + \frac{1}{2} \norm{{w}}^2
            \\ &\quad  - \sum_{r=1}^T \sum_{i=1}^{\npertask_r}{ \alpha_i^r [y_{i}^r (\lambda_r \dotp{w}{\phi(x_i^r)} + (1 - \lambda_r) (\dotp{\fv{v}}{e_r \otimes \psi({x}_i^r)}) + b_r) - p_{i}^r + \xi_{i}^r]   } ,
    \end{aligned}
\end{equation}
where $\alpha_i^r \geq 0$.
Taking derivatives with respect to the primal variables the results are
The gradients with respect to the primal variables are
\begin{align}
    \grad_{{w}} \lagr = 0  &\implies \optim{{w}} = \sum_{r= 1}^\ntasks \lambda_r \sum_{i=1}^{m_r} {\alpha_i^r} \left\lbrace y_i^r \phi(x_i^r) \right\rbrace , \label{eq:common_repr_cvxgl_l2} \\
    \grad_{\fv{v}} \lagr = 0 &\implies  \left((\fm{I}_\ntasks + \nu \fm{L}) \otimes I_\rkhs \right) \fv{v} = \sum_{r=1}^\ntasks (1 - \lambda_r) \sum_{i=1}^{\npertask_r}{\alpha_i^r y_i^r (e_r \otimes \psi({x}_i^r))}, \label{eq:specific_repr_cvxgl_l2} \\
    \grad_{{b}_r} \lagr = 0 &\implies \sum_{i=1}^{m_r} {\alpha_i^r} y_i^r = 0 , \label{eq:specific_eqconstr_cvxgl_l2} \\
    \grad_{\xi_i^r} \lagr = 0 &\implies C \xi_i^r - {\alpha_i^r} = 0 . \label{eq:xi_feas_cvxgl_l2}
\end{align}
    






\paragraph*{Convex Graph Laplacian LS-SVM.\\}




\subsection{Experiments}

\section{Adaptive Graph Laplacian Algorithm}

If we want to learn $A$ from data we would like the matrix to meet some requirements:
\begin{itemize}
    \item $A$ has to be symmetric, so we can express the regularizer using the Laplacian in the dual form.
    \item The rows of $A$ add up to 1. 
\end{itemize} 

\subsection{Experiments}


%\section{Experiments}

\section{Conclusions}\label{sec-conclusions-4}

In this chapter, we have\dots
