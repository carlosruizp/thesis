% Chapter 8

\chapter{Conclusions And Future Work} % Write in your own chapter title
\label{ChapterConclusions}
\lhead{Chapter \ref{ChapterConclusions}. \emph{Conclusions And Future Work}} 
%Bayesian optimization is used to optimize black-box functions, \textit{i.e.}, potentially noise expensive functions with unknown analytical expressions. In particular, examples of black-box optimization include hyper-parameter tuning of machine learning algorithms to optimize an estimation of the generalization error, industry problems such as the optimization of the configuration of a robot or even curious applications such as the optimization of the cooking recipe of a cookie. Concretely, the standard BO scenario concerns the optimization of a single black-box function. However, this thesis has introduced methods that extend the applicability of BO to broader scenarios such as constrained multi-objective scenarios (Chapter \ref{Chapter4}), parallel BO (Chapter \ref{Chapter5}) or dealing with integer and categorical-valued variable (Chapter \ref{Chapter6}). We also include a real application regarding the optimization of a hybrid grouping genetic algorithm applied to an extreme learning machine for robust prediction of ocean wave features in Chapter \ref{Chapter7}. In those chapters, we have illustrated the usefulness of the proposed methods with toy, synthetic, benchmark or real problems. Finally, in this section, we illustrate the main conclusions of this thesis.

\section{Conclusions}
%
\acrfull{ml} has the goal of automatizing the learning process so machines can be programmed to learn. For this purpose, several approaches have been considered, but all of them ultimately try to estimate some function from data inductively, i.e. from particular realizations of some process, \acrshort{ml} methods try to infer the underlying function generating this data. Some of the most relevant \acrshort{ml} methods are kernel methods (such as the \acrfull{svm}) and the \acrshort{nn}.

\acrfull{mtl} is a field of \acrshort{ml} that considers learning multiple tasks jointly with the goal of improving the learning process, that is, multiple functions are to be estimated and the goal is to obtain leverage by learning them together. The trivial approximations for these scenarios would be \acrfull{ctl}, where a common model is applied for all tasks, and \acrfull{itl}, where task-specific models are considered, without any relation between them. However, \acrshort{mtl} considers another way, where there is one model for each task, but there exist a connection between them.

For the goals of \acrshort{mtl}, several strategies have been considered; in this work, in Chapter~\refeq{Chapter3}, we have categorized them in three groups: feature-based, parameter-based and regularization-based.
In short, feature-based methods try to find a shared representation of the data that is useful for all tasks, parameter-based methods try to couple different tasks using regularization schemes, and combination-based ones combine a common and task-specific parts.
Different \acrshort{ml} methods, based on their characteristics, are more easily adaptable to one strategy or another, as described in Chapter~\ref{Chapter3}. In particular, \acrfull{nns} are most compatible with feature-based approaches: a very natural way to implement \acrshort{mtl} is to define a network with shared hidden layers and task-specific output neurons.
Other example is linear models, which are also compatible with feature-based strategies; however, the parameter-based ones are more interesting in this case, where different properties can be enforced on the linear models (such as different coupling schemes, sparsity...) through specialized regularizers.
Looking at the kernel methods, although they bring desirable properties, they present a more rigid framework to adapt \acrshort{mtl} strategies for them. Since the features depend on the choice of the kernel, feature-based strategies cannot be easily applied. Moreover, the parameters of these models are elements of a \acrfull{rkhs}, so designing specialized regularizers for them is more difficult; however, combination-based strategies fit well with kernel methods.
%
In this work we develop novel \acrshort{mtl} strategies for kernel methods, which are also applicable to other models such as \acrshort{nns}.


%
First, we describe a convex combination approach in Chapter~\ref{Chapter4}. One of the main challenges of combination-based strategies is selecting the optimal hyperparameters for our models, where it is necessary to select the weight of the common and task-specific part in each model. In particular, we propose a convex formulation, where the combination parameters $\lambda_r \in [0, 1]$ are task-specific, have a clear interpretation and can recover the trivial \acrshort{ctl} and \acrshort{itl} cases.
%
For a deeper understanding, we develop this convex combination approach for three \acrshort{svm} variants: L1, L2 and LS-\acrshort{svm}; we also discuss its properties both in the dual and primal formulations.
%
In addition, we consider a natural alternative, which applies the convex combination of pre-trained common and task-specific models. We describe how the optimal combination parameters can be obtained in this case.
%
In the experiments of Chapter~\ref{Chapter6} we show how, in most cases, this convex approach outperforms the \acrshort{ctl} and \acrshort{itl} strategies, as well as the convex combination of pre-trained models.
The experiments concerning energy prediction are specially relevant; here, we observe that defining tasks in our data to build more specialized models can improve our results.

%
Furthermore, this convex formulation is a general one, and it is applicable to other models; to the best of our knowledge, we propose in this work the first combination-based \acrshort{mtl} strategy for \acrshort{nns}.
%
In Chapter~\ref{Chapter6}, using image datasets, we observe how this approach gets the best result when compared with its \acrshort{ctl} and \acrshort{itl} counterparts, and also when compared with the commonly used approach of sharing weights in the hidden layers.


%
As we have already seen, parameter-based strategies are more difficult to implement for kernel methods; however, in Chapter~\ref{Chapter5} we propose a formulation based on tensor products of \acrfull{rkhss} that enables us to apply some of these strategies.
As a result of this formulation, we observe that we can define kernels that split the task and feature information in two parts, which are multiplied.

In particular, taking this tensor-based formulation, in Chapter~\ref{Chapter5} we develop a \acrfull{gl} approach for the L1, L2 and LS-\acrshort{svm}. In this strategy, the tasks are seen as nodes in a graph, and the pairwise distance between task parameters are penalized, where the graph adjacency matrix is used to weight these distances.
Our formulation enables us to write a kernel that splits the task relatedness, as indicated in the adjacency matrix, and the features similarities.
%
Additionally, we combine the \acrshort{gl} approach with the convex formulation, where we use a convex part and task-specific parts coupled through a graph regularization.
%
The results of our experiments in Chapter~\ref{Chapter6} show that this convex \acrshort{gl} strategy is an interesting alternative to the plain convex formulation, each capturing different properties of the problems. We observe that the choice of the best strategy depends on the characteristics of each problem.

As we have explained, the \acrshort{gl} models strongly depend on the selection of an adjacency matrix, so, in Chapter~\ref{Chapter5}, we develop an algorithm to automatically learn this matrix from the data.
We provide multiple experimental results in Chapter~\ref{Chapter6} that show how this approach enables us to learn the task relations and, in consequence, improve obtain better models.


\section{Future Work}
Our proposed methods show how we can adapt different \acrshort{mtl} strategies principally to kernel methods, but also to other model such as \acrshort{nns}. 
In this work, we have proposed a general convex formulation for \acrshort{mtl}, which we apply on kernel methods and \acrshort{nns}. Also, we have presented a \acrshort{gl} approach that we have applied to the L1, L2 and LS-\acrshort{svm}.
%
These proposals offer the possibility of new extensions, and they also bring challenges for which we can develop novel solutions. Precisely, to conclude this work, we describe some of the ideas that can be studied as future work.

% Hyperparameter selection (general) BO, simulated annealing 
One of the main difficulties of using the combination-based \acrshort{mtl} strategies with kernel methods is the large number of hyperparameters that have to be selected. Recall that, with these approaches, it is possible to apply a common kernel and task-specific ones, which can all be different. In addition, we have other hyperparameters, such as the regularization ones or, in our convex formulation, those that govern the convex combinations. 
%
Therefore, to fully exploit the expressive power of these models we would like to select the optimal values for each one of these hyperparameters. Standard methods, such as using a grid search, are not feasible for this kind of models; however, we can consider alternative strategies, such as Bayesian Optimization or Simulated Annealing, to fully hyperparametrize these models.
% Kernel hyperpar selection (why is relevant)
In the case of kernel hyperparameters, such as the kernel width in the Gaussian kernel, strategies related to kernel learning can be applied and extended for the \acrshort{mtl} case. 
% convex hyperpar selection + extconvex
In our proposal, the convex \acrshort{mtl} formulation, the selection of the convex combination parameters is particularly relevant. One approximation is to act with them as with the rest of hyperparameters and rely on general strategies (such as Bayesian Optimization), but we can also develop specific methods that learn these parameters from data, similarly to what we have done with the adjacency matrix of the \acrshort{gl} proposal.
% convex hyperpar selection with NN
Furthermore, in the case of the convex \acrshort{mtl} \acrshort{nn} it is not necessary to develop any novel algorithm, we can treat these convex combination hyperparameters as parameters of the network and optimize them using standard gradient descent techniques.

% gl with nn
Shifting our attention to the \acrshort{gl} approaches, a possible extension of our work would be to consider applying the \acrshort{gl} regularization to \acrshort{nns}.
The regularization penalizing the distances between the weights of the task-specific networks (or subnetworks) is differentiable, so gradient descent would be a valid technique to optimize these models.



% We provide a list with the main conclusions of this thesis regarding the work that has been shown in the previous chapters.
% \begin{itemize}
%     \item First, we developed a method that performs Bayesian optimization in constrained multi-objective scenarios. This method is called Predictive Entropy Search for multi-objective optimization with constraints (PESMOC). Concretely, this method optimizes conflicting black-boxes under the presence of several constraints. In particular, solutions that do not fulfill the constraints are not considered valid. Most importantly, PESMOC iteratively suggests the recommendation that minimizes the expected reduction on the entropy of the Pareto set. In Chapter \ref{Chapter4} we showed the usefulness of PESMOC for optimizing several objectives and constraints regarding ensembles and the implementation of deep neural networks in hardware. In those experiments, PESMOC outperforms other methods tackling the constrained multi-objective scenario such as a generalization of the expected improvement acquisition function, that is expected to be greedy, or a random search, that only performs pure exploration.
%     \item As we have already seen, Bayesian optimization suggests a single recommendation point in every iteration of the algorithm. Nevertheless, there are settings where we may have a cluster of nodes available to process suggestions. Unfortunately, standard Bayesian optimization leaves them idle as it can only provides a single suggestion at a time. In order to solve this issue, we generalized the previous constrained multi-objective approach to make it suggest a batch of points for every iteration. In particular, this extension of PESMOC is called Parallel Predictive entropy search for multi-objective optimization with constraints (PPESMOC). In this case, the acquisition function now suggests a batch of points that minimize the expected reduction on the entropy search of the Pareto set. As in the case of PESMOC, we tested PPESMOC on the same experiments of PESMOC but comparing it with greedy versions of PPESMOC where we iteratively create a batch of points using the PESMOC acquisition function. Critically, PPESMOC outperforms or performs similarly to these methods on the proposed experiments. Moreover, it scales the batch size better than the proposed baselines.
%     \item We have studied that BO uses Gaussian processes to model the black-box. GPs assume continuous input real variables. When the problem involves other variables, such as integer-valued or categorical variables, it is common to perform a one hot encoding approximation for categorical variables or to round the integer-valued variables to the closest integer. In this work, we show how such procedures incur in a bad performance of the BO method. To circumvent this issue, we propose a transformation for the input space variables that alleviates the issues that arise in previous procedures. In particular, we include empirical evidence that shows how our transformation outperforms previous approaches dealing with integer-valued and categorical variables.
%     \item Finally, we propose the Bayesian optimization of a hybrid grouping genetic algorithm for attribute selection combined with an extreme learning machine (GGA-ELM) approach for robust prediction of ocean wave features. The contribution of the thesis regarding this work was the design and implementation of the experiments. In Chapter \ref{Chapter7}, we perform two sets of experiments regarding the prediction of the wave energy flux and wave height optimization. Critically, we showed how BO outperforms the configuration suggested by experts on the field and the performance delivered by random search.
% \end{itemize}

% \section{Future Work}
% Our proposed methods show how BO can be effectively adapted to cover a wide range of different scenarios with an excellent performance. Moreover, there are a plethora of scenarios that BO can cover in addition to the described settings that could be targeted by future work. Precisely, to conclude this document, we include some ideas as further work that can be studied by future research.
% \begin{itemize}
%     \item PESMOC assumes that the black-boxes to be optimized are independent. But this is not necessarily true in real-world problems. The black-boxes can have dependencies. For example, consider the optimization of the number of workers of the company and the benefit. In this example, the benefit is dependent on the hired number of employees. We hypothesize that, by modelling these dependencies, we can improve the performance of the PESMOC method \citep{shah2016pareto}. This work will need to use a multi-output GP to model these dependencies and to propose a new acquisition function, for example extending PESMOC, that takes into account the information provided by this model \citep{moreno2018heterogeneous}.
%     \item We have used Bayesian optimization with Gaussian processes. Gaussian processes are flexible priors over functions, but may have difficulties modelling non-stationary functions. An extension of GPs, deep Gaussian processes, are designed to be priors over non-stationary functions \citep{damianou2013deep, bui2016deep}. Deep GPs consist of multiple GP mappings organized in several layers. The input of every layer is the output of the previous layer, that consist of several sparse GP. The nodes are sparse GPs to make the deep GP scale to more observations, typically being used from $500$ to $5000000$ observations. However, Bayesian optimization scenarios do not usually consider more than $300$ observations. Hence, if deep GPs are used for BO, they need to be modified to include GPs in their layers as the complexity in the number of observations is not a problem in the BO setting but if we use sparse GPs the performance will suffer. A future line of research is to design a deep GP that can perform well for BO scenarios. In order to do so, we will need to perform hyper-parameter sampling, that is not usually done for deep GPs on regression problems.
%     \item The acquisition function of Bayesian optimization can easily be optimized in a low number of dimensions with a grid search procedure and a local optimizer method such as L-BFGS. In most cases, it is also possible to compute the gradients of the acquisition function. In particular, the number of points of the grid search is implemented to scale linearly with the number of dimensions. However, due to the curse of dimensionality, as we increment the number of dimensions or for parallel acquisition functions such as PPESMOC (whose dimensional complexity is a function of the size of the batch) and the number of dimensions of the input space this methodology will deliver worse and worse results. In order to circumvent this issue, it would be interesting to test evolutionary strategies such as NSGA-II or other metaheuristic strategies to optimize the acquisition function in a high number of dimensions to enhance the results obtained by the proposed methods \citep{deb2002fast}.
%     \item An interesting application where we can apply constrained multi-objective BO is to deploy fair ML algorithms \citep{perrone2020fair}. Fairness strategies ensure ML algorithms not to incur in discriminations such as racism, machism or ageism. These strategies may be conflicting with the optimization of the estimation of the generalization error. Moreover, they can be considered as black-boxes. Additionally, if we want that these strategies can invalidate certain solutions, we can also implement them as constraints. Therefore, the methods proposed, and more precisely PESMOC, could be used to address the problem described. Further research may analyze the utility of PESMOC for finding fair machine learning models.
% \end{itemize}
