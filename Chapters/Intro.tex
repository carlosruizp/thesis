% Chapter 1

\chapter{Introduction} % Write in your own chapter title
\label{Chapter1}
\lhead{Chapter \ref{Chapter1}. \emph{Introduction}} % Write in your own chapter title to set the page header

{\bf \small{
We begin this manuscript...
}}


\section{Introduction}

\section{Publications}

This section presents, in chronological order, the work published during the doctoral period in which this thesis was written. We also include other research work related to this thesis, but not directly included on it. Finally, this document includes content that has not been published yet and is under revision.



\subsection*{Related Work}



\subsection*{Work In Progress}


\section{Summary by Chapters}\label{sumchap}

In this section\dots
\begin{description}

\item [{Chapter \ref{Chapter2}}] provides an introduction to GPs and the expectation propagation algorithm. Both are necessary concepts for the BO methods that we will describe in the following chapters. This chapter reviews the fundamentals of GPs and why they are so interesting for BO. More concretely, we review the most popular kernels, the analysis of the posterior and predictive distribution and how to tune the hyper-parameters of GPs: whether by maximizing the marginal likelihood or by generating samples from the hyper-parameter posterior distribution. Other alternative probabilistic surrogate models are also described briefly. Some of the proposed approaches of this thesis are extensions of an acquisition function called predictive entropy search, that is based on the expectation propagation approximate inference technique. That is why we provide in this chapter an explanation of the expectation propagation algorithm.

\item [{Chapter \ref{Chapter3}}] introduces the basics of BO and information theory. BO works with probabilistic models such as GPs and with acquisition functions such as predictive entropy search, that uses information theory. Having studied GPs in Chapter \ref{Chapter2}, BO can be now understood and it is described in detail. This chapter will also describe the most popular acquisition functions, how information theory can be applied in BO and why BO is useful for the hyper-parameter tuning of machine learning algorithms.

\item [{ Chapter \ref{Chapter4}}] describes an information-theoretical mechanism that generalizes BO to simultaneously optimize multiple objectives under the presence of several constraints. This algorithm is called predictive entropy search for multi-objective BO with constraints (PESMOC) and it is an extension of the predictive entropy search acquisition function that is described in Chapter \ref{Chapter3}. The chapter compares the empirical performance of PESMOC with respect to a state-of-the-art approach to constrained multi-objective optimization based on the expected improvement acquisition function. It is also compared with a random search through a set of synthetic, benchmark and real experiments. 

\item [{ Chapter \ref{Chapter5}}] addresses the problem that faces BO when not only one but multiple input points can be evaluated in parallel that has been described in Section \ref{sec_complex}. This chapter introduces an extension of PESMOC called parallel PESMOC (PPESMOC) that adapts to the parallel scenario. PPESMOC builds an acquisition function that assigns a value for each batch of points of the input space. The maximum of this acquisition function corresponds to the set of points that maximizes the expected reduction in the entropy of the Pareto set in each evaluation. Naive adaptations of PESMOC and the method based on expected improvement for the parallel scenario are used as a baseline to compare their performance with PPESMOC. Synthetic, benchmark and real experiments show how PPESMOC obtains an advantage in most of the considered scenarios. All the mentioned approaches are described in detail in this chapter. 

\item [{ Chapter \ref{Chapter6}}] addresses a transformation that enables standard GPs to deliver better results in problems that contain integer-valued and categorical variables. We can apply BO to problems where we need to optimize functions that contain integer-valued and categorical variables with more guarantees of obtaining a solution with low regret. A critical advantage of this transformation, with respect to other approaches, is that it is compatible with any acquisition function. This transformation makes the uncertainty given by the GPs in certain areas of the space flat. As a consequence, the acquisition function can also be flat in these zones. This phenomenom raises an issue with the optimization of the acquisition function, that must consider the flatness of these areas. We use a one exchange neighbourhood approach to optimize the resultant acquisition function. We test our approach in synthetic and real problems, where we add empirical evidence of the performance of our proposed transformation. 

\item[{Chapter \ref{Chapter7}}] shows a real problem where BO has been applied with success. In this problem, BO has been used to obtain the optimal parameters of a hybrid Grouping Genetic Algorithm for attribute selection. This genetic algorithm is combined with an Extreme Learning Machine (GGA-ELM) approach for prediction of ocean wave features. Concretely, the significant wave height and the wave energy flux at a goal marine structure facility on the Western Coast of the USA is predicted. This chapter illustrates the experiments where it is shown that BO improves the performance of the GGA-ELM approach. Most importantly, it also outperforms a random search of the hyper-parameter space and the human expert criterion. 

\item [{Chapter \ref{Chapter8}}] provides a summary of the work done in this thesis. We include the conclusions retrieved by the multiple research lines covered in the chapters. We also illustrate lines for future research.

\end{description}

\section{Definitions and Notation}

