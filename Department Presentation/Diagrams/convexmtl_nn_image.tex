% \begin{figure}[t!]
    \centering
    \begin{tikzpicture}

        % Input Layer
        \foreach \i in {1,...,\inputnum}
        {
            \node[circle, 
                minimum size = \minnodesize,
                fill=orange!30] (Input_common-\i) at (0,-\i- \hiddennum) {};
        }

        \node[circle, 
            minimum size = \minnodesize,
            fill=purple!30] (Pred1) at (8,-1.2 * \hiddennum) {};

        \node[circle, 
        minimum size = \minnodesize,
        fill=purple!30] (Pred2) at (8,-2.2 * \hiddennum) {};

        \draw[->, shorten <=1pt] (Pred1) -- ++(1,0)
            node[right]{$h_1(\fv{x})$};

        \draw[->, shorten <=1pt] (Pred2) -- ++(1,0)
        node[right]{$h_2(\fv{x})$};

        %%%%%%%%%%%%%%%%%%%% Specific NN 1 %%%%%%%%%%%%%%%%%%%%%%%%%%
    %     % Input Layer
    % \foreach \i in {1,...,\inputnum}
    % {
    %     \node[circle, 
    %         minimum size = \minnodesize,
    %         fill=orange!30] (Input_sp1-\i) at (0,-\i) {};
    % }
     
    
    % Hidden Layer 1
    \foreach \i in {1,...,\hiddennum}
    {
        \node[circle, 
            minimum size = \minnodesize,
            fill=teal!50,
            yshift=(\hiddennum-\inputnum)*5 mm
        ] (Hidden1_sp1-\i) at (2,-\i) {};
    }
    
    % Hidden Layer 2
    \foreach \i in {1,...,\hiddennum}
    {
        \node[circle, 
            minimum size = \minnodesize,
            fill=teal!50,
            yshift=(\hiddennum-\inputnum)*5 mm
        ] (Hidden2_sp1-\i) at (4,-\i) {};
    }
     
    % Output Layer
    \foreach \i in {1,...,1}
    {
        \node[circle, 
            minimum size = \minnodesize,
            fill=black!50,
            yshift=(1-\inputnum)*5 mm
        ] (Output_sp1-\i) at (6,-\i) {$g_1(\fv{x})$};
    }
     
    % Connect neurons In-Hidden
    \foreach \i in {1,...,\inputnum}
    {
        \foreach \j in {1,...,\hiddennum}
        {
            \draw[->, shorten >=1pt,blue!80!black, dashed] (Input_common-\i) -- (Hidden1_sp1-\j);   
        }
    }


    % Connect neurons Hidden-Hidden
    \foreach \i in {1,...,\hiddennum}
    {
        \foreach \j in {1,...,\hiddennum}
        {
            \draw[->, shorten >=1pt,blue!80!black, dashed] (Hidden1_sp1-\i) -- (Hidden2_sp1-\j);   
        }
    }
     
    % Connect neurons Hidden-Out
    \foreach \i in {1,...,\hiddennum}
    {
        \foreach \j in {1,...,1}
        {
            \draw[->, shorten >=1pt,blue!80!black, dashed] (Hidden2_sp1-\i) -- (Output_sp1-\j);
        }
    }
     
    % % Inputs
    % \foreach \i in {1,...,\inputnum}
    % {            
    %     \draw[<-, shorten <=1pt] (Input_sp1-\i) -- ++(-1,0)
    %         node[left]{$x_{\i}$};
    % }
     
    % Outputs
    % \foreach \i in {1,...,1}
    % {            
    %     \draw[->, shorten <=1pt] (Output_sp1-\i) -- ++(2,0)
    %         node[right]{$g_1(x)$};
    % }

    \draw[->, ultra thick] (Output_sp1-1) -- (Pred1) node [midway, fill=white] {$1 - \lambda$};

    
    
    \draw[thick]     ($(Hidden1_sp1-1.north west)+(-0.5,0.15)$) rectangle ($(Output_sp1-1.south east)+(0.3,-0.45)$);
    
    
    %%%%%%%%%%%%%%%%%% Common NN %%%%%%%%%%%%%%%%%%%%%%%%%%%%%%%%%%%

    
     
    
    % Hidden Layer 1
    \foreach \i in {1,...,\hiddennum}
    {
        \node[circle, 
            minimum size = \minnodesize,
            fill=teal!50,
            yshift=(\hiddennum-\inputnum)*5 mm
        ] (Hidden1_common-\i) at (2,-\i- \hiddennum) {};
    }
    
    % Hidden Layer 2
    \foreach \i in {1,...,\hiddennum}
    {
        \node[circle, 
            minimum size = \minnodesize,
            fill=teal!50,
            yshift=(\hiddennum-\inputnum)*5 mm
        ] (Hidden2_common-\i) at (4,-\i- \hiddennum) {};
    }
     
    % Output Layer
    \foreach \i in {1,...,1}
    {
        \node[circle, 
            minimum size = \minnodesize,
            fill=black!50,
            yshift=(1-\inputnum)*5 mm
        ] (Output_common-\i) at (6,-\i- \hiddennum) {$g(\fv{x})$};
    }
     
    % Connect neurons In-Hidden
    \foreach \i in {1,...,\inputnum}
    {
        \foreach \j in {1,...,\hiddennum}
        {
            \draw[->, shorten >=1pt,red!80!black, densely dashdotted] (Input_common-\i) -- (Hidden1_common-\j);   
        }
    }

    % Connect neurons In-Hidden
    \foreach \i in {1,...,\hiddennum}
    {
        \foreach \j in {1,...,\hiddennum}
        {
            \draw[->, shorten >=1pt,red!80!black, densely dashdotted] (Hidden1_common-\i) -- (Hidden2_common-\j);   
        }
    }
     
    % Connect neurons Hidden-Out
    \foreach \i in {1,...,\hiddennum}
    {
        \foreach \j in {1,...,1}
        {
            \draw[->, shorten >=1pt,red!80!black, densely dashdotted] (Hidden2_common-\i) -- (Output_common-\j);
        }
    }
     
    % Inputs
    \foreach \i in {1,...,\inputnum}
    {            
        \draw[<-, shorten <=1pt] (Input_common-\i) -- ++(-1,0)
            node[left]{};
    }
     
    % % Outputs
    % \foreach \i in {1,...,1}
    % {            
    %     \draw[->, shorten <=1pt] (Output_common-\i) -- ++(2,0)
    %         node[right]{$g(x)$};
    % }

    \draw[->, ultra thick] (Output_common-1) -- (Pred1) node [midway, fill=white] {$\lambda$};
    \draw[->, ultra thick] (Output_common-1) -- (Pred2) node [midway, fill=white] {$\lambda$};


    \draw[thick,blue,dotted]     ($(Hidden1_common-1.north west)+(-0.5,0.15)$) rectangle ($(Output_common-1.south east)+(0.3,-0.45)$);

    %%%%%%%%%%%%%%%%%%%% Specific NN 2 %%%%%%%%%%%%%%%%%%%%%%%%%%
        % % Input Layer
        % \foreach \i in {1,...,\inputnum}
        % {
        %     \node[circle, 
        %         minimum size = \minnodesize,
        %         fill=orange!30] (Input_sp2-\i) at (0,-\i- 2*\hiddennum) {};
        % }
         
        
        % Hidden Layer 1
        \foreach \i in {1,...,\hiddennum}
        {
            \node[circle, 
                minimum size = \minnodesize,
                fill=teal!50,
                yshift=(\hiddennum-\inputnum)*5 mm
            ] (Hidden1_sp2-\i) at (2,-\i- 2*\hiddennum) {};
        }
        
        % Hidden Layer 2
        \foreach \i in {1,...,\hiddennum}
        {
            \node[circle, 
                minimum size = \minnodesize,
                fill=teal!50,
                yshift=(\hiddennum-\inputnum)*5 mm
            ] (Hidden2_sp2-\i) at (4,-\i- 2*\hiddennum) {};
        }
         
        % Output Layer
        \foreach \i in {1,...,1}
        {
            \node[circle, 
                minimum size = \minnodesize,
                fill=black!50,
                yshift=(1-\inputnum)*5 mm
            ] (Output_sp2-\i) at (6,-\i- 2*\hiddennum) {$g_2(\fv{x})$};
        }
         
        % Connect neurons In-Hidden
        \foreach \i in {1,...,\inputnum}
        {
            \foreach \j in {1,...,\hiddennum}
            {
                \draw[->, shorten >=1pt] (Input_common-\i) -- (Hidden1_sp2-\j);   
            }
        }
    
        % Connect neurons In-Hidden
        \foreach \i in {1,...,\hiddennum}
        {
            \foreach \j in {1,...,\hiddennum}
            {
                \draw[->, shorten >=1pt] (Hidden1_sp2-\i) -- (Hidden2_sp2-\j);   
            }
        }
         
        % Connect neurons Hidden-Out
        \foreach \i in {1,...,\hiddennum}
        {
            \foreach \j in {1,...,1}
            {
                \draw[->, shorten >=1pt] (Hidden2_sp2-\i) -- (Output_sp2-\j);
            }
        }
         
        % % Inputs
        % \foreach \i in {1,...,\inputnum}
        % {            
        %     \draw[<-, shorten <=1pt] (Input_sp2-\i) -- ++(-1,0)
        %         node[left]{$x_{\i}$};
        % }
         
        % Outputs
        % \foreach \i in {1,...,1}
        % {            
        %     \draw[->, shorten <=1pt] (Output_sp2-\i) -- ++(2,0)
        %         node[right]{$g_2(x)$};
        % }

        \draw  [->, ultra thick] (Output_sp2-1) -- (Pred2) node [midway, fill=white] {$1 - \lambda$};
         
        \draw[thick]     ($(Hidden1_sp2-1.north west)+(-0.5,0.15)$) rectangle ($(Output_sp2-1.south east)+(0.3,-0.45)$);

    \end{tikzpicture}
%     \caption{Ejemplo de formulación convexa con redes neuronales para dos tareas.
% 	}
%     \label{fig:convexmtl_nn}
% \end{figure}
