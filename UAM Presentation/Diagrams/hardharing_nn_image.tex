% \begin{figure}[t!]
    \centering
    \begin{tikzpicture}
 
        % Input Layer
        \foreach \i in {1,...,\inputnum}
        {
            \node[circle, 
                minimum size = 6mm,
                fill=orange!30] (Input-\i) at (0,-\i) {};
        }
         
        
        % Hidden Layer 1
        \foreach \i in {1,...,\hiddennumhs}
        {
            \node[circle, 
                minimum size = 6mm,
                fill=teal!50,
                yshift=(\hiddennumhs-\inputnum)*5 mm
            ] (Hidden1-\i) at (2.5,-\i) {};
        }
        
        % Hidden Layer 2
        \foreach \i in {1,...,\hiddennumhs}
        {
            \node[circle, 
                minimum size = 6mm,
                fill=teal!50,
                yshift=(\hiddennumhs-\inputnum)*5 mm
            ] (Hidden2-\i) at (5,-\i) {};
        }
         
        % Output Layer
        \foreach \i in {1,...,\outputnum}
        {
            \node[circle, 
                minimum size = 6mm,
                fill=purple!50,
                yshift=(\outputnum-\inputnum)*5 mm
            ] (Output-\i) at (7.5,-\i) {};
        }
         
        % Connect neurons In-Hidden
        \foreach \i in {1,...,\inputnum}
        {
            \foreach \j in {1,...,\hiddennumhs}
            {
                \draw[->, shorten >=1pt, red!80!black, densely dashdotted] (Input-\i) -- (Hidden1-\j);   
            }
        }

        % Connect neurons Hidden-Hidden
        \foreach \i in {1,...,\hiddennumhs}
        {
            \foreach \j in {1,...,\hiddennumhs}
            {
                \draw[->, shorten >=1pt, red!80!black, densely dashdotted] (Hidden1-\i) -- (Hidden2-\j);   
            }
        }
         
        % Connect neurons Hidden-Out
        \foreach \i in {1,...,\hiddennumhs}
        {
            \foreach \j in {1}
            {
                \draw[->, shorten >=1pt, blue!80!black, dashed] (Hidden2-\i) -- (Output-\j);
            }
        }

        \foreach \i in {1,...,\hiddennumhs}
        {
            \foreach \j in {2,...,\outputnum}
            {
                \draw[->, shorten >=1pt] (Hidden2-\i) -- (Output-\j);
            }
        }
         
        % Inputs
        \foreach \i in {1,...,\inputnum}
        {            
            \draw[<-, shorten <=1pt] (Input-\i) -- ++(-1,0)
                node[left]{};
        }
         
        % Outputs
        \foreach \i in {1,...,\outputnum}
        {            
            \draw[->, shorten <=1pt] (Output-\i) -- ++(1,0)
                node[right]{$h_{\i}(\fv{x})$};
        }
         
    \end{tikzpicture}
    
%     \caption{Ejemplo de \emph{Hard Sharing} para dos tareas . 
%     %The input neurons are shown in orange, the hidden ones in cyan and the output ones in magenta.
%     %Assuming a sample belonging to task $1$ is used, the shared weights updated in training are represented in red, and in blue the updated specific weights. 
%     }
%     \label{fig:hardsharing_nn}
% \end{figure}