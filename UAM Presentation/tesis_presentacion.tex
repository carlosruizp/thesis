\documentclass{beamer}
\usepackage{amsfonts,amsmath,oldgerm}
\usepackage{bm}
\usepackage{siunitx}       % typesetting values with units
\usepackage[ruled]{algorithm2e}

% color gradient bar
\usepackage{tikz}
\usepackage{tabularx}

\graphicspath{{Figures/}}  % Location of the graphics files (set up for graphics to be in PDF format)


%algorithm2e
%%% Coloring the comment as blue
\newcommand\mycommfont[1]{\footnotesize\ttfamily\textcolor{blue}{#1}}
\SetCommentSty{mycommfont}

\SetKwInput{KwInput}{\small{Input}}                % Set the Input
\SetKwInput{KwOutput}{\small{Output}}              % set the Output
\SetKwInput{KwData}{\small{Data}}              % set the Output

%tikz
\usetikzlibrary{arrows,positioning} 
\tikzset{
    %Define standard arrow tip
    >=stealth',
    %Define style for boxes
    punkt/.style={
           rectangle,
           rounded corners,
           draw=black, very thick,
           text width=6.5em,
           minimum height=2em,
           text centered},
    % Define arrow style
    pil/.style={
           ->,
           thick,
           shorten <=2pt,
           shorten >=2pt,}
}
\usetikzlibrary{calc}
%Tikz picture
% Input layer neurons'number
\newcommand{\inputnum}{2} 
 
% Hidden layer neurons'number
\newcommand{\hiddennum}{2}  

% Hidden layer neurons'number hard sharing
\newcommand{\hiddennumhs}{4}  

% Hidden layer neurons'number hard sharing
\newcommand{\hiddennumsp}{3}  
 
% Output layer neurons'number
\newcommand{\outputnum}{2} 

% Min Node size
\newcommand{\minnodesize}{5mm} 

\usetheme{sintef}

\newtheorem{proposition}[theorem]{Proposicion}

\newcommand{\testcolor}[1]{\colorbox{#1}{\textcolor{#1}{test}}~\texttt{#1}}

\newcommand\Mark[2][8.4]{%
  \rlap{\tikz[baseline=(current bounding box.south)]{
        \shade[left color=darkgray, right color=maincolor!#2!darkgray]
               (0,0) rectangle ++(#1*#2/100,0.3);}%
  }%
}

% My definitions

  % Math Operators
  \DeclareMathOperator*{\argmin}{arg\min}
  \DeclareMathOperator*{\argmax}{arg\max}
  \DeclareMathOperator*{\arginf}{arg\inf}
  \DeclareMathOperator*{\argsup}{arg\sup}
  \DeclareMathOperator{\Tr}{tr}
  \DeclareMathOperator{\Span}{span}
  \DeclareMathOperator{\Diag}{diag}
  \DeclareMathOperator{\rank}{rank}
  \DeclareMathOperator{\sign}{sign}
  \DeclareMathOperator{\expect}{E}
  \DeclareMathOperator{\VCdim}{VCdim}
  
  \DeclareMathOperator{\mn}{\mathcal{M_N}}
  \DeclareMathOperator{\tn}{\mathcal{T_N}}
  
  
  
  \DeclareMathOperator{\Ima}{Im}
  \DeclareMathOperator{\Ker}{Ker}
  \DeclareMathOperator{\Vector}{vec}
  
%   \DeclarePairedDelimiter\floor{\lfloor}{\rfloor}
%   \DeclarePairedDelimiter\ceil{\lceil}{\rceil}
%   \DeclarePairedDelimiter\equivclass{\lbrack}{\rbrack}
  
  
  
  \newcommand{\comm}[1]{{\color{red} #1}}
  
  \newcommand{\norm}[1]{\left\lVert#1\right\rVert}
  \newcommand{\abs}[1]{\left|#1\right|}
  \newcommand{\mymax}[1]{\max\left(#1\right)}
  \newcommand{\mymin}[1]{\min\left(#1\right)}
  \newcommand{\pospart}[1]{\left[#1\right]_{+}}
  \newcommand{\epsins}[1]{\left[#1\right]_{\epsilon}}
  
  \newcommand{\set}[1]{\left\{#1\right\}}
  \newcommand{\cardinal}[1]{\left|#1\right|}
  \newcommand{\defeq}{\vcentcolon=}
  \newcommand{\hypf}{h}
  \newcommand{\hypfun}[1]{\hypf\left(#1\right)}
  \newcommand{\hyp}[2]{\hypf\left(#1, #2\right)}
  \newcommand{\fun}[1]{f\left(#1\right)}
  \newcommand{\den}[1]{p\left( #1 \right)}
  \newcommand{\opt}[1]{{#1}^*}
  \newcommand{\cond}[2]{P\left( #1 \;\middle\vert\; #2 \right)}
  \newcommand{\normal}[1]{N \left(#1\right)}
  \newcommand{\multinormal}[1]{\mn \left(#1\right)}
  \newcommand{\tensornormal}[1]{\tn \left(#1\right)}
  \newcommand{\tendsto}[2]{\xrightarrow[#1]{} #2}
  \newcommand{\diffp}[2]{\frac{\partial #1}{\partial #2}}
  \newcommand{\optim}[1]{{#1}^*}
  \newcommand{\priv}[1]{{#1}^{\diamond}}
  
  \newcommand{\adj}[1]{{#1}^{\#}}
  
  
  \newcommand{\crefrangeconjunction}{--}
  \newcommand{\svset}{\mathcal{I}}
  
  
  
  
  \newcommand{\upper}[1]{\expandafter\MakeUppercase\expandafter{#1}}
  \newcommand{\mymat}[1]{\upper{#1}}
  \newcommand{\myvec}[1]{\bm{#1}}
  \newcommand{\fv}[1]{\myvec{#1}}
  \newcommand{\fm}[1]{\mymat{#1}}
  \newcommand{\frow}[1]{\fv{#1}}
  \newcommand{\vect}[1]{\bm{\text{vect}}\left(#1\right)}
  
  \newcommand{\trace}[1]{\Tr{\left(#1\right)}}
  \newcommand{\dotp}[2]{\bm{\left\langle} #1, #2 \bm{\right\rangle}}
  \newcommand{\ydotp}[2]{\left[ #1, #2 \right]_\mathcal{Y}}
  
  \newcommand{\ind}{\bm{1}}
  
  \newcommand{\domain}{\mathcal{D}}
  \newcommand{\cvxset}{X}
  \newcommand{\vecspace}{\reals^\dimx}
  
  \newcommand{\nsamples}{n}
  \newcommand{\dimx}{d}
  \newcommand{\vcdim}[1]{\VCdim\left(#1\right)}
  \newcommand{\vc}{d}
  \newcommand{\hplane}{w}
  \newcommand{\bias}{b}
  
  \newcommand{\ntasks}{T}
  \newcommand{\nclusters}{C}
  
  \newcommand{\npertask}{m}
  \newcommand{\idotsn}{i=1, \ldots, \nsamples}
  \newcommand{\hypspace}{\mathcal{H}}
  \newcommand{\hypspacef}{\mathbb{H}}
  \newcommand{\reals}{\mathbb{R}}
  \newcommand{\naturals}{\mathbb{N}}
  \newcommand{\param}{\alpha}
  \newcommand{\paramspace}{A}
  \newcommand{\fparam}{\beta}
  \newcommand{\fparamspace}{B}
  \newcommand{\hilbertspace}{\mathcal{H}}
  \newcommand{\lagr}{\mathcal{L}}
  \newcommand{\grad}{\nabla}
  
  
  
  \newcommand{\lossf}{\ell}
  \newcommand{\loss}[2]{\lossf\left( #1, #2\right)}
  
  \newcommand{\distf}{P}
  \newcommand{\sample}{D}
  \newcommand{\samplen}{D_{\nsamples}}
  
  \newcommand{\err}{e}
  \newcommand{\risk}{R}
  \newcommand{\emprisk}{\hat{\risk}_{\sample}}
  \newcommand{\empriskn}{\hat{\risk}_{\samplen}}
  
  \newcommand{\regrisk}{\hat{\risk}_{\sample, \lambda}}
  \newcommand{\exprisk}{\risk_\distf}
  \newcommand{\hypemp}{\opt{\hypf}_\nsamples}
  \newcommand{\hypexp}{\opt{\hypf}_\distf}
  
  % bias learning
  
  \newcommand{\bprobspace}{\mathcal{P}}
  \newcommand{\bdistf}{Q}
  \newcommand{\bprobseq}{\bm{P}}
  \newcommand{\bsample}{\bm{\sample}}
  \newcommand{\bemprisk}{\hat{\risk}_{\bsample}}
  \newcommand{\bexprisk}{\risk_\bdistf}
  \newcommand{\bsetsample}{\hypspacef^{\ntasks}}
  \newcommand{\bsetdist}{\hypspacef^{*}}
  \newcommand{\capf}{C}
  \newcommand{\capacity}[2]{\capf\left(#1, #2\right)}
  \newcommand{\dist}[1]{d_{#1}}
  
  \newcommand{\bigO}[1]{O\left( #1 \right)}
  
  \newcommand{\Xspace}{\mathcal{X}}
  \newcommand{\Tspace}{\mathcal{T}}
  \newcommand{\Yspace}{\mathcal{Y}}
  \newcommand{\Fspace}{\mathcal{V}}
  \newcommand{\Vspace}{\mathcal{V}}
  
  
  \newcommand{\toprob}{\xrightarrow{P}}
  
  \newcommand{\powerset}[1]{\wp\left( #1 \right)}
  
  \newcommand{\frelf}{f}
  \newcommand{\frel}[1]{\frelf\left[ #1 \right]}
  \newcommand{\frelset}{\mathcal{F}}
  
  \newcommand{\rkhs}{\mathcal{H}}
  
  \newcommand{\E}{\mathrm{E}}
  \newcommand{\Var}{\mathrm{Var}}
  \newcommand{\Cov}{\mathrm{Cov}}
  
  
  % Tables
  \newcommand{\fcode}[1]{\texttt{#1}}
  \newcommand{\fheadmulti}[2]{\multicolumn{#1}{c}{\boldmath\textbf{#2}}}
  \newcommand{\fhead}[1]{\fheadmulti{1}{#1}}
  \newcommand{\fmax}[1]{{\num[detect-weight=true,math-rm=\mathbf]{#1}}}
  \newcommand{\fmaxn}[1]{\textbf{#1}}
  \newcommand{\fdata}[1]{\textsf{#1}}
  \newcommand{\fmod}[1]{\textsf{#1}}
  \newcommand{\ftt}[1]{\text{#1}}
  
  % energies
  \newcommand{\fmodt}[2]{\fmod{(#1)}\_\fmod{#2}}
  
  % Units
  \DeclareSIUnit\utc{UTC}
  \newcommand{\utc}[1]{\SI[parse-numbers=false]{#1}{\utc}}
  \newcommand{\mwh}[1]{\SI{#1}{\mega{\watt\hour}}}
  \newcommand{\mw}[1]{\SI{#1}{\mega\watt}}
  \newcommand{\mwhu}{\si{\mega{\watt\hour}}}
  \newcommand{\mwhusq}{\si{\mega{\watt\hour}\squared}}
  
  \newcommand{\km}[1]{\SI{#1}{\kilo\metre}}




\usefonttheme[onlymath]{serif}

\titlebackground*{assets/uambackground}

\newcommand{\hrefcol}[2]{\textcolor{cyan}{\href{#1}{#2}}}

\title{Advanced Kernel Methods for
Multi-Task Learning}
\subtitle{Tesis dirigida por José Dorronsoro y Carlos Alaíz}
%\course{Tesis dirigida por José Dorronsoro y Carlos Alaíz}
\author{\href{mailto:carlos.ruizp@uam.es}{Carlos Ruiz Pastor}}
%\IDnumber{1234567}

\begin{document}
\maketitle

\begin{frame}

Acknowledgements 1

\vspace{\baselineskip}

Acknowledgements 2

\end{frame}


\begin{frame}{Outline}{}
      \tableofcontents
\end{frame}

%%%%%%%%%%%%%%%%%%%%%%%%%%%%%%%%%%%%%%%%%%%%%%%%%%%%%%%%%%%%%%%%%%%%%%%%%%%%%%%%%%%%%
\section{Introducción}

% Qué es ML
\begin{frame}
      {Introducción al Aprendizaje Automático}

      \begin{itemize}
            \item El Aprendizaje Automático intenta automatizar el proceso de aprendizaje
            \item En el aprendizaje supervisado tenemos:
            \begin{itemize}
                  \item un espacio de entrada $\Xspace$,
                  \item un espacio de salida $\Yspace$,
                  \item y una distribución $P(x, y)$ (desconocida) sobre $\Xspace \times \Yspace$
            \end{itemize} 
            \item Dada una función $f: \Xspace \to \Yspace$, definimos una función de pérdida como
            \begin{equation}
                  \begin{aligned}
              \nonumber
              \lossf:\; &\Yspace \times \Yspace &\to &&[0, \infty) \\
              &(y, f(x)) &\to  && \lossf(y, f(x)) ,
          \end{aligned}
          \end{equation}
          tal que $\lossf(y, y) = 0$ para todo $y \in \Yspace$
            
      \end{itemize}
      
\end{frame}


\begin{frame}
      \frametitle{Loss Functions}

      \begin{itemize}
            \item In classification, with the class labels $y_i \in \set{-1, 1}$, we can use:
            \begin{equation}
                  \nonumber
                  %\label{eq:hinge_def}
                  \lossf(y, f(x)) = \pospart{1 - yf(x)} = 
                  \begin{cases}
                      0, & y f(x) \geq 1 ,\\
                      1 - y f(x), & y f(x) < 1 .
                  \end{cases}
            \end{equation}
            \item 
      \end{itemize}

\end{frame}


\begin{frame}
      {Expected Risk}
      \begin{itemize}
            \item Given a space of hypothesis $\hypspace = \set{\hyp{\cdot}{\param}, \param \in \paramspace}$
            \item Definition: Expected Risk
            \begin{equation}
                  \nonumber
                  \exprisk(\param) = \int_{\Xspace \times \Yspace} \ell(y, \hyp{x}{\param}) d\distf(x, y)
            \end{equation}
            \item Our goal is to find 
            \begin{equation}
                  \nonumber
                  \opt{\param} = \argmin_{\param \in \paramspace} \left\{ \exprisk(\param) = \int_{\Xspace \times \Yspace} \ell(y, \hyp{x}{\param}) d\distf(x, y) \right\} ,
              \end{equation}
            however the distribution $\distf(x, y)$ is unknown
      \end{itemize}
      
\end{frame}


\begin{frame}
      \frametitle{Empirical Risk}

      \begin{itemize}
            \item Instead, we have a set of $\nsamples$ instances sampled from $\distf(x,y)$:
            \begin{equation}
                  \nonumber
                  \samplen = \set{(x_i, y_i), \; i=1, \ldots, \nsamples} ,
              \end{equation}
            \item Definition: Empirical Risk
            $$ \empriskn(\param) = \frac{1}{\nsamples} \sum_{i=1}^\nsamples \ell(y_i, \hyp{x_i}{\param}) $$
            \item Instead of the Expected Risk, we minimize this empirical risk:
            \begin{equation}
                  \nonumber
                  \argmin_{\param \in \paramspace} \left\{ \emprisk(\param) = \frac{1}{\nsamples} \sum_{i=1}^\nsamples \ell(y_i, \hyp{x_i}{\param}) \right\}
              \end{equation}
      \end{itemize}

\end{frame}



\subsection{Multi-Task Learning}

\begin{frame}
      \frametitle{Multi-Task Learning}

      

\end{frame}

\subsection{Support Vector Machines}

%%%%%%%%%%%%%%%%%%%%%%%%%%%%%%%%%%%%%%%%%%%%%%%%%%%%%%%%%%%%%%%%%%%%%%%%%%%%%%%%%%%%%%%
\section{Una Formulación Convexa para Aprendizaje Multitarea}

\begin{frame}
    \frametitle{Formulación Aditiva}

    \begin{itemize}
        \item Una manera de implementar el MTL es combinar una parte común y otras específicas
        \item La formulación aditiva para el aprendizaje multitarea es
        \begin{equation}
            \nonumber
            h_r(\cdot) = g(\cdot) + g_r(\cdot) 
        \end{equation}
        donde
        \begin{itemize}
            \item $g(\cdot)$ es la parte común
            \item $g_r(\cdot)$ es la parte específica
        \end{itemize}
        \item Fue propuesta para SVM lineales con los modelos
        \begin{equation}
            \nonumber
            h_r(\cdot) = \dotp{w + v_r}{\cdot} + b_r
        \end{equation}
    \end{itemize}
    
\end{frame}

  

\begin{frame}
      \frametitle{Formulación Convexa}
  
      \begin{itemize}
          \item Proponemos la siguiente formulación convexa para el aprendizaje multitarea:
          \begin{equation}
              \nonumber
              h_r(\cdot) = \lambda_r g(\cdot) + (1 - \lambda_r) g_r(\cdot) ,
          \end{equation}
          con $\lambda_r \in [0,1]$.
          \item Los hiperparámetros $\lambda_r$ regulan la influencia de cada parte:
          \begin{itemize}
              \item $\lambda_1, \ldots, \lambda_\ntasks=0$: modelos independientes (ITL)
              \item $\lambda_1, \ldots, \lambda_\ntasks=1$: modelo común (CTL)
          \end{itemize}
          \item La interpretación de los hiperparámetros es más sencilla \\
            %\Mark{100}
      \end{itemize}
      
  \end{frame}


\subsection{Convex Multi-Task Learning with Kernel Methods}

\begin{frame}
      \frametitle{Formulacion Convexa con Métodos de Kernel}

      \begin{itemize}
            \item La formulación aditiva con métodos de kernel puede expresarse con los modelos:
            \begin{equation}
                  \nonumber
                  h_r(\cdot) = \left\lbrace \dotp{w}{\phi(\cdot)} + b  \right\rbrace + \left\lbrace \dotp{{v}_r}{\phi_r(\cdot)} + d_r \right\rbrace
            \end{equation}
            \item Con nuestra formulación convexa los modelos son:
            \begin{equation}
                  \nonumber
                  h_r(\cdot) = \lambda_r \left\lbrace \dotp{w}{\phi(\cdot)} + b  \right\rbrace + (1 - \lambda_r) \left\lbrace \dotp{{v}_r}{\phi_r(\cdot)} + d_r \right\rbrace
            \end{equation}
            \item Desarrollamos tres variantes de SVM:
            \begin{itemize}
                  \item L1-SVM
                  \item L2-SVM
                  \item LS-SVM
            \end{itemize}
      \end{itemize}

\end{frame}


\begin{frame}
      \frametitle{Formulación Aditiva para MTL L1-SVM}
  
      \begin{block}{Problema Primal - L1-SVM Aditiva}
          \begin{equation}\nonumber
              \begin{aligned}
              & \argmin_{w, \fv{v}, \fv{b}, \xi}
              & & {J({w}, \fv{v}, \fv{b}, \fv{\xi}) = C \sum_{r= 1}^T \sum_{i=1}^{m_r} {\xi_{i}^r} + \frac{1}{2} \sum_{r= 1}^T{\norm{{v}_r}^2} + \frac{\mu}{2} {\norm{{w}}}^2} \\
              & \text{s.t.}
              & & y_{i}^r (\dotp{w}{\phi(x_{i}^r)} + \dotp{v_r}{\phi_r(x_{i}^r)} + b_r) \geq p_{i}^r - \xi_{i}^r ,  \\
              & & & \xi_{i}^r \geq 0; \;  i=1 , \dotsc , m_r, \;  r= 1,\dotsc, T  . \\
              \end{aligned}
          \end{equation}   
      \end{block}
      \begin{itemize}
          \item El parámetro $\mu$ (junto con $C$) regula la influencia de cada parte:
          \begin{itemize}
              \item $\mu \to \infty$: modelos independientes (ITL)
              \item $C \to 0,\; \mu \to 0$: modelo común (CTL)
          \end{itemize}
      \end{itemize}
  \end{frame}
  
  
  \begin{frame}
        \frametitle{Formulación Aditiva para MTL L1-SVM}
    
        \begin{block}{Problema Dual - L1-SVM Aditiva}
              \begin{equation}\nonumber
                    \begin{aligned}
                    & \min_{\fv{\alpha}} && \Theta(\fv{\alpha}) = \frac{1}{2} \fv{\alpha}^\intercal \left(\frac{1}{\mu} \fm{Q} + \fm{K} \right) \fv{\alpha} - \fv{p} \fv{\alpha} \\
                    & \text{s.t.}
                    & & 0 \leq \alpha_i^r \leq C ; \; i=1, \ldots, \npertask_r,\;\; r=1, \ldots, \ntasks ,\\
                    & & & \sum_{i=1}^{m_r} \alpha_i^r y_i^r = 0;\;  r=1, \ldots, \ntasks . \\
                    \end{aligned}
                \end{equation}
        \end{block}
        \begin{itemize}
            \item El parámetro $\mu$ (junto con $C$) regula la influencia de cada parte:
            \begin{itemize}
                \item $\mu \to \infty$: modelos independientes (ITL)
                \item $C \to 0,\; \mu \to 0$: modelo común (CTL)
            \end{itemize}
        \end{itemize}
    \end{frame}
  
  
  
  \begin{frame}
      \frametitle{Formulación Convexa para MT L1-SVM}
  
      \begin{block}{Problema Primal - L1-SVM Convexa}
            \begin{equation}\nonumber
                  \begin{aligned}
                  & \min_{w, \fv{v}, b, \fv{d}, \fv{\xi}}
                  & & {J({w}, \fv{v}, b, \fv{d}, \fv{\xi}) = C \sum_{r= 1}^T \sum_{i=1}^{m_r} {\xi_{i}^r} + \frac{1}{2} \sum_{r= 1}^T{\norm{{v}_r}^2} + \frac{1}{2} {\norm{{w}}}^2} \\
                  & \text{s.t.}
                  & & y_{i}^r \left(\lambda_r \left\lbrace \dotp{w}{\phi(x_{i}^r)} + b  \right\rbrace + (1 - \lambda_r) \left\lbrace \dotp{{v}_r}{\phi_r(x_{i}^r)} + d_r \right\rbrace  \right) \geq p_{i}^r - \xi_{i}^r ,  \\
                  & & & \xi_{i}^r \geq 0, \;  i=1 , \dotsc , m_r, \;  r= 1,\dotsc, T  . \\
                  \end{aligned}
              \end{equation}   
      \end{block}
      \begin{itemize}
            \item Los hiperparámetros $\lambda_r$ regulan la influencia de cada parte:
            \begin{itemize}
                \item $\lambda_1, \ldots, \lambda_\ntasks=0$: modelos independientes (ITL)
                \item $\lambda_1, \ldots, \lambda_\ntasks=1$: modelo común (CTL)
            \end{itemize}
            \item El hiperparámetro $C$ no interviene en la definición de los modelos
      \end{itemize}

  \end{frame}


  \begin{frame}
      \frametitle{Formulación Convexa para MT L1-SVM}
  
      \begin{block}{Problema Dual - SVM Convexa}
            \begin{equation}\nonumber
                  \begin{aligned}
                  & \min_{\fv{\alpha}} && \Theta(\fv{\alpha}) = \frac{1}{2} \fv{\alpha}^\intercal \left(\Lambda \fm{Q} \Lambda + \left(\fm{I}_{\nsamples} - \Lambda \right) \fm{K} \left(\fm{I}_{\nsamples} - \Lambda \right) \right) \fv{\alpha} - \fv{p} \fv{\alpha} \\
                  & \text{s.t.}
                  & & 0 \leq \alpha_i^r \leq C ; \; i=1, \ldots, \npertask_r,\; r=1, \ldots, \ntasks ,\\
                  & & & \sum_{i=1}^{m_r} \alpha_i^r y_i^r = 0;\;  r=1, \ldots, \ntasks , \\
                  \end{aligned}
              \end{equation}
              donde
              \begin{equation}\nonumber
                  \Lambda = \Diag(\overbrace{\lambda_1, \ldots, \lambda_1}^{\npertask_1}, \ldots, \overbrace{\lambda_\ntasks, \ldots, \lambda_\ntasks}^{\npertask_\ntasks})
              \end{equation}
      \end{block}

  \end{frame}



  \begin{frame}
      \frametitle{Formulación Convexa para MT L1-SVM}
  
      \begin{block}{Problema Dual - SVM Convexa ($\lambda$ común)}
            \begin{equation}\nonumber
                  \begin{aligned}
                  & \min_{\fv{\alpha}} && \Theta(\fv{\alpha}) = \frac{1}{2} \fv{\alpha}^\intercal \left(\lambda^2 \fm{Q} + \left( 1 - \lambda \right)^2 \fm{K}  \right) \fv{\alpha} - \fv{p} \fv{\alpha} \\
                  & \text{s.t.}
                  & & 0 \leq \alpha_i^r \leq C ; \; i=1, \ldots, \npertask_r,\; r=1, \ldots, \ntasks ,\\
                  & & & \sum_{i=1}^{m_r} \alpha_i^r y_i^r = 0;\;  r=1, \ldots, \ntasks , \\
                  \end{aligned}
              \end{equation}
      \end{block}
      \begin{itemize}
            \item El hiperparámetro $\lambda$ regula la influencia de cada parte:
            \begin{itemize}
                \item $\lambda=0$: modelos independientes (ITL)
                \item $\lambda=1$: modelo común (CTL)
            \end{itemize}
            \item El hiperparámetro $C$ no interviene en la definición de los modelos
      \end{itemize}

  \end{frame}


\begin{frame}
      \frametitle{Proposiciones}

      \begin{proposition}[Equivalencia entre formulaciones para SVM]
            Para valores $\lambda \in (0, 1)$, la formulación aditiva con hiperparámetros $C_\text{add}, \mu$ y la formulación convexa con $C_\text{conv}$ y un $\lambda$ común, $\lambda_1, \ldots, \lambda_\ntasks = \lambda$, son equivalentes cuando
            $$C_\text{add} = (1 - \lambda)^2 C_\text{conv}, \; \mu = (1 - \lambda)^2 / \lambda^2 .$$ 
      \end{proposition}

      \begin{proposition}[Equivalencia con CTL e ITL]
            \begin{itemize}
                  \item Para $\lambda = 0$, la formulación convexa con un $\lambda$ común es equivalente a modelos independientes (ITL).
                  \item Para $\lambda = 1$ la formulación convexa con un $\lambda$ común es equivalente a un modelo común (CTL).
            \end{itemize}
      \end{proposition}

\end{frame}


\begin{frame}
      \frametitle{Formulación convexa para MT L2-SVM}

      \begin{block}{Problema Primal - MTL L2-SVM Convexa}
            \begin{equation}\nonumber
                  \begin{aligned}
                  & \argmin_{w, \fv{v}, b, \fv{d}, \xi}
                  & & {J({w}, \fv{v}, b, \fv{d}, \fv{\xi}) = \frac{C}{2} \sum_{r= 1}^T \sum_{i=1}^{m_r} ({\xi_{i}^r})^2 + \frac{1}{2} \sum_{r= 1}^T{\norm{{v}_r}^2} + \frac{1}{2} {\norm{{w}}}^2} \\
                  & \text{s.t.}
                  & & y_{i}^r \left(\lambda_r \left\lbrace \dotp{w}{\phi(x_{i}^r)} + b  \right\rbrace + (1 - \lambda_r) \left\lbrace \dotp{{v}_r}{\phi_r(x_{i}^r)} + d_r \right\rbrace  \right) \geq p_{i}^r - \xi_{i}^r ,  \\
                  \end{aligned}
              \end{equation}
      \end{block}

      

\end{frame}


% \begin{frame}
%       \frametitle{Extensiones a L2 y LS-SVM}

%       \begin{itemize}
%             \item En la SVM (L1-SVM) hemos definido los modelos:
%             \begin{equation}
%                   \nonumber
%                   h_r(\cdot) = \lambda_r \left\lbrace \dotp{w}{\phi(\cdot)} + b  \right\rbrace + (1 - \lambda_r) \left\lbrace \dotp{{v}_r}{\phi_r(\cdot)} + d_r \right\rbrace
%             \end{equation}
%       \end{itemize}

% \end{frame}


\begin{frame}
      \frametitle{Formulación Convexa para MT L2-SVM}
  
      \begin{block}{Problema Dual - MTL L2-SVM Convexa}
            \begin{equation}\nonumber
                  \begin{aligned}
                  & \min_{\alpha} && \Theta(\alpha) = \frac{1}{2} \fv{\alpha}^\intercal \left( \left\lbrace \Lambda \fm{Q} \Lambda + \left(\fm{I}_{\nsamples} - \Lambda \right) \fm{K} \left(\fm{I}_{\nsamples} - \Lambda \right) \right\rbrace + \frac{1}{C} \fm{I} \right) \fv{\alpha} - \fv{p} \fv{\alpha} \\
                  & \text{s.t.}
                  & & 0 \leq \alpha_i^r , \; i=1, \ldots, \npertask_r,\; r=1, \ldots, \ntasks, \\
                  & & & \sum_{i=1}^{m_r} \alpha_i^r y_i^r = 0, \;  r=1, \ldots, \ntasks . \\
                  \end{aligned}
              \end{equation}
      \end{block}

  \end{frame}


  \begin{frame}
      \frametitle{Formulación convexa para MT LS-SVM}

      \begin{block}{Problema Primal - MTL LS-SVM Convexa}
            \begin{equation}\nonumber
                  \begin{aligned}
                  & \argmin_{w, \fv{v}, b, \fv{d}, \xi}
                  & & {J({w}, \fv{v}, b, \fv{d}, \fv{\xi}) = \frac{C}{2} \sum_{r= 1}^T \sum_{i=1}^{m_r} \left({\xi_{i}^r}\right)^2 + \frac{1}{2} \sum_{r= 1}^T{\norm{{v}_r}^2} + \frac{1}{2} {\norm{{w}}}^2} \\
                  & \text{s.t.}
                  & & y_{i}^r \left(\lambda_r \left\lbrace \dotp{w}{\phi(x_{i}^r)} + b  \right\rbrace + (1 - \lambda_r) \left\lbrace \dotp{{v}_r}{\phi_r(x_{i}^r)} + d_r \right\rbrace  \right) = p_{i}^r - \xi_{i}^r ,  \\
                  \end{aligned}
              \end{equation}
      \end{block}

\end{frame}


\begin{frame}
      \frametitle{Formulación Convexa para MT LS-SVM}
  
      \begin{block}{Problema Dual - MTL LS-SVM Convexa}
            \begin{equation}
                  \nonumber
                  \begin{aligned}
                  \left[
                  \begin{array}{c|c|c}
                  0 & \fv{0}_\ntasks^\intercal &  \fv{y}^\intercal \Lambda \\
                  \hline
                  \fv{0}_\ntasks & \fv{0}_{\ntasks \times \ntasks} & \fm{A}^\intercal \fm{y} (I_\nsamples - \Lambda)\\
                  \hline
                  \fv{y} & \fm{y} \fm{A} & \widehat{\fm{Q}} + \frac{1}{C} \fm{I}_\nsamples
                  \end{array}
                  \right] 
                  \begin{pmatrix}
                      b \\
                      d_1 \\
                      \vdots \\
                      d_\ntasks \\
                      \fv{\alpha}
                  \end{pmatrix}
                  = 
                  \begin{pmatrix}
                      0 \\
                      \fv{0}_\ntasks \\
                      \fv{p}
                  \end{pmatrix}, 
                  \end{aligned}
              \end{equation}
      \end{block}

  \end{frame}



\subsection{Convex Multi-Task Learning with Neural Networks}
\begin{frame}
      \frametitle{Redes Neuronales MT}

      \begin{itemize}
            \item La manera más común de adaptar las redes neuronales es el \emph{hard sharing}
            \begin{itemize}
                  \item Capas ocultas compartidas por todas las tareas
                  \item Capas de salida específicas para cada tarea
            \end{itemize}
            \item El modelo se puede expresar como:
            \begin{equation}
                  \nonumber
                  %\label{eq:convexmtl_nn}
                  \begin{aligned}
                      h_r(\cdot) &=  g_r(\cdot; w_r, \Theta)
                     =  \lbrace \dotp{w_r}{f(\cdot; \Theta)} \rbrace + d_r
                  \end{aligned}
            \end{equation}
            \begin{itemize}
                  \item $w_r, d_r$ son los parámetros de las capas de salida específicas
                  \item $\Theta$ son los parámetros de las capas ocultas compartidas
            \end{itemize}
      \end{itemize}

\end{frame}

\begin{frame}

      % \begin{figure}[t!]
    \centering
    \begin{tikzpicture}
 
        % Input Layer
        \foreach \i in {1,...,\inputnum}
        {
            \node[circle, 
                minimum size = 6mm,
                fill=orange!30] (Input-\i) at (0,-\i) {};
        }
         
        
        % Hidden Layer 1
        \foreach \i in {1,...,\hiddennumhs}
        {
            \node[circle, 
                minimum size = 6mm,
                fill=teal!50,
                yshift=(\hiddennumhs-\inputnum)*5 mm
            ] (Hidden1-\i) at (2.5,-\i) {};
        }
        
        % Hidden Layer 2
        \foreach \i in {1,...,\hiddennumhs}
        {
            \node[circle, 
                minimum size = 6mm,
                fill=teal!50,
                yshift=(\hiddennumhs-\inputnum)*5 mm
            ] (Hidden2-\i) at (5,-\i) {};
        }
         
        % Output Layer
        \foreach \i in {1,...,\outputnum}
        {
            \node[circle, 
                minimum size = 6mm,
                fill=purple!50,
                yshift=(\outputnum-\inputnum)*5 mm
            ] (Output-\i) at (7.5,-\i) {};
        }
         
        % Connect neurons In-Hidden
        \foreach \i in {1,...,\inputnum}
        {
            \foreach \j in {1,...,\hiddennumhs}
            {
                \draw[->, shorten >=1pt, red!80!black, densely dashdotted] (Input-\i) -- (Hidden1-\j);   
            }
        }

        % Connect neurons Hidden-Hidden
        \foreach \i in {1,...,\hiddennumhs}
        {
            \foreach \j in {1,...,\hiddennumhs}
            {
                \draw[->, shorten >=1pt, red!80!black, densely dashdotted] (Hidden1-\i) -- (Hidden2-\j);   
            }
        }
         
        % Connect neurons Hidden-Out
        \foreach \i in {1,...,\hiddennumhs}
        {
            \foreach \j in {1}
            {
                \draw[->, shorten >=1pt, blue!80!black, dashed] (Hidden2-\i) -- (Output-\j);
            }
        }

        \foreach \i in {1,...,\hiddennumhs}
        {
            \foreach \j in {2,...,\outputnum}
            {
                \draw[->, shorten >=1pt] (Hidden2-\i) -- (Output-\j);
            }
        }
         
        % Inputs
        \foreach \i in {1,...,\inputnum}
        {            
            \draw[<-, shorten <=1pt] (Input-\i) -- ++(-1,0)
                node[left]{};
        }
         
        % Outputs
        \foreach \i in {1,...,\outputnum}
        {            
            \draw[->, shorten <=1pt] (Output-\i) -- ++(1,0)
                node[right]{$h_{\i}(\fv{x})$};
        }
         
    \end{tikzpicture}
    
%     \caption{Ejemplo de \emph{Hard Sharing} para dos tareas . 
%     %The input neurons are shown in orange, the hidden ones in cyan and the output ones in magenta.
%     %Assuming a sample belonging to task $1$ is used, the shared weights updated in training are represented in red, and in blue the updated specific weights. 
%     }
%     \label{fig:hardsharing_nn}
% \end{figure}

\end{frame}

\begin{frame}
      \frametitle{Formulación Convexa para Redes Neuronales MT}

      \begin{itemize}
            \item Proponemos la formulación convexa para redes neuronales MT, combinando:
            \begin{itemize}
                  \item Una parte común $g(\cdot; w, \Theta)$
                  \item Una parte específica $g_r(\cdot; w_r, \Theta_r)$
            \end{itemize}
            \item Los modelos son:
            \begin{equation}
                  \nonumber
                  \begin{aligned}
                      h_r(\cdot) &= \lambda_r g(\cdot; w, \Theta) + (1 - \lambda_r) g_r(\cdot; w_r, \Theta_r)
                     \\&= \lambda_r \lbrace \dotp{w}{f(\cdot; \Theta)} + b \rbrace + (1 - \lambda_r) \lbrace \dotp{w_r}{f_r(\cdot; \Theta_r)} + d_r \rbrace.
                  \end{aligned} 
              \end{equation}
              \begin{itemize}
                  \item $w, \Theta$ son los parámetros de la red común (capa de salida y ocultas)
                  \item $w_r, \Theta_r$ son los parámetros de las redes específicas (capa de salida y ocultas)

              \end{itemize}
      \end{itemize}

\end{frame}

\begin{frame}

      % \begin{figure}[t!]
    \centering
    \begin{tikzpicture}

        % Input Layer
        \foreach \i in {1,...,\inputnum}
        {
            \node[circle, 
                minimum size = \minnodesize,
                fill=orange!30] (Input_common-\i) at (0,-\i- \hiddennum) {};
        }

        \node[circle, 
            minimum size = \minnodesize,
            fill=purple!30] (Pred1) at (8,-1.2 * \hiddennum) {};

        \node[circle, 
        minimum size = \minnodesize,
        fill=purple!30] (Pred2) at (8,-2.2 * \hiddennum) {};

        \draw[->, shorten <=1pt] (Pred1) -- ++(1,0)
            node[right]{$h_1(\fv{x})$};

        \draw[->, shorten <=1pt] (Pred2) -- ++(1,0)
        node[right]{$h_2(\fv{x})$};

        %%%%%%%%%%%%%%%%%%%% Specific NN 1 %%%%%%%%%%%%%%%%%%%%%%%%%%
    %     % Input Layer
    % \foreach \i in {1,...,\inputnum}
    % {
    %     \node[circle, 
    %         minimum size = \minnodesize,
    %         fill=orange!30] (Input_sp1-\i) at (0,-\i) {};
    % }
     
    
    % Hidden Layer 1
    \foreach \i in {1,...,\hiddennum}
    {
        \node[circle, 
            minimum size = \minnodesize,
            fill=teal!50,
            yshift=(\hiddennum-\inputnum)*5 mm
        ] (Hidden1_sp1-\i) at (2,-\i) {};
    }
    
    % Hidden Layer 2
    \foreach \i in {1,...,\hiddennum}
    {
        \node[circle, 
            minimum size = \minnodesize,
            fill=teal!50,
            yshift=(\hiddennum-\inputnum)*5 mm
        ] (Hidden2_sp1-\i) at (4,-\i) {};
    }
     
    % Output Layer
    \foreach \i in {1,...,1}
    {
        \node[circle, 
            minimum size = \minnodesize,
            fill=black!50,
            yshift=(1-\inputnum)*5 mm
        ] (Output_sp1-\i) at (6,-\i) {$g_1(\fv{x})$};
    }
     
    % Connect neurons In-Hidden
    \foreach \i in {1,...,\inputnum}
    {
        \foreach \j in {1,...,\hiddennum}
        {
            \draw[->, shorten >=1pt,blue!80!black, dashed] (Input_common-\i) -- (Hidden1_sp1-\j);   
        }
    }


    % Connect neurons Hidden-Hidden
    \foreach \i in {1,...,\hiddennum}
    {
        \foreach \j in {1,...,\hiddennum}
        {
            \draw[->, shorten >=1pt,blue!80!black, dashed] (Hidden1_sp1-\i) -- (Hidden2_sp1-\j);   
        }
    }
     
    % Connect neurons Hidden-Out
    \foreach \i in {1,...,\hiddennum}
    {
        \foreach \j in {1,...,1}
        {
            \draw[->, shorten >=1pt,blue!80!black, dashed] (Hidden2_sp1-\i) -- (Output_sp1-\j);
        }
    }
     
    % % Inputs
    % \foreach \i in {1,...,\inputnum}
    % {            
    %     \draw[<-, shorten <=1pt] (Input_sp1-\i) -- ++(-1,0)
    %         node[left]{$x_{\i}$};
    % }
     
    % Outputs
    % \foreach \i in {1,...,1}
    % {            
    %     \draw[->, shorten <=1pt] (Output_sp1-\i) -- ++(2,0)
    %         node[right]{$g_1(x)$};
    % }

    \draw[->, ultra thick] (Output_sp1-1) -- (Pred1) node [midway, fill=white] {$1 - \lambda$};

    
    
    \draw[thick]     ($(Hidden1_sp1-1.north west)+(-0.5,0.15)$) rectangle ($(Output_sp1-1.south east)+(0.3,-0.45)$);
    
    
    %%%%%%%%%%%%%%%%%% Common NN %%%%%%%%%%%%%%%%%%%%%%%%%%%%%%%%%%%

    
     
    
    % Hidden Layer 1
    \foreach \i in {1,...,\hiddennum}
    {
        \node[circle, 
            minimum size = \minnodesize,
            fill=teal!50,
            yshift=(\hiddennum-\inputnum)*5 mm
        ] (Hidden1_common-\i) at (2,-\i- \hiddennum) {};
    }
    
    % Hidden Layer 2
    \foreach \i in {1,...,\hiddennum}
    {
        \node[circle, 
            minimum size = \minnodesize,
            fill=teal!50,
            yshift=(\hiddennum-\inputnum)*5 mm
        ] (Hidden2_common-\i) at (4,-\i- \hiddennum) {};
    }
     
    % Output Layer
    \foreach \i in {1,...,1}
    {
        \node[circle, 
            minimum size = \minnodesize,
            fill=black!50,
            yshift=(1-\inputnum)*5 mm
        ] (Output_common-\i) at (6,-\i- \hiddennum) {$g(\fv{x})$};
    }
     
    % Connect neurons In-Hidden
    \foreach \i in {1,...,\inputnum}
    {
        \foreach \j in {1,...,\hiddennum}
        {
            \draw[->, shorten >=1pt,red!80!black, densely dashdotted] (Input_common-\i) -- (Hidden1_common-\j);   
        }
    }

    % Connect neurons In-Hidden
    \foreach \i in {1,...,\hiddennum}
    {
        \foreach \j in {1,...,\hiddennum}
        {
            \draw[->, shorten >=1pt,red!80!black, densely dashdotted] (Hidden1_common-\i) -- (Hidden2_common-\j);   
        }
    }
     
    % Connect neurons Hidden-Out
    \foreach \i in {1,...,\hiddennum}
    {
        \foreach \j in {1,...,1}
        {
            \draw[->, shorten >=1pt,red!80!black, densely dashdotted] (Hidden2_common-\i) -- (Output_common-\j);
        }
    }
     
    % Inputs
    \foreach \i in {1,...,\inputnum}
    {            
        \draw[<-, shorten <=1pt] (Input_common-\i) -- ++(-1,0)
            node[left]{};
    }
     
    % % Outputs
    % \foreach \i in {1,...,1}
    % {            
    %     \draw[->, shorten <=1pt] (Output_common-\i) -- ++(2,0)
    %         node[right]{$g(x)$};
    % }

    \draw[->, ultra thick] (Output_common-1) -- (Pred1) node [midway, fill=white] {$\lambda$};
    \draw[->, ultra thick] (Output_common-1) -- (Pred2) node [midway, fill=white] {$\lambda$};


    \draw[thick,blue,dotted]     ($(Hidden1_common-1.north west)+(-0.5,0.15)$) rectangle ($(Output_common-1.south east)+(0.3,-0.45)$);

    %%%%%%%%%%%%%%%%%%%% Specific NN 2 %%%%%%%%%%%%%%%%%%%%%%%%%%
        % % Input Layer
        % \foreach \i in {1,...,\inputnum}
        % {
        %     \node[circle, 
        %         minimum size = \minnodesize,
        %         fill=orange!30] (Input_sp2-\i) at (0,-\i- 2*\hiddennum) {};
        % }
         
        
        % Hidden Layer 1
        \foreach \i in {1,...,\hiddennum}
        {
            \node[circle, 
                minimum size = \minnodesize,
                fill=teal!50,
                yshift=(\hiddennum-\inputnum)*5 mm
            ] (Hidden1_sp2-\i) at (2,-\i- 2*\hiddennum) {};
        }
        
        % Hidden Layer 2
        \foreach \i in {1,...,\hiddennum}
        {
            \node[circle, 
                minimum size = \minnodesize,
                fill=teal!50,
                yshift=(\hiddennum-\inputnum)*5 mm
            ] (Hidden2_sp2-\i) at (4,-\i- 2*\hiddennum) {};
        }
         
        % Output Layer
        \foreach \i in {1,...,1}
        {
            \node[circle, 
                minimum size = \minnodesize,
                fill=black!50,
                yshift=(1-\inputnum)*5 mm
            ] (Output_sp2-\i) at (6,-\i- 2*\hiddennum) {$g_2(\fv{x})$};
        }
         
        % Connect neurons In-Hidden
        \foreach \i in {1,...,\inputnum}
        {
            \foreach \j in {1,...,\hiddennum}
            {
                \draw[->, shorten >=1pt] (Input_common-\i) -- (Hidden1_sp2-\j);   
            }
        }
    
        % Connect neurons In-Hidden
        \foreach \i in {1,...,\hiddennum}
        {
            \foreach \j in {1,...,\hiddennum}
            {
                \draw[->, shorten >=1pt] (Hidden1_sp2-\i) -- (Hidden2_sp2-\j);   
            }
        }
         
        % Connect neurons Hidden-Out
        \foreach \i in {1,...,\hiddennum}
        {
            \foreach \j in {1,...,1}
            {
                \draw[->, shorten >=1pt] (Hidden2_sp2-\i) -- (Output_sp2-\j);
            }
        }
         
        % % Inputs
        % \foreach \i in {1,...,\inputnum}
        % {            
        %     \draw[<-, shorten <=1pt] (Input_sp2-\i) -- ++(-1,0)
        %         node[left]{$x_{\i}$};
        % }
         
        % Outputs
        % \foreach \i in {1,...,1}
        % {            
        %     \draw[->, shorten <=1pt] (Output_sp2-\i) -- ++(2,0)
        %         node[right]{$g_2(x)$};
        % }

        \draw  [->, ultra thick] (Output_sp2-1) -- (Pred2) node [midway, fill=white] {$1 - \lambda$};
         
        \draw[thick]     ($(Hidden1_sp2-1.north west)+(-0.5,0.15)$) rectangle ($(Output_sp2-1.south east)+(0.3,-0.45)$);

    \end{tikzpicture}
%     \caption{Ejemplo de formulación convexa con redes neuronales para dos tareas.
% 	}
%     \label{fig:convexmtl_nn}
% \end{figure}


\end{frame}

\begin{frame}
      \frametitle{Formulación Convexa para Redes Neuronales MT}

      \begin{itemize}
            \item El riesgo a minimizar en este caso es
            \begin{equation}
                  \nonumber
                  %\label{eq:regrisk_convex_nn}
                  \begin{aligned}
                      \emprisk = \sum_{r=1}^\ntasks \sum_{i=1}^{m_r} \lossf(h_r(x_i^r), y_i^r) + \frac{\mu}{2} \left( \norm{w}^2 + \sum_{r=1}^\ntasks \norm{w_r}^2 + \Omega(\Theta) + \Omega(\Theta_r)\right) .
                  \end{aligned}
            \end{equation}
            \item Se puede aplicar el descenso por gradiente con
            \begin{equation}\nonumber
                  \begin{aligned}       
                      &\nabla_{w} h_t(x_i^t)  
                      = \lambda_t  f(x_i^t, \Theta) ,
                      &&\nabla_{\Theta} h_t(x_i^t)  
                      = \lambda_t  \dotp{w}{\nabla_\Theta f(x_i^t, \Theta)} ; \\
                      &\nabla_{w_t} h_t(x_i^t)  
                      = (1 - \lambda_t)  f_t(x_i^t, \Theta) ,
                      &&\nabla_{\Theta_t} h_t(x_i^t)  
                      = (1 - \lambda_t)   \dotp{w}{\nabla_{\Theta_t} f_t(x_i^t, \Theta_t)} ; \\
                      &\nabla_{w_r} h_t(x_i^t)  
                      =  0 , 
                      &&\nabla_{\Theta_r} h_t(x_i^t)  
                      =  0 , \text{ for } r \neq t .\\
                  \end{aligned}    
              \end{equation}
            \item Los gradientes se escalan adecuadamente con $\lambda_t$ y $(1 - \lambda_t)$
      \end{itemize}

\end{frame}


\begin{frame}
      \frametitle{Formulación Convexa para Redes Neuronales MT}

      \begin{algorithm}[H]
            \caption{Pase ``forward''}
            \DontPrintSemicolon
              \KwInput{$X_\text{mb}, t_\text{mb}$ \tcp*{Minibatch data and task labels}}
              \KwOutput{$f$ \tcp*{Forward pass for the minibatch}}
              \KwData{$\lambda$ \tcp*{Parameter of convex combination}}
              \KwData{$g; g_1, \ldots, g_\ntasks$ \tcp*{Modules of the common and specific networks}}      
              \For{$x_i, t_i \in (X_\text{mb}, t_\text{mb}) $}    
                    { 
                        $f_i \gets \lambda g(x_i) + (1 - \lambda) g_{t_i}(x_i)$   \tcp*{Convex combination}
        
                    }
      \end{algorithm}
      \begin{itemize}
            \item El pase ``backward'' se hace con la diferenciación automática de \texttt{PyTorch}
      \end{itemize}

\end{frame}


\subsection{Combinación Convexa de modelos Preentrenados}

\begin{frame}
      \frametitle{Combinación Convexa de modelos Preentrenados}

      \begin{itemize}
            \item Alternativa a la formulación convexa para aprendizaje MT
            \item Consideramos la combinación convexa de
            \begin{itemize}
                  \item modelo común $g(\cdot)$ entrenado
                  \item modelos específicos $g_r(\cdot)$ entrenados
            \end{itemize}
            \item Minimizamos el riesgo eligiendo los hiperparámetros $\lambda_1, \ldots, \lambda_\ntasks$ óptimos
            \begin{equation}
                  \nonumber
                  \emprisk(\lambda_1, \ldots, \lambda_\ntasks) = \sum_{r=1}^\ntasks \sum_{i=1}^{\npertask_r} \lossf(\lambda_r g(x_i^r) + (1 - \lambda_r) g_r(x_i^r), y_i^r) ,
              \end{equation}

      \end{itemize}

\end{frame}



\begin{frame}
      \frametitle{Formulación Unificada Clasificación}

      \begin{itemize}
            \item Hinge loss (classification):
            \begin{equation}
                \nonumber%\label{eq:hingeloss_emprisk}
                \emprisk(\lambda_1, \ldots, \lambda_\ntasks) = \sum_{r=1}^\ntasks \sum_{i=1}^{\npertask_r} \pospart{1 - y_i^r \left\lbrace\lambda_r g(x_i^r) + (1 - \lambda_r) g_r(x_i^r) \right \rbrace} .
            \end{equation}
            \item Squared hinge loss (classification):
            \begin{equation}
                \nonumber%\label{eq:sqhingeloss_emprisk}
                \emprisk(\lambda_1, \ldots, \lambda_\ntasks) = \sum_{r=1}^\ntasks \sum_{i=1}^{\npertask_r} \pospart{1 - y_i^r \left\lbrace\lambda_r g(x_i^r) + (1 - \lambda_r) g_r(x_i^r) \right \rbrace}^2 .
            \end{equation}
            \item Ambas se pueden expresar como:
            \begin{equation}
                  \nonumber
                  \sum_{r=1}^\ntasks \sum_{i=1}^{\npertask_r} u(\lambda_r c_i^r + d_i^r) , \; \text{donde} \; c_i^r =  y_i^r (g_r(x_i^r) - g(x_i^r))  , \;  d_i^r =  1 - y_i^r g_r(x_i^r)
              \end{equation}
            %   donde
            %   \begin{equation}
            %       \label{eq:changevar_clas}
            %       c_i^r =  y_i^r (g_r(x_i^r) - g(x_i^r))  , \;  d_i^r =  1 - y_i^r g_r(x_i^r) ,
            %   \end{equation}
        \end{itemize}

\end{frame}


\begin{frame}
      \frametitle{Formulación Unificada Regresión}

      \begin{itemize}
            \item Absolute loss (regression):
            \begin{equation}
                \nonumber%\label{eq:absloss_emprisk}
                \emprisk(\lambda_1, \ldots, \lambda_\ntasks) = \sum_{r=1}^\ntasks \sum_{i=1}^{\npertask_r} \abs{y_i^r - \left\lbrace\lambda_r g(x_i^r) + (1 - \lambda_r) g_r(x_i^r) \right \rbrace} .
            \end{equation}
            \item Squared loss (regression):
            \begin{equation}
                \nonumber%\label{eq:sqloss_emprisk}
                \emprisk(\lambda_1, \ldots, \lambda_\ntasks) = \sum_{r=1}^\ntasks \sum_{i=1}^{\npertask_r} \left( {y_i^r - \left\lbrace\lambda_r g(x_i^r) + (1 - \lambda_r) g_r(x_i^r) \right \rbrace} \right)^2 .
            \end{equation}
            \item Ambas se pueden expresar como:
            \begin{equation}
                  \nonumber
                  \sum_{r=1}^\ntasks \sum_{i=1}^{\npertask_r} u(\lambda_r c_i^r + d_i^r) ,\; \text{donde} \; c_i^r = g(x_i^r) - g_r(x_i^r)  , \;  d_i^r =  g_r(x_i^r) - y_i^r
              \end{equation}
            % donde $c_i^r = g(x_i^r) - g_r(x_i^r)  , \;  d_i^r =  g_r(x_i^r) - y_i^r .$
            %   \begin{equation}
            %       \label{eq:changevar_reg}
            %       c_i^r = g(x_i^r) - g_r(x_i^r)  , \;  d_i^r =  g_r(x_i^r) - y_i^r .
            %   \end{equation}
        \end{itemize}

\end{frame}

\begin{frame}
      \frametitle{Formulación Unificada}

      \begin{itemize}
            \item En todos los casos tenemos que minimizar 
            \begin{equation}
                  \nonumber
                  \emprisk(\lambda_1, \ldots, \lambda_\ntasks) =
                  \sum_{r=1}^\ntasks \sum_{i=1}^{\npertask_r} u(\lambda_r c_i^r + d_i^r)
            \end{equation}
            \item Como es separable, tenemos en cada tarea el problema
            \begin{equation}
                  \nonumber
                  \argmin_{\lambda_r \in [0, 1]} \mathcal{J}(\lambda_r) = \sum_{i=1}^{\npertask_r} u(\lambda_r c_i^r + d_i^r),
            \end{equation}
            \item Usando el Teorema de Fermat
            \begin{equation}
                  \nonumber
                  \lambda^* = \argmin_{0 \leq \lambda \leq 1} \mathcal{J}(\lambda) \iff (0 \in \partial \mathcal{J}(\lambda^*) \text{ and } \lambda^* \in (0, 1) ) \text{ or } \lambda^*=0 \text{ or } \lambda^*=1 .
              \end{equation}
      \end{itemize}

\end{frame}


\begin{frame}
      \frametitle{Combinación Convexa con Error Cuadrático}

      \begin{itemize}
            \item La función a minimizar es 
            \begin{equation}
                  \nonumber
                  %\label{eq:opt_sq}
                  \argmin_{\lambda \in [0, 1]} \mathcal{J}(\lambda) = \sum_{i=1}^{\npertask} \left({\lambda c_i + d_i}\right)^2 .
            \end{equation}
            \item La derivada es
            \begin{equation}
                  \nonumber
                  \mathcal{J}'(\lambda) = \sum_{i=1}^\npertask 2 c_i (\lambda c_i + d_i) .
            \end{equation}
            \item Como es derivable, resolviendo $\mathcal{J}'(\lambda)= 0$ obtenemos
            \begin{equation*}
                  \lambda' =  -\frac{\sum_{i=1}^{\npertask} d_i c_i }{\sum_{i=1}^{\npertask} (c_i)^2 } .
                  \end{equation*}
            \item La solución es entonces $\opt{\lambda} = \max(\min(\lambda', 1), 0)$
      \end{itemize}
        %
        
\end{frame}


\begin{frame}

      \begin{figure}[t!]
            \centering
            \includegraphics[width=.6\textwidth]{Chapter4/NeuroCom2021/ejemplo2_mse.pdf}
            \caption{Error using the squared loss function (top) and its corresponding derivative (bottom). The green line represents the $0$ constant function. The yellow dot is the point minimizing the error, and whose corresponding derivative contains the value $0$.}
            \label{fig:sq_error}
        \end{figure}      

\end{frame}



\begin{frame}
      \frametitle{Combinación Convexa con Error Absoluto}

      \begin{proposition}[$\lambda^*$ óptimo para el problema con valor absoluto]\label{prop:abs_neurocom2020}
            \begin{itemize}
                  \item $\lambda^*=0$ es óptimo si y solo si: $- \sum_{i: \; 0 > \lambda_{(i)}} \abs{c_{(i)}} + \sum_{i: \; 0 < \lambda_{(i)}} \abs{c_{(i)}} \leq 0$
                  \item $\lambda^* \in (0,1)$ es óptimo si y solo si $0 < \lambda^* = \lambda_{(k)} < 1$ para algún $k=1, \dotsc, \npertask$, y
                  \begin{equation}
                  \nonumber    
                  - \sum_{i:\; \lambda_{(k)} > \lambda_{(i)}} \abs{c_{(i)}} + \sum_{i:\; \lambda_{(k)} < \lambda_{(i)}} \abs{c_{(i)}} \in \left[ -  \abs{c_{(k)}},  \abs{c_{(k)}}  \right] 
                  \end{equation}
                  \item $\lambda^*=1$ es óptimo en otro caso
            \end{itemize}
        \end{proposition}
\end{frame}


\begin{frame}

      \begin{figure}[t!]
            \centering
            \includegraphics[width=.6\textwidth]{Chapter4/NeuroCom2021/ejemplo2_mae.pdf}
            \label{fig:sq_error}
        \end{figure}      

\end{frame}




\begin{frame}
      \frametitle{Combinación Convexa con Error Hinge}
      \begin{proposition}[$\lambda^*$ óptimo para el problema con error hinge]\label{prop:hinge_neurocom2020}
            \begin{itemize}
                  \item $\lambda^*=0$ es óptimo si y solo si: $-\sum_{i:\; 0 > \lambda_{(i)}} \mymax{0, c_{(i)}} - \sum_{0 < \lambda_{(i)}} \mymin{0, c_{(i)}} \leq 0 $
                  \item $\lambda^* \in (0,1)$ es óptimo si y solo si $0 < \lambda^* = \lambda_{(k)} < 1$ para algún $k=1, \dotsc, \npertask$, y
                  \begin{equation}
                        \nonumber%\label{eq:sol_hinge}
                        -\sum_{i:\; \lambda_{(k)} > \lambda_{(i)}} \mymax{0, c_{(i)}} - \sum_{i:\; \lambda_{(k)} <\lambda_{(i)}} \mymin{0, c_{(i)}} \in \left[\mymin{0, c_{(k)}}, \mymax{0, c_{(k)}} \right] 
                  \end{equation}
                  \item $\lambda^*=1$ es óptimo en otro caso
            \end{itemize}
        \end{proposition}
      % \begin{proposition}[Optimal $\lambda^*$ with Hinge Loss]\label{prop:hinge_neurocom2020}
      %       In~\eqref{eq:opt_hinge_l1}, $\lambda^*=0$ is optimal iff
      %       \begin{equation}
      %           \nonumber
      %           %\label{eq:sol_hinge_0}
      %           -\sum_{i:\; 0 > \lambda_{(i)}} \mymax{0, c_{(i)}} - \sum_{0 < \lambda_{(i)}} \mymin{0, c_{(i)}} \leq 0 .
      %           \end{equation}
      %           If this condition does not hold, a value $\lambda^* \in (0, 1)$ is optimal for problem~\eqref{eq:opt_hinge_l1} \emph{iff} $\lambda^*$ is a feasible elbow, that is, $0 < \lambda^* = \lambda_{(k)} < 1$ for some $k=1, \dotsc, \npertask$, and
      %       \begin{equation}
      %           \label{eq:sol_hinge}
      %           -\sum_{i:\; \lambda_{(k)} > \lambda_{(i)}} \mymax{0, c_{(i)}} - \sum_{i:\; \lambda_{(k)} <\lambda_{(i)}} \mymin{0, c_{(i)}} \in \left[\mymin{0, c_{(k)}}, \mymax{0, c_{(k)}} \right] .
      %       \end{equation}
      %       If none of the previous conditions hold, then $\lambda^*=1$ is optimal.
      %   \end{proposition}
\end{frame}


\begin{frame}

      \begin{figure}[t!]
            \centering
            \includegraphics[width=.6\textwidth]{Chapter4/NeuroCom2021/ejemplo2_hinge.pdf}
            \label{fig:sq_error}
        \end{figure}      

\end{frame}


\begin{frame}
      \frametitle{Combinación Convexa con Error Hinge Cuadrático}


      \begin{proposition}[$\lambda^*$ óptimo para el problema con error hinge cuadrático]\nonumber%\label{prop:hinge_neurocom2020}
            \begin{itemize}
                  \item $\lambda^*=0$ es óptimo si y solo si: $-\sum_{i:\; 0 > c_{(i)}, 0 < \lambda_{(i)}} {2 c_i d_i} - \sum_{i:\; 0 < c_{(i)}, 0 > \lambda_{(i)}} {2 c_i d_i}  \leq 0 $
                  \item $\lambda^* \in (0,1)$ es óptimo si y solo si $0 < \lambda^* = \widehat{\lambda}_{(k)} < 1$ para algún $k=1, \dotsc, \npertask$, donde
                  \begin{equation}\nonumber%\label{eq:sol_hinge_2}
                        \widehat{\lambda}_{(k)} = - \frac{\sum_{i:\; \lambda_{(k+1)} \geq \lambda_{(i)}} \mymax{0, c_{(i)}} d_{(i)} + \sum_{i:\; \lambda_{(k)} \leq \lambda_{(i)}} \mymin{0, c_{(i)}} d_{(i)}}{\sum_{i:\; \lambda_{(k+1)} \geq \lambda_{(i)}} \mymax{0, c_{(i)}}^2 + \sum_{i:\; \lambda_{(k)} \leq \lambda_{(i)}} \mymin{0, c_{(i)}}^2} ,
                  \end{equation}
                  y además $\lambda_{(k)} \leq \widehat{\lambda}_k \leq  \lambda_{(k+1)}$
                  \item $\lambda^*=1$ es óptimo en otro caso
            \end{itemize}
        \end{proposition}

      % \begin{proposition}[Optimal $\lambda^*$ with Squared Hinge Loss]\label{prop:sqhinge_neurocom2020}
      %       In~\eqref{eq:opt_hinge_l2},
      %       $\lambda^*=0$ is optimal iff 
      %       \begin{equation}\nonumber
      %           -\sum_{\substack{i:\; 0 > c_{(i)},\\ \;\; 0 < \lambda_{(i)}}} {2 c_i d_i} - \sum_{\substack{i:\; 0 < c_{(i)},\\ \;\; 0 > \lambda_{(i)}}} {2 c_i d_i}  \leq 0 .
      %          \end{equation}
      %       If this condition does not hold, 
      %       consider the sorted list of elbows $\lambda_{(1)}, \ldots, \lambda_{(\npertask)}$; then, for each interval between elbows $(\lambda_{(k)}, \lambda_{(k+1)})$, we define the value $\widehat{\lambda}_k$ for $k=1, \ldots,  \npertask-1$ as %
      %   \begin{equation}\label{eq:sol_hinge_2}
      %       \widehat{\lambda}_{(k)} = - \frac{\sum_{i:\; \lambda_{(k+1)} \geq \lambda_{(i)}} \mymax{0, c_{(i)}} d_{(i)} + \sum_{i:\; \lambda_{(k)} \leq \lambda_{(i)}} \mymin{0, c_{(i)}} d_{(i)}}{\sum_{i:\; \lambda_{(k+1)} \geq \lambda_{(i)}} \mymax{0, c_{(i)}}^2 + \sum_{i:\; \lambda_{(k)} \leq \lambda_{(i)}} \mymin{0, c_{(i)}}^2} .
      %   \end{equation}
      %   %
      %   If $\widehat{\lambda}_k$ satisfies that $\lambda_{(k)} \leq \widehat{\lambda}_k \leq  \lambda_{(k+1)}$ and $0 \leq \widehat{\lambda}_{(k)} \leq 1$ for some $k=1, \ldots, \npertask$ then $\lambda^* = \widehat{\lambda}_k$ is optimal.
      %   Finally, if none of the previous conditions holds, \eqref{eq:opt_hinge_l2} has a minimum at $\lambda^* = 1$.
      %   \end{proposition}
\end{frame}

\begin{frame}

      \begin{figure}[t!]
            \centering
            \includegraphics[width=.6\textwidth]{Chapter4/NeuroCom2021/ejemplo2_sqhinge.pdf}
            \label{fig:sq_error}
        \end{figure}      

\end{frame}




%%%%%%%%%%%%%%%%%%%%%%%%%%%%%%%%%%%%%%%%%%%%%%%%%%%%%%%%%%%%%%%%%%%%%%%%%%%%%%%%%%%%%%%%
\section{Laplaciano Adaptativo para Aprendizaje Multitarea}

\begin{frame}
      \frametitle{Aprendizaje Multitarea con Regularización Laplaciana}

      \begin{itemize}
            \item Otra manera de acoplar distintas tareas es usar una regularización Laplaciana
            \item Consideramos un grafo donde
            \begin{itemize}
                  \item Los nodos representan tareas
                  \item Las aristas y sus pesos representan las relaciones entre las tareas
            \end{itemize}
            \item La matriz de adyacencia $A$ tiene los pesos de las aristas
            \item La matriz de grados $D$ es una matriz diagonal donde
            $$ (D)_{rr} = \sum_{s=1}^\ntasks (A)_{rs}$$
            \item La matriz Laplaciana se define como $L = D - A$
      \end{itemize}
      
\end{frame}


\begin{frame}
      \frametitle{Aprendizaje Multitarea con Regularización Laplaciana}

      \begin{itemize}
            \item Dados los modelos para cada tarea definidos como
            $$ \hypf_r(\cdot) = \dotp{w_r}{\cdot} + b_r$$
            \item Definimos la regularización
            \begin{equation}
                  \nonumber
                  %\label{eq:gl_regularization}
                  \sum_{r=1}^\ntasks \sum_{s=1}^\ntasks (A)_{rs} \norm{w_r - w_s}^2 ,
              \end{equation}
            \item Esta regularización se puede expresar como
            \begin{equation}
                  \nonumber
                  %\label{eq:gl_regularization}
                  \sum_{r=1}^\ntasks \sum_{s=1}^\ntasks (A)_{rs} \norm{w_r - w_s}^2 = \sum_{r=1}^\ntasks \sum_{s=1}^\ntasks (L)_{rs} \dotp{w_r}{w_s} ,
              \end{equation}
      \end{itemize}
      
\end{frame}


\subsection{Laplaciano de Grafo con Métodos de Kernel}

\begin{frame}
      \frametitle{Laplaciano de Grafo con Métodos de Kernel}

      \begin{itemize}
            % \item La regularización Laplaciana penaliza las distancias
            % \item ¿Cómo calcular las distancias en un Espacio de Hilbert con Kernel Reproductor?
            \item Consideramos el problema de minimización
            \begin{equation}
                  \label{eq:mtl_kernel_altext_original}
                  \begin{aligned}
                       & R({u_1, \ldots, u_T}) = \sum_{r=1}^{\ntasks} \sum_{i=1}^{\npertask_r} \lossf(y_i^r, \dotp{u_r}{\phi(x_i^r)}) + \mu \sum_r \sum_s (E)_{rs} \dotp{u_r}{u_s} , \\
                  \end{aligned}
            \end{equation}
            \item Si usamos el vector $\fv{u}^\intercal = (u_1^\intercal, \ldots, u_\ntasks^\intercal)$ lo expresamos como
            \begin{equation}
                  \label{eq:mtl_kernel_altext_tensor}
                  \begin{aligned}
                          & R(\myvec{u}) = \sum_{r=1}^{\ntasks} \sum_{i=1}^{\npertask_r} \lossf(y_i^r, \dotp{\myvec{u}}{e_r \otimes \phi(x_i^r)}) + \mu \left(  \myvec{u}^\intercal (E \otimes I) \myvec{u} \right). \\
                  \end{aligned}
              \end{equation}
      \end{itemize}

\end{frame}

\begin{frame}
      \frametitle{Laplaciano de Grafo con Métodos de Kernel}

      \begin{lemma}\label{lemma:regproblems_kernel}
            Las soluciones $u_1^*, \ldots, u_\ntasks^*$ de~\eqref{eq:mtl_kernel_altext_original}, o equivalentemente la solución $\opt{\fv{u}}$ de~\eqref{eq:mtl_kernel_altext_tensor},    
            se pueden obtener minimizando
            \begin{equation}
                \label{eq:mtl_kernel_tensor}
                \begin{aligned}
                     & S(\myvec{w}) = \sum_{r=1}^{\ntasks} \sum_{i=1}^{\npertask_r} \lossf(y_i^r, \dotp{\myvec{w}}{(B_r \otimes \phi(x_i^r))}) + \mu  \myvec{w}^\intercal \myvec{w} , \\
                \end{aligned}
            \end{equation}
            donde $\bm{w} \in \reals^p \otimes \hilbertspace$ con $p \geq \ntasks$ y $B_r$ son las columnas de $B \in \reals^{p \times \ntasks}$, una matriz de rango máximo tal que $\mymat{E}^{-1} = \mymat{B}^\intercal \mymat{B}$.
        \end{lemma}
        El kernel reproductor correspondiente es:
        \begin{equation}
            \nonumber
            \dotp{B_r \otimes \phi(x_i^r)}{B_s \otimes \phi(x_j^s)} = \left(E^{-1} \right)_{rs} k(x_i^r, x_j^s) 
        \end{equation}



\end{frame}

\subsection{Adaptive Graph Laplacian Algorithm}


\section{Summary}

\begin{frame}
\frametitle{Good Luck!}
\begin{itemize}
\item Enough for an introduction! You should know enough by now
\end{itemize}
\end{frame}

\backmatter
\end{document}
